\chapter{运动与力} 
\thispagestyle{empty}
\section{牛顿运动定律} 
\dy[牛顿第一定律]{NDYDDL} 任何物体都保持静止的或沿一条直线
\margin{牛顿第一定律引出了惯性和力的概念以及惯性参考系的定义}
作匀速运动的状态,除非作用在它上面的例迫使它改改变这种状态.\jg 
\par \quad \dy[惯性]{GX} 物体保本身要保持运动状态不变的性质,或物体抵抗运动变化的性质.\jg  
\par \quad \dy[惯性参考系]{GXCKX} 在一个参考系中观测,如果一个不受力作用的物体保持不变,这一参考系就叫做惯性参考系.\jg \jg

\par \dy[牛顿第二定律]{NDDEDL} 运动的变化与所加的动力成正比,并且发生在这力所沿的直线上.\jg
\begin{equation}
	\bm{F}=\frac{\d \bm{p}}{\d t}=\frac{\d }{\d t}(m\bm{v})
\end{equation}
其中$\bm{p}=m\bm{v}$为质点的动量,当质点速度$\bm{v}$远小于光速$c$时,质点的质量近似与速度无关,即
\begin{equation}
\bm{F}=m\frac{\d v}{\d t}=m\bm{a}
\end{equation}

\par \dy[力的叠加原理]{LDDJYL}\jg
\begin{equation}
\bm{F}=\sum\bm{F}_i
\end{equation}
 上述牛顿第二定律公式中的$\bm{F}$是作用在质点上所有力的合力.\jg \jg
 
 \par \dy[牛顿第三定律]{NDDSDL} 两个物体之间的相互作用力同时存在,分别作用,方向相反,大小相等
 \begin{equation}
 \bm{F}_{12}=-\bm{F}_{21}
 \end{equation}
\par 牛顿定律仅在惯性系中成立.

\section{常见的几种力}
\dy[重力]{ZL}\jg
\begin{equation}
\bm{P}=m\bm{g}
\end{equation}
\par \dy[弹力]{TL}
发生形变而要恢复原状所产生的力,主要有以下三种:\jg\jg
\par \quad \dy[支持力]{ZCL} 与垂直于接触面的作用力相等(压力).\jg
\par \quad \dy[张力]{ZL} 拉紧的绳或线对被拉的物体有拉力.\jg
\par \quad \dy[胡克定律]{HKDL} 在弹性限度内,弹簧的弹力和形变成正比
\margin{\\[1.5em]\kg$k$称为弹簧的劲度系数.}
\begin{equation}
\eq[f=-kx]
\end{equation} 

\par \dy[摩擦力]{MCL}\jg\jg

\par \quad \dy[滑动摩擦力]{HDMCL}
\margin{\\ \kg $\mu_k$ 是滑动摩擦系数.}
\begin{equation}
f_k=\mu_kN
\end{equation}

\par \quad \dy[静摩擦力]{JMCL}
\margin{\\ \kg $\mu_s$ 是静摩擦系数.}
\begin{equation}
f_s \le f_{s\,\text{max}}=\mu_sN
\end{equation}

\par \dy[流体阻力]{LTZL} 与流体中的物体相对与流体的速度方向相反,当二者的相对速度较小时,
\begin{equation}
f_d=kv
\end{equation}
\par 当二者的相对速度较大以致于后方出现漩涡时,
\margin{\\\kg$C$ 为曳引系数.\\\kg$\rho$为空气密度.\\\kg$A$为有效横截面积.}
\begin{equation}
f_d=\frac{1}{2}C\rho A v^2
\end{equation}

\par \dy[表面张力]{BMZL} 液体表面总处于一种绷紧状态.这是由于液面各部分之间存在着相互拉紧的力.
\margin{$\gamma$为表面张力系数.\\\kg$l$为边界线长度.}
\begin{equation}
F=\gamma l
\end{equation}

\section{惯性力}
\dy[惯性力]{GXL} 在非惯性系中引入的和参考系本身的加速运动相联系的力.\jg\jg
\margin{潮汐就是地球上观察到的一种惯性力作用的表现.}
\par \quad \dya[平动加速参考系]
\begin{equation}
\bm{F}_i=-m\bm{a}_0
\end{equation}

\par \quad \dya[转动参考系]
\begin{equation}
\bm{F}_i=m\omega^2\bm{r}
\end{equation}

\par \quad \dya[科里奥利力]
\begin{equation}
\bm{F}_C=2m\bm{v'}\times \bm{\omega }
\end{equation}

