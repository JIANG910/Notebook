%模板
\documentclass[10pt,a4paper,twoside]{book}

\usepackage{ctex}
\usepackage{makecell}
\usepackage{enumerate}%罗列专用宏包
\usepackage{graphicx}%插入图片的宏包
\usepackage{subfigure}
\usepackage{newtxtext}
\usepackage{newtxmath}
\usepackage{bm}
\DeclareMathSizes{10}{10}{5.5}{4}
\usepackage{makeidx}%索引专用
\makeindex  %添加索引
\usepackage{fancyhdr}
%\usepackage{textcomp}%树叶图案在这个包里
%\usepackage{bbding}%很多漂亮的图案
\usepackage[dvipsnames, svgnames, x11names]{xcolor}%导入了所有颜色配置文件的宏包
\usepackage{geometry}%页边距调整
\geometry{left=2cm,right=2cm,bottom=2cm,top=2cm}
\usepackage{titletoc}%目录页的宏包
\usepackage{titlesec}%改变章节或标题的样式的宏包
\usepackage[bookmarks=true,colorlinks,linkcolor=black]{hyperref}
\usepackage{enumerate}%使用改宏包优化罗列环境
\usepackage{tcolorbox}%box宏包
\tcbuselibrary{most}
\usepackage{xcolor}
\usepackage{colortbl,booktabs}%第二个包定义了几个*rule  
\usepackage{multicol}
\usepackage{multirow}
\usepackage{tikz}
\usepackage{capt-of}
%\usepackage{longtable}
%\usepackage{polynom}% 除法竖式
\usetikzlibrary{shapes.geometric}
\usetikzlibrary{arrows,arrows.meta}
%\usetikzlibrary{circuits.ee.IEC}

%字体设置
\setCJKmainfont[BoldFont={PingFangSC-Medium}]{PingFangSC-Regular}

%章节或标题的样式
\titleformat{\chapter}{\bfseries\Huge\color{titlepurple}}{第\ \thechapter\ 章\ \quad}{0pt}{}
\titleformat{\section}{\Large\color{titlepurpleb}}{\bfseries{\thesection}\quad  }{0pt}{}
\titleformat{\subsection}{\large\color{titlepurplec}}{\bfseries{\thesubsection}\quad  }{0pt}{}
\titlespacing{\subsection}{1.5em}{0.1em}{1em}[1em]
%格式如下:\titlespacing*{章节名称}{左间距}{(前)行间距}{(后)行间距}[右间距(一般都没用,填0.1em即可,但不能不填)]
\titlespacing*{\subsubsection}{2em}{3em}{1em}[1em]

%目录调整
\newcounter{mycontents}
\newcommand{\thecontents}{\refstepcounter{mycontents} \alph{mycontents}.}
%\titlecontents{标题名}[左间距]{标题格式}{标题标志}{无序号标题}{指引线与页码}[下间距]
\titlecontents{chapter}
[0cm]
{\bf \large \vspace{0.8em} }{\contentspush{第 \thecontentslabel\ 章 \hspace*{0.8em}}}{}{\titlerule*[0.5pc]{$\cdot$}\contentspage}
\titlecontents{section}[1.7cm]{\bf  \vspace{0.5em} }{\contentslabel{2.4em}}{\hspace*{-2.5em} \thecontents \hspace*{0.8em}}{\titlerule*[0.5pc]{$\cdot$}\contentspage}
\titlecontents{subsection}[2.5cm]{\small \vspace{0.2em} }{\contentslabel{3em}}{}{\titlerule*[0.5pc]{$\cdot$}\contentspage}

%定义颜色
%定义某个颜色,对应颜色代号查表
\definecolor{titlepurple}{HTML}{5758BB}%一级标题(目前:蓝紫色)
\definecolor{titlepurpleb}{HTML}{3A006F}%二级标题(目前:深紫色)
\definecolor{titlepurplec}{HTML}{006266}%三级标题(目前:墨绿色)
\definecolor{tab1}{HTML}{9698ED}%表格1
\definecolor{tab2}{HTML}{DBDCFF}%表格2
\definecolor{dy0}{HTML}{EA7500}%小标题定义专用(目前:橙黄色)
\definecolor{dl}{HTML}{007500}%小标题定理专用(目前:深绿色)
\definecolor{inference}{HTML}{343300}%小标题推论专用(目前:墨绿色)
\definecolor{ex}{HTML}{7158e2}%小标题例专用(目前:紫色)
\definecolor{dy}{HTML}{BF0060}%夹杂在文本中的定义词的颜色1(目前:深红色)
\definecolor{dy2}{HTML}{FF0000}%夹杂在文本中的定义词的颜色2(目前:红紫色)
\definecolor{dya}{HTML}{FFFFFF}
\definecolor{超链接}{HTML}{0000C6}%含超链接的文字专用色(目前:蓝紫色)
\definecolor{文字底色}{HTML}{F8FF00}%强调的文字底色(目前:黄色)
\definecolor{eq}{HTML}{F0F0F0}
\definecolor{tl}{HTML}{D94600}


%定义计数器
\newcounter{theorem}[chapter]
\newcounter{defination}[chapter]
\newcounter{example}[chapter]
\newcounter{inference}[chapter]
\newcounter{examples}[chapter]
\newcounter{tl}[chapter]
\newcounter{F}[section]
\newcounter{G}[section]
\newcounter{H}[section]
\renewcommand{\thetheorem}{{ 定理} \textbf{\thechapter.\arabic{theorem}}}
\renewcommand{\thedefination}{{ 定义} \textbf{\thechapter.\arabic{defination}}}
\renewcommand{\theexample}{{ 题型} \textbf{\thechapter.\arabic{example}}}
\renewcommand{\theinference}{{ 方法} \textbf{\thechapter.\arabic{inference}}}
\renewcommand{\theexamples}{{ 例}  \textbf{\thechapter.\arabic{examples}}}
\renewcommand{\thetl}{{ 推论}  \textbf{\thechapter.\arabic{tl}}}
\newcommand{\s}{\hspace*{-2.7em}}

%定义环境
\newcommand{\mybox}[2][]{
	\begin{tcolorbox}[on line,
		arc=0pt,outer arc=0pt,colback=#1!10!white,colframe=#1,
		boxsep=0pt,left=3pt,right=3pt,top=6pt,bottom=6pt,
		boxrule=0pt,leftrule=1.5pt]#2
\end{tcolorbox}}

%定理类
\newcommand{\theorem}[2][]{\vspace{1em}\s \refstepcounter{theorem} \mybox[dl]{{\color{dl}\thetheorem\hspace{1em}#1}\\[0.1em] \hspace*{2em}#2}\vspace{0.5em}  \par}

%推论类
\newcommand{\inference}[2][]{\vspace{1em}\s \refstepcounter{inference} \mybox[inference]{{\color{inference}\theinference\hspace{1em}#1}\\ \hspace*{1.5em}#2}\vspace{0.5em}   \par}

%定义类
\newcommand{\defination}[2][]{\vspace{1em}\s \refstepcounter{defination} \mybox[dy0]{{\color{dy0}\thedefination\hspace{1em}#1}\\[0.1em] \hspace*{2em}#2}\vspace{0.5em} \par}

%题型类(无标签)
\newcommand{\example}[1][]{\vspace{1em} \s \refstepcounter{example} \mybox[ex]{\color{ex}\theexample\hspace{1em}#1}\vspace{0.5em} \par }

%\theoremstyle{break}
%\theoremindent0.2cm
%\newtheorem*{theorem}{\hspace{0.2cm}\color{dl}\label{#1} \mybox[dl]{\color{dl}定理\addtocounter{A}{1} \thesection.\arabic{A}}}
%\newtheorem*{defination}{\hspace{0.2cm}\color{dy0}\label{#1} \mybox[dy0]{\color{dy0}定义\addtocounter{B}{1} \thesection.\arabic{B}}}
%\newtheorem*{feature}{\hspace{-0.16cm}\color{ffa725}\label{#1} \mybox[ffa725]{\color{ffa725}性质\addtocounter{C}{1} \thesection.\arabic{C}}}
%\newtheorem*{inference}{\hspace{-0.16cm}\color{1a9850}\label{#1} \mybox[1a9850]{\color{1a9850}推论\addtocounter{D}{1} \thesection.\arabic{D}}}
%\newtheorem*{method}{\hspace{-0.16cm}\color{6a3d9a}\label{#1} \mybox[6a3d9a]{\color{6a3d9a}方法\addtocounter{E}{1} \thesection.\arabic{E}}}
%\newtheorem*{example}{\hspace{-0.16cm}\color{53a9ab}\label{#1} \mybox[53a9ab]{\color{53a9ab}例题\addtocounter{F}{1} \thesection.\arabic{F}}}


%调整间距(倍数)
\linespread{1.5}

%自定义页眉页脚---------------
\pagestyle{fancy}
\renewcommand{\chaptermark}[1]{\markboth{\;第\ \thechapter\ 章\quad#1\;}{}}
\renewcommand{\sectionmark}[1]{\markright{\;\thesection\ #1\;}}
\fancyhf{}
%\fancyfoot[C]{\bfseries\thepage}
\fancyhead[LO]{\small\CJKfamily{song}\rightmark}
\fancyhead[RE]{\small\CJKfamily{song}\leftmark}
\fancyhead[RO,LE]{\;\thepage\;}
\fancyfoot[RO,LE]{\footnotesize\CJKfamily{heilight}{飞行器结构力学}}
\fancyfoot[RE,LO]{\footnotesize\CJKfamily{heilight}Aircraft Structural Mechanics}
\renewcommand{\headrulewidth}{0.4pt} % 注意不用\setlength
%\renewcommand{\footrulewidth}{0pt}
\fancyheadoffset[LE,RO]{0cm}
\fancyfootoffset[LE,RO]{0cm}
% 注意不用\setlength
%\renewcommand{\footrulewidth}{0pt}

%自定义命令
%% 文本设置类
\newcommand{\link}[1][]{\hyperref[#1] {\color{超链接}#1}}
\newcommand{\ds}[1][]{\colorbox{文字底色}{#1}}
\newcommand{\red}[1][]{\textcolor{red}{#1}}
\newcommand{\blue}[1][]{\textcolor{blue}{#1}}

%% 公式字符类
\renewcommand{\d}{{\rm{d}}}
\newcommand{\e}{{\rm{e}}}
\newcommand{\n}{{\rm{n}}}
\renewcommand{\t}{\text{t}}
\renewcommand{\j}{\text{j}}
\def\degree{{}^{\circ}}
\newcommand{\hvdots}{\hspace*{2mm}\vdots\hspace*{2mm}}
\newcommand{\RMn}[1][]{\romannumeral#1}
\newcommand{\RMN}[1][]{\uppercase\expandafter{\romannumeral#1}}
\newcommand{\vi}{\bm{i}}
\newcommand{\vj}{\bm{j}}
\newcommand{\vk}{\bm{k}}

%% 定义索引类
\newcommand{\dy}[2][]{{\color{dy}#1}\index{#2@#1}}
\newcommand{\dya}[2][]{\vspace*{0.7em} \noindent \tcbox[colframe =Chocolate , colback =Coral,boxrule=0.5mm,size=small,on line]{\color{dya}{\textbf{#1}}}  \index{#2@#1} \hspace*{1em}}
\newcommand{\dyb}[1][]{\vspace*{0.7em} \noindent \tcbox[colframe =Chocolate, colback =Coral,boxrule=0.5mm,size=small,on line]{\color{dya}{\textbf{#1}}} \hspace*{1em} }

%--------------------------------------------------------图框定义---------------------------------------------------------%
%证明和解
\newcommand{\proof}{\vspace*{1em} \noindent  \hspace*{0.2em}  \tcbox[colframe =black, colback =black!10!white,boxrule=0.5mm,size=small,on line]{\color{black}{{ 证}}}\hspace{1.5em}}
\newcommand{\solve}{\vspace*{1em} \noindent  \hspace*{0.2em}  \tcbox[colframe =black, colback =black!10!white,boxrule=0.5mm,size=small,on line]{\color{black}{{ 解}}\hspace*{0.25em}}\hspace{1.5em}}
\newcommand{\solveother}{\vspace*{1em} \noindent  \hspace*{0.2em}  \tcbox[colframe =black, colback =black!10!white,boxrule=0.5mm,size=small,on line]{\color{black}{{ 另解}}}\hspace{1.5em}}
\newcommand{\errsolve}{\vspace*{1em} \noindent  \hspace*{0.2em}  \tcbox[colframe =red, colback =red!10!white,boxrule=0.5mm,size=small,on line]{\color{red}{{ 错解}}}\hspace{1.5em}}
\newcommand{\errreason}{\vspace*{1em} \noindent  \hspace*{0.2em}  \tcbox[colframe =red, colback =red!10!white,boxrule=0.5mm,size=small,on line]{\color{red}{{ 错因}}}\hspace{1.5em}}
\newcommand{\solvereason}{\vspace*{1em} \noindent  \hspace*{0.2em}  \tcbox[colframe =ForestGreen
	, colback =ForestGreen!15!white,boxrule=0.5mm,size=small,on line]{\color{ForestGreen}{{ 解析}}}\hspace{1.5em}}

%例
\newcommand{\examples}{\vspace*{1em}\noindent  \refstepcounter{examples} \tcbox[colframe =ex, colback =ex!10!white,boxrule=0.5mm,size=small,on line]{\color{ex}{\theexamples}\hspace*{0.3em}}\hspace{1.5em}}
\newcommand{\simpleexamples}{ \noindent  \tcbox[colframe =ex, colback =ex!10!white,boxrule=0.5mm,size=small,on line]{  \color{ex}{例}}\hspace{1.5em}}

%推论
\newcommand{\tl}{\vspace*{1em}\noindent \refstepcounter{tl} \tcbox[colframe =tl, colback =tl!10!white,boxrule=0.5mm,size=small,on line]{\color{tl}{\thetl}\hspace*{0.3em}}\hspace{1.5em}}

%注意
\newcommand{\warn}[1][]{
	\vspace*{0.5em}
	\begin{tcolorbox}[colframe=red!75!black, colback=yellow!10!white,title=注意,fonttitle = ]
		#1
\end{tcolorbox}}
\newcommand{\summarize}[1][]{
	\vspace*{0.5em}
	\begin{tcolorbox}[colframe=white!20!black, colback=white!98!black,title=评注,fonttitle = ]
		#1
\end{tcolorbox}}

%文本高亮
\newcommand{\highlights}[1][]{\tcbox[colframe =Chocolate , colback =Coral,boxrule=0.5mm,size=small,on line]{\color{white}{#1}}}


%文章标题
\title{
	\Huge{\textbf{XXXXXXX}}
	\vspace*{18em}
}
\author{
	{  \large {作者}}\\
	{  \large XXXXXX大学}\vspace*{0.5em}\\
	内部版本号:V0.02.002$\,\,$(内测版)\\
}

%-----------------------------------------------------------------------------------------------------------------------------%

%文档开始
\begin{document}
	%标题及目录
	\pagenumbering{Roman}
	\maketitle
	\clearpage \phantom{s} \thispagestyle{empty} \clearpage
	\setcounter{page}{1}
	\tableofcontents
	
	%正文部分
	\cleardoublepage
	\setcounter{page}{1}
	\pagenumbering{arabic}
	
	% 插入每一章的文档
	\chapter{随机事件}
\thispagestyle{empty}
\section{随机事件}
\subsection{随机现象}
\dy[确定性现象]{QDXXX}
在一定条件下必然出现的结果\jg\\
\dy[随机现象]{SJXX}
事先无法准确与之其结果的现象\jg
\subsection{随机现象的统计性规律}
\dy[统计规律性]{TJGLX}
随机现象在大量重复出现时所表现出来的规律性.\jg\\
\dy[随机试验]{SJSY}
对随机现象的观察.\jg\\
\dya[随机试验的特点]
\begin{enumerate}[1.]
	\setlength{\itemindent}{3em}
	\setlength{\topsep}{0.01em}
	\setlength{\itemsep}{0.01em}
	\item 可重复性
	\item 可观察性
	\item 随机性
\end{enumerate}


\subsection{样本空间}
\dy[样本点]{YBD}
随机试验的每一个可能结果.\jg\\
\dy[样本空间]{YBKJ}
样本点的全体.\jg

\subsection{随机事件}
\dy[事件]{SJ}
实验结果具备的某一可观察的特征.\jg\\
\dy[随机事件]{SJSJ}
在随机试验中可能发生也可能不发生.\jg\\
\dy[必然事件]{BRSJ}
在试验中必然发生.\jg\\
\dy[不可能事件]{BKNSJ}
在试验中一定不发生.\jg\\
\dy[基本事件]{JBSJ}
对应一个唯一的可能结果,即样本点.\jg

\subsection{事件的集合表示}

\subsection{事件建的关系和运算}
\dy[事件的包含]{SJDBH}
$A$发生必然导致$B$发生,则称事件$B$包含事件$A$,记作$B\supset A$或$A\subset B$.\jg\\
\dy[事件的相等]{SJDXD}
事件$A$包含事件$B$,事件$B$也包含事件$A$,则称事件$A$与$B$相等,记作$A=B$.\jg\\
\dy[事件的并(或和)]{SJDB}
``事件$A$与$B$至少有一个发生"这一事件称为事件$A$和$B$的并(或和),记作$A\cup B$或$A+B$.\jg\\
\dy[事件的交(或积)]{SJDJ}
``事件$A$与$B$都发生"这一事件称为事件$A$与$B$的交(或积),记作$A\cap B$.\jg\\
\dy[事件的差]{SJDC}
``事件$A$发生而$B$不发生"这一事件称为事件$A$和$B$的差,记作$A-B$.\jg\\
\dy[互不相容事件]{HBXRSJ}
若事件$A$与$B$不能同时发生,也就是说$AB$时不可能事件,即$AB=\varnothing$,则称事件$A$与$B$是不可能事件.\jg\\
\dy[对立事件]{DLSJ}
``事件$A$不发生"这一事件称为事件$A$的对立事件,记作$\overline{A}$,易见,$\overline{A}=\Omega -A$,且$\overline{(\overline{A})}$.\jg\\
\dya[有限个事件的并与交]\jg
\newpage 
\noindent\dy[完备事件组]{WBSHZ}
\par 完备事件组设$A_1,A_2,\cdots,A_n,\cdots$是有限或可数个事件,如果其满足
\par \quad (1)  $A_iA_j=\varnothing,i\ne j,\quad i,j=1,2,\cdots $
\par \quad (2)  $\bigcup\limits_iA_i=\Omega$
\par 则称$A_1,A_2,\cdots,A_n,\cdots$是一个完备事件组.\jg\\
\dya[事件的关系与运算的文氏图]\jg

\subsection{随机事件的运算律}
\dya[求和运算]\jg
\par \quad 交换律
\begin{equation}
A \cup B =B \cup A
\end{equation}
\par \quad 结合律
\begin{equation}
(A \cup B)\cup C =A\cup (B\cup C)=A\cup B\cup C
\end{equation}
\dya[求交运算]\jg
\par \quad 交换律
\begin{equation}
A\cap B=B\cap A
\end{equation}
\par \quad 结合律
\begin{equation}
(A \cap B)\cap C=A\cap (B\cap C)=A \cap B \cap C
\end{equation}
\dya[混合运算]\jg
\par \quad 第一分配律
\begin{equation}
A \cap (B \cup C)=(A\cap B)\cup (A \cap C)
\end{equation}
\par \quad 第二分配律
\begin{equation}
A \cup (B \cap C)=(A\cup B)\cap (A \cup C)
\end{equation}
\dya[求对立事件的运算]\jg
\par \quad 自反律
\begin{equation}
\overline{(\overline{A})}=A
\end{equation}
\dya[求和及交事件的对立事件]\jg
\par \quad 第一对偶律
\begin{equation}
\overline{A \cup B}=\overline{A} \cap \overline{B} 
\end{equation}
\par \quad 第二对偶律
\begin{equation}
\overline{A \cap B}=\overline{A} \cup \overline{B} 
\end{equation}

\section{随机事件的概率}
\subsection{概率及其频率解释}
参见$\rm{P}_9$
\subsection{从频率的性质看概率的性质}
参见$\rm{P}_{10}$
\subsection{概率的公理化定义}
\sj
\defination[概率公理化]
设$\Omega $是一个样本空间,定义在$\Omega $的事件域$F$上的一个实值函数$P(\cdot)$如果它满足下列三条公理:
\begin{enumerate}[1.]
	\setlength{\itemindent}{4em}
	\setlength{\topsep}{0.01em}
	\setlength{\itemsep}{0.01em}
	\item $P(\Omega )=1$
	\item 对任意事件$A$,有$P(A) \le 0$
	\item 对任意可数的两两不相容的事件$A_1,A_2,\cdots,A_n,\cdots$,有$\displaystyle P\left( \bigcup_{i=1}^{\infty } A_i\right)=\sum_{i=1}^{\infty }P(A_i) $
\end{enumerate}
则称实值函数$P(\cdot)$为$\Omega $上的一个概率测度.\jg

\subsection{概率测度的性质}
\begin{enumerate}[1.]
	\setlength{\itemindent}{4em}
	\setlength{\topsep}{0.01em}
	\setlength{\itemsep}{0.01em}
	\item $P(\varnothing)=0$
	\item 有限可加性:$\displaystyle P\left( \bigcup_{i=1}^{\infty } A_i\right)=\sum_{i=1}^{\infty }P(A_i) $
	\item $P(\overline{A})=1-P(A)$
	\item $P(A-B)=P(A)-P(AB)=P(B)-P(AB)$
	\item $0\le P(A) \le 1$
	\item $P(A\cup B)=P(A)+P(B)-P(AB)$
\end{enumerate}

\section{古典概型与集合概型}
\subsection{古典概型}
\tdefination[古典概型]
古典概型是满足下面两个假设条件的概率模型:
\begin{enumerate}[1.]
	\setlength{\itemindent}{4em}
	\setlength{\topsep}{0.01em}
	\setlength{\itemsep}{0.01em}
	\item 随机试验只有有限个结果
	\item 每一个可能记过发生的概率相同
\end{enumerate}
所以,古典概型的概率测度可表述为:
\begin{equation}
P(A)=\frac{A\mbox{中的元素个数}}{\Omega \mbox{中的元素个数}}=\frac{\mbox{使}A\mbox{发生的基本事件数}}{\mbox{基本事件总数}}
\end{equation}

\subsection{几何概型}
\tdefination[几何概型]
几何概型的概率测度可表述为
\begin{equation}
P(A)=\frac{S(A)}{S(\Omega)}
\end{equation}

\section{条件概率}
\subsection{条件概率的定义}
\tdefination[条件概率]
给定概率空间$\Omega,P$,$A,B$是其上的两个事件,且$P(A)>0$,则称$\displaystyle P(B|A)=\frac{P(AB)}{P(A)}$为已知事件$A$发生的条件下,事件$B$发生的条件概率.

\subsection{乘法公式}
\ttheorem[乘法公式]
乘法公式的两个形式:
\begin{equation}
P(AB)=P(A)\cdot P(B|A),\,P(A)>0
\end{equation}
\begin{equation}
P(AB)=P(B)\cdot P(A|B),\,P(B)>0
\end{equation}

\subsection{全概率公式}
\ttheorem[全概率公式]
设$\lbrace A_i \rbrace$是一列有限或可数无穷个两两不相容的非零概率事件,且$\bigcup\limits_{i}A_i=\Omega $,则对任意事件$B,P(B)>0$,有
\begin{equation}
P(B)=\sum\limits_{i}P(A_i)\cdot P(B|A_i)
\end{equation}

\subsection{贝叶斯公式}
\ttheorem[贝叶斯公式]
设$\lbrace A_i \rbrace$是一列有限或可数无穷个两两不相容的非零概率事件,且$\bigcup_{i=1}^{\infty}A_i=\Omega $,则对任意事件$B,P(B)>0$,有
\begin{equation}
P(A_i|B)=\frac{P(A_iB)}{P(B)}=\frac{P(A_i)\cdot P(B|A_j)}{\sum\limits_{j}P(A_j)\cdot P(B|A_j)}
\end{equation}

\section{事件的独立性}
\subsection{两个事件的独立性}
\dy[两个事件的独立性]{LGSJDDLX}
如果$P(AB)=P(A)P(B)$,则称$A$与$B$相互独立,简称$A$与$B$独立.\jg\\
\dy[有限个事件的独立性]{YXGSJDDLX}
(1)  如果有$n(n\le 2)$个事件:$A_i,A_2,\cdots,A_n$中任意两个使劲按均相互独立,即对任意$1\le i\le j \le n$,均有$P(A_iA_j)=P(A_i)P(A_j)$,则称$n$个事件$A_i,A_2,\cdots,A_n$两两独立.
\par (2)  设$A_i,A_2,\cdots,A_n$为$n(n\le 2)$个事件,如果对其中任何$k(2\le k\le n)$个事件$A_{i_1},A_{i_2},\cdots,A_{i_k}\,(1 \le i_1<i_2<\cdots<i_k\le n)$,均有$P(A_{i_1}A_{i_2}\cdots A_{i_k})=P(A_{i_1})P(A_{i_2})\cdots P(A_{i_k})$,则称
事件$A_i,A_2,\cdots,A_n$为$n(n\le 2)$相互独立.

\subsection{相互独立性的性质}
\ttheorem[相互独立性的性质]
1.  如果$n$个事件$A_1,A_2,⋯,A_n$相互独立,则将其中任何$m(1\leq m \leq n)$个事件改为相应的对立事件,形成的新的$n$个事件仍然相互独立.
\par 2.  如果$n$个事件$A_1,A_2,⋯,A_n$相互独立,则有
\begin{equation}
	P\left( \bigcup_{i=1}^{n} A_i\right) =1-\prod_{i=1}^{n}P\left(\overline{A_i} \right) =1-\prod_{i=1}^{n}\left[1- P\left(A_i \right)\right]
\end{equation}

\subsection{伯努利概型}
\tdefination[伯努利概型]
只有两个可能的结果的试验称为伯努利试验,一个伯努利试验独立重复$n$次形成的试验序列称为$n$重伯努利试验.
\jg

\theorem[伯努利定理]
在一次试验中,事件$A$发生的概率为$p(0<p<1)$,则在$n$重伯努利试验中,事件$A$恰好发生$k$次的概率$b(k;n,p)$为
\begin{equation}
b(k;n,p)=C_n^k\,p^k\,q^{n-k}
\end{equation}
其中,$q=1-p.$
\par 在伯努利试验序列中,设每次试验中事件$A$发生的概率为$p$,“事件$A$在第$k$次试验中才首次发生”$(k≥1)$这一事件的概率为
\begin{equation}
g(k,p)=p\,q^{k-1}
\end{equation}




	
	
	%打印索引—————————————
	\cleardoublepage
	\addcontentsline{toc}{chapter}{附录}
	\addcontentsline{toc}{section}{索引}
	\color{titlepurplec}
	\appendix
	\printindex
	%———————————————
	
\end{document}