%模板
\documentclass[10pt,a4paper,twoside]{book}

\usepackage{ctex}
\usepackage{xeCJK}
\usepackage[linesnumbered,boxed]{algorithm2e}
\usepackage{ulem}
\usepackage{makecell}
\usepackage{verbatim}
\usepackage{enumerate}%罗列专用宏包
\usepackage{graphicx}%插入图片的宏包
\usepackage{subfigure} 
\usepackage{newtxtext}
\usepackage{newtxmath}
\usepackage{extarrows}
\usepackage{bm}
\usepackage{esint}
\usepackage{makeidx}%索引专用
\makeindex  %添加索引
\usepackage{fancyhdr}
\usepackage{framed}
\usepackage{amsfonts}
\usepackage{wrapfig}
\usepackage{textcomp}%树叶图案在这个包里
\usepackage{bbding}%很多漂亮的图案
\usepackage[dvipsnames, svgnames, x11names]{xcolor}%导入了所有颜色配置文件的宏包
\usepackage{CJKfntef}
\usepackage{geometry}%页边距调整
\geometry{left=2cm,right=2cm,bottom=2cm,top=2cm}
\usepackage{titletoc}%目录页的宏包
\usepackage{titlesec}%改变章节或标题的样式的宏包
\usepackage[bookmarks=true,colorlinks,linkcolor=black]{hyperref}
\usepackage{enumerate}%使用改宏包优化罗列环境
\usepackage{tcolorbox}%box宏包
\usepackage{colortbl,booktabs}%第二个包定义了几个*rule  
\usepackage{multicol}
\usepackage{multirow}
\usepackage{tikz}
\usetikzlibrary{shapes.geometric}
\usetikzlibrary{arrows,arrows.meta}

%字体设置
\setCJKmainfont[BoldFont={PingFangSC-Medium}]{PingFangSC-Regular}
\setCJKfamilyfont{kai}{STKaitiSC-Regular}%都是用来定义字体的(此处使用电脑自带楷书)
\DeclareMathSizes{10}{10}{5.5}{4}

%章节或标题的样式
\titleformat{\chapter}{\bfseries\Huge\color{titlepurple}}{第\ \thechapter\ 章\ \quad}{0pt}{}
\titleformat{\section}{\Large\color{titlepurpleb}}{\bfseries{\thesection}\quad  }{0pt}{}
\titleformat{\subsection}{\large\color{titlepurplec}}{\bfseries{\thesubsection}\quad  }{0pt}{}
\titlespacing{\section}{0em}{1em}{1em}[1em]
\titlespacing{\subsection}{1.5em}{1em}{0.5em}[1em]
%格式如下:\titlespacing*{章节名称}{左间距}{(前)行间距}{(后)行间距}[右间距(一般都没用,填0.1em即可,但不能不填)]
\titlespacing*{\subsubsection}{2em}{3em}{1em}[1em]


%目录调整
\newcounter{mycontents}
\newcommand{\thecontents}{\refstepcounter{mycontents} \alph{mycontents}.}
%\titlecontents{标题名}[左间距]{标题格式}{标题标志}{无序号标题}{指引线与页码}[下间距]
\titlecontents{chapter}
[0cm]
{\bf \large \vspace{0.8em} }{\contentspush{第 \thecontentslabel\ 章 \hspace*{0.8em}}}{}{\titlerule*[0.5pc]{$\cdot$}\contentspage}
\titlecontents{section}[1.7cm]{\bf  \vspace{0.5em} }{\contentslabel{2.4em}}{\hspace*{-2.5em} \thecontents \hspace*{0.8em}}{\titlerule*[0.5pc]{$\cdot$}\contentspage}
\titlecontents{subsection}[2.5cm]{\small \vspace{0.2em}}{\contentslabel{3em}}{}{\titlerule*[0.5pc]{$\cdot$}\contentspage}

%定义颜色
%定义某个颜色,对应颜色代号查表
\definecolor{titlepurple}{HTML}{5758BB}%一级标题(目前:蓝紫色)
\definecolor{titlepurpleb}{HTML}{3A006F}%二级标题(目前:深紫色)
\definecolor{titlepurplec}{HTML}{006266}%三级标题(目前:墨绿色)
\definecolor{tab1}{HTML}{9698ED}%表格1
\definecolor{tab2}{HTML}{DBDCFF}%表格2
\definecolor{dy0}{HTML}{EA7500}%小标题定义专用(目前:橙黄色)
\definecolor{dl}{HTML}{007500}%小标题定理专用(目前:深绿色)
\definecolor{inference}{HTML}{343300}%小标题推论专用(目前:墨绿色)
\definecolor{ex}{HTML}{7158e2}%小标题例专用(目前:紫色)
\definecolor{dy}{HTML}{BF0060}%夹杂在文本中的定义词的颜色1(目前:深红色)
\definecolor{dy2}{HTML}{6C3365}%夹杂在文本中的定义词的颜色2(目前:红紫色)
\definecolor{超链接}{HTML}{0000C6}%含超链接的文字专用色(目前:蓝紫色)
\definecolor{文字底色}{HTML}{F8FF00}%强调的文字底色(目前:黄色)
\definecolor{eq}{HTML}{F0F0F0}
\definecolor{tl}{HTML}{D94600}


%定义计数器
\newcounter{theorem}[chapter]
\newcounter{defination}[chapter]
\newcounter{example}[chapter]
\newcounter{inference}[chapter]
\newcounter{examples}[chapter]
\newcounter{tl}[chapter]
\newcounter{F}[section]
\newcounter{G}[section]
\newcounter{H}[section]
\renewcommand{\thetheorem}{{ 定理} \textbf{\thechapter.\arabic{theorem}}}
\renewcommand{\thedefination}{{ 定义} \textbf{\thechapter.\arabic{defination}}}
\renewcommand{\theexample}{{ 题型} \textbf{\thechapter.\arabic{example}}}
\renewcommand{\theinference}{{ 方法} \textbf{\thechapter.\arabic{inference}}}
\renewcommand{\theexamples}{{ 例}  \textbf{\thechapter.\arabic{examples}}}
\renewcommand{\thetl}{{ 推论}  \textbf{\thechapter.\arabic{tl}}}
\newcommand{\s}{\hspace*{-2.7em}}

%定义环境
\newcommand{\mybox}[2][]{
	\begin{tcolorbox}[on line,
		arc=0pt,outer arc=0pt,colback=#1!10!white,colframe=#1,
		boxsep=0pt,left=3pt,right=3pt,top=3.5pt,bottom=3.5pt,
		boxrule=0pt,leftrule=1.5pt]#2
\end{tcolorbox}}

%命令格式说明:正常情况的命令就是中文对应的英文名,以下有几个特殊情况进行了微调
%1. 小标题在列表上方,使用enup+英文名;小标题在列表下方,使用enbelow+英文名
%2. 标题间隔太大,采用t+英文名
%3. 间距太小,用add+英文名
%4. 在列举环境中间距太小用adden+英文名

%定理类
\newcommand{\theorem}[1][]{\vspace{1em}\s \refstepcounter{theorem} \mybox[dl]{\color{dl}\thetheorem\hspace{1em}#1}\vspace{0.5em}  \par}
\newcommand{\enuptheorem}[1][]{\vspace{1em}\s \refstepcounter{theorem}\label{#1} \mybox[dl]{\color{dl}\thetheorem\hspace{1em}#1}\vspace*{-0.8cm}}
\newcommand{\enbelowtheorem}[1][]{\hspace*{-1.5em}\s \refstepcounter{theorem}\label{#1} \mybox[dl]{\color{dl}\thetheorem\hspace{1em}#1}  \par}
\newcommand{\ttheorem}[1][]{\s \refstepcounter{theorem}\label{#1} \mybox[dl]{\color{dl}\thetheorem\hspace{1em}#1}\vspace{0.5em}\par }
\newcommand{\addtheorem}[1][]{\vspace{1.2em}\s \refstepcounter{theorem}\label{#1} \mybox[dl]{\color{dl}\thetheorem\hspace{1em}#1}\vspace*{0.5em}  \par}
\newcommand{\addentheorem}[1][]{\vspace{1.2em}\hspace*{-1.5em}\s \refstepcounter{theorem}\label{#1} \mybox[dl]{\color{dl}\thetheorem\hspace{1em}#1}\vspace{0.5em}  \par}

%推论类
\newcommand{\inference}[1][]{\vspace{1em}\s \refstepcounter{inference} \mybox[inference]{\color{inference}\theinference\hspace{1em}#1}\vspace{0.5em}   \par}
\newcommand{\enupinference}[1][]{\vspace{1em}\s \refstepcounter{inference} \mybox[inference]{\color{inference}\theinference\hspace{1em}#1}\vspace*{-0.8cm}  }
\newcommand{\enbelowinference}[1][]{\hspace*{-1.5em}\s \refstepcounter{inference} \mybox[inference]{\color{inference}\theinference\hspace{1em}#1}  \par}
\newcommand{\tinference}[1][]{\s \refstepcounter{inference}\label{#1} \mybox[inference]{\color{inference}\theinference\hspace{1em}#1}\vspace*{0.5em} \par }
\newcommand{\addinference}[1][]{\vspace{1.2em}\s \refstepcounter{inference}\label{#1} \mybox[inference]{\color{inference}\theinference\hspace{1em}#1}\vspace{0.5em} \par}
\newcommand{\addeninference}[1][]{\vspace{1.2em}\hspace*{-1.5em}\s \refstepcounter{inference}\label{#1} \mybox[inference]{\color{inference}\theinference\hspace{1em}#1}\vspace{0.5em} \par}

%定义类
\newcommand{\defination}[1][]{\vspace{1em}\s \refstepcounter{defination} \mybox[dy0]{\color{dy0}\thedefination\hspace{1em}#1}\vspace{0.5em} \par}
\newcommand{\enupdefination}[1][]{\vspace{1em}\s \refstepcounter{defination}\label{#1} \mybox[dy0]{\color{dy0}\thedefination\hspace{1em}#1}\vspace*{-0.8cm}}
\newcommand{\enbelowdefination}[1][]{\hspace*{-1.5em}\s \refstepcounter{defination}\label{#1} \mybox[dy0]{\color{dy0}\thedefination\hspace{1em}#1} \par }
\newcommand{\tdefination}[1][]{   \s \refstepcounter{defination} \mybox[dy0]{\color{dy0}\thedefination\hspace{1em}#1}\vspace*{0.5em}  \par }
\newcommand{\adddefination}[1][]{\vspace{1.2em}\s \refstepcounter{defination}\label{#1} \mybox[dy0]{\color{dy0}\thedefination\hspace{1em}#1}\vspace{0.5em} \par}
\newcommand{\addendefination}[1][]{\vspace{1.2em}\hspace*{-1.5em}\s \refstepcounter{defination}\label{#1} \mybox[dy0]{\color{dy0}\thedefination\hspace{1em}#1}\vspace{0.5em} \par}

%题型类(无标签)
\newcommand{\example}[1][]{\vspace{1em} \s \refstepcounter{example} \mybox[ex]{\color{ex}\theexample\hspace{1em}#1}\vspace{0.5em} \par }
\newcommand{\enupexample}[1][]{\vspace{1em}\s \refstepcounter{example} \mybox[ex]{\color{ex}\theexample\hspace{1em}#1}\vspace*{-0.8cm}}
\newcommand{\enbelowexample}[1][]{\hspace*{-1.5em}\s \refstepcounter{example}\mybox[ex]{\color{ex}\theexample\hspace{1em}#1} \par }
\newcommand{\texample}[1][]{  \s \refstepcounter{example} \mybox[ex]{\color{ex}\theexample\hspace{1em}#1} \vspace{0.5em} \par }
\newcommand{\addexample}[1][]{\vspace{1.2em}\s \refstepcounter{example} \mybox[ex]{\color{ex}\theexample\hspace{1em}#1}\vspace{0.5em} \par }
\newcommand{\addenexample}[1][]{\vspace{1.2em}\hspace*{-1.5em}\s \refstepcounter{example} \mybox[ex]{\color{ex}\theexample\hspace{1em}#1}\vspace{0.5em} \par }

%\theoremstyle{break}
%\theoremindent0.2cm
%\newtheorem*{theorem}{\hspace{0.2cm}\color{dl}\label{#1} \mybox[dl]{\color{dl}定理\addtocounter{A}{1} \thesection.\arabic{A}}}
%\newtheorem*{defination}{\hspace{0.2cm}\color{dy0}\label{#1} \mybox[dy0]{\color{dy0}定义\addtocounter{B}{1} \thesection.\arabic{B}}}
%\newtheorem*{feature}{\hspace{-0.16cm}\color{ffa725}\label{#1} \mybox[ffa725]{\color{ffa725}性质\addtocounter{C}{1} \thesection.\arabic{C}}}
%\newtheorem*{inference}{\hspace{-0.16cm}\color{1a9850}\label{#1} \mybox[1a9850]{\color{1a9850}推论\addtocounter{D}{1} \thesection.\arabic{D}}}
%\newtheorem*{method}{\hspace{-0.16cm}\color{6a3d9a}\label{#1} \mybox[6a3d9a]{\color{6a3d9a}方法\addtocounter{E}{1} \thesection.\arabic{E}}}
%\newtheorem*{example}{\hspace{-0.16cm}\color{53a9ab}\label{#1} \mybox[53a9ab]{\color{53a9ab}例题\addtocounter{F}{1} \thesection.\arabic{F}}}


%文章标题
		\title{
	\Huge{高等数学$\,$\uppercase\expandafter{\romannumeral1}$\,$ 总结}\\
	\quad\\
	\quad\\
	\quad\\
	\quad\\
	\quad\\
	\quad\\
	\quad\\
	\quad\\
	\quad\\
}
\author{
	{\CJKfamily{kai} \large {易鹏}}\\
	{\CJKfamily{kai} \large 中山大学}\vspace*{0.5em}\\
	内部版本号:V4.21.59.053$\,\,$(内测版)\\
}



%调整间距(倍数)
\linespread{1.5}

%自定义页眉页脚---------------
\pagestyle{fancy}
\renewcommand{\chaptermark}[1]{\markboth{\;第\ \thechapter\ 章\quad#1\;}{}}
\renewcommand{\sectionmark}[1]{\markright{\;\thesection\ #1\;}}
\fancyhf{}
%\fancyfoot[C]{\bfseries\thepage}
\fancyhead[LO]{\small\CJKfamily{song}\rightmark}
\fancyhead[RE]{\small\CJKfamily{song}\leftmark}
\fancyhead[RO,LE]{\;\thepage\;}
\fancyfoot[RO,LE]{\footnotesize\CJKfamily{heilight}{高等数学}}
\fancyfoot[RE,LO]{\footnotesize\CJKfamily{heilight}Advanced Mathematics}
\renewcommand{\headrulewidth}{0.4pt} % 注意不用\setlength
%\renewcommand{\footrulewidth}{0pt}
\fancyheadoffset[LE,RO]{0cm}
\fancyfootoffset[LE,RO]{0cm}
% 注意不用\setlength
%\renewcommand{\footrulewidth}{0pt}

%自定义命令
\newcommand{\blackbf}[1][]{{\CJKfamily{heiti}#1}}
\newcommand{\link}[1][]{\hyperref[#1] {\color{超链接}#1}}
\newcommand{\sj}{\vspace*{-1em}}
\newcommand{\eqsj}{\vspace*{-2.5em}}
\newcommand{\kg}{\hspace*{2em}}
\newcommand{\jg}{\vspace*{1em}}
\newcommand{\huo}{\quad \mbox{或} \quad}
\renewcommand{\d}{{\rm{d}}}
\newcommand{\e}{{\rm{e}}}
\newcommand{\dy}[2][]{{\color{dy}#1}\index{#2@#1}}
\newcommand{\eq}[1][]{\colorbox{eq}{$\displaystyle #1$}}
\renewcommand{\oiint}{\varoiint}
\newcommand{\ds}[1][]{\colorbox{文字底色}{#1}}
\def\degree{{}^{\circ}}

%--------------------------------------------------------图框定义---------------------------------------------------------%
%证明和解
\newcommand{\proof}{\vspace*{1em} \noindent  \hspace*{0.2em}  \tcbox[colframe =black, colback =black!10!white,boxrule=0.5mm,size=small,on line]{\color{black}{{ 证}}\hspace*{0.25em}}\hspace{1.5em}}
\newcommand{\solve}{\vspace*{1em} \noindent  \hspace*{0.2em}  \tcbox[colframe =black, colback =black!10!white,boxrule=0.5mm,size=small,on line]{\color{black}{{ 解}}\hspace*{0.25em}}\hspace{1.5em}}
\newcommand{\solveother}{\vspace*{1em} \noindent  \hspace*{0.2em}  \tcbox[colframe =black, colback =black!10!white,boxrule=0.5mm,size=small,on line]{\color{black}{{ 另解}}\hspace*{0.25em}}\hspace{1.5em}}
\newcommand{\errsolve}{\vspace*{1em} \noindent  \hspace*{0.2em}  \tcbox[colframe =red, colback =red!10!white,boxrule=0.5mm,size=small,on line]{\color{red}{{ 错解}}\hspace*{0.25em}}\hspace{1.5em}}
\newcommand{\errreason}{\vspace*{1em} \noindent  \hspace*{0.2em}  \tcbox[colframe =red, colback =red!10!white,boxrule=0.5mm,size=small,on line]{\color{red}{{ 错因}}\hspace*{0.25em}}\hspace{1.5em}}
\newcommand{\solvereason}{\vspace*{1em} \noindent  \hspace*{0.2em}  \tcbox[colframe =ForestGreen
	, colback =ForestGreen!15!white,boxrule=0.5mm,size=small,on line]{\color{ForestGreen}{{ 解析}}\hspace*{0.25em}}\hspace{1.5em}}

%例
\newcommand{\examples}{\vspace*{1em}\noindent  \refstepcounter{examples} \tcbox[colframe =ex, colback =ex!10!white,boxrule=0.5mm,size=small,on line]{\color{ex}{\theexamples}\hspace*{0.3em}}\hspace{1.5em}}
\newcommand{\simpleexamples}{ \noindent  \tcbox[colframe =ex, colback =ex!10!white,boxrule=0.5mm,size=small,on line]{  \color{ex}{例}}\hspace{1.5em}}

%推论
\newcommand{\tl}{\vspace*{1em}\noindent \refstepcounter{tl} \tcbox[colframe =tl, colback =tl!10!white,boxrule=0.5mm,size=small,on line]{\color{tl}{\thetl}\hspace*{0.3em}}\hspace{1.5em}}

%注意
\newcommand{\warn}[1][]{
	\vspace*{0.5em}
	\begin{tcolorbox}[colframe=red!75!black, colback=yellow!10!white,title=注意,fonttitle = ]
		#1
\end{tcolorbox}}
\newcommand{\summarize}[1][]{
	\vspace*{0.5em}
	\begin{tcolorbox}[colframe=white!20!black, colback=white!98!black,title=评注,fonttitle = ]
		#1
\end{tcolorbox}}

%小结
\newcommand{\summary}[1][]{
	\vspace*{0.5em}
	\begin{tcolorbox}[title=小结]
		#1
\end{tcolorbox}}

%文本高亮
\newcommand{\highlights}[1][]{\tcbox[colframe =Chocolate , colback =Coral,boxrule=0.5mm,size=small,on line]{\color{white}{#1}}}
%------------------------------------------------------------------------------------------------------------------------%

%文档开始
\begin{document}
%标题及目录
\pagenumbering{Roman}
\clearpage {\pagestyle{empty}}
\maketitle
\setcounter{page}{1}
\tableofcontents

%正文部分
\newpage
\setcounter{page}{1}
\pagenumbering{arabic}

%第一章-函数与极限
\chapter{随机事件}
\thispagestyle{empty}
\section{随机事件}
\subsection{随机现象}
\dy[确定性现象]{QDXXX}
在一定条件下必然出现的结果\jg\\
\dy[随机现象]{SJXX}
事先无法准确与之其结果的现象\jg
\subsection{随机现象的统计性规律}
\dy[统计规律性]{TJGLX}
随机现象在大量重复出现时所表现出来的规律性.\jg\\
\dy[随机试验]{SJSY}
对随机现象的观察.\jg\\
\dya[随机试验的特点]
\begin{enumerate}[1.]
	\setlength{\itemindent}{3em}
	\setlength{\topsep}{0.01em}
	\setlength{\itemsep}{0.01em}
	\item 可重复性
	\item 可观察性
	\item 随机性
\end{enumerate}


\subsection{样本空间}
\dy[样本点]{YBD}
随机试验的每一个可能结果.\jg\\
\dy[样本空间]{YBKJ}
样本点的全体.\jg

\subsection{随机事件}
\dy[事件]{SJ}
实验结果具备的某一可观察的特征.\jg\\
\dy[随机事件]{SJSJ}
在随机试验中可能发生也可能不发生.\jg\\
\dy[必然事件]{BRSJ}
在试验中必然发生.\jg\\
\dy[不可能事件]{BKNSJ}
在试验中一定不发生.\jg\\
\dy[基本事件]{JBSJ}
对应一个唯一的可能结果,即样本点.\jg

\subsection{事件的集合表示}

\subsection{事件建的关系和运算}
\dy[事件的包含]{SJDBH}
$A$发生必然导致$B$发生,则称事件$B$包含事件$A$,记作$B\supset A$或$A\subset B$.\jg\\
\dy[事件的相等]{SJDXD}
事件$A$包含事件$B$,事件$B$也包含事件$A$,则称事件$A$与$B$相等,记作$A=B$.\jg\\
\dy[事件的并(或和)]{SJDB}
``事件$A$与$B$至少有一个发生"这一事件称为事件$A$和$B$的并(或和),记作$A\cup B$或$A+B$.\jg\\
\dy[事件的交(或积)]{SJDJ}
``事件$A$与$B$都发生"这一事件称为事件$A$与$B$的交(或积),记作$A\cap B$.\jg\\
\dy[事件的差]{SJDC}
``事件$A$发生而$B$不发生"这一事件称为事件$A$和$B$的差,记作$A-B$.\jg\\
\dy[互不相容事件]{HBXRSJ}
若事件$A$与$B$不能同时发生,也就是说$AB$时不可能事件,即$AB=\varnothing$,则称事件$A$与$B$是不可能事件.\jg\\
\dy[对立事件]{DLSJ}
``事件$A$不发生"这一事件称为事件$A$的对立事件,记作$\overline{A}$,易见,$\overline{A}=\Omega -A$,且$\overline{(\overline{A})}$.\jg\\
\dya[有限个事件的并与交]\jg
\newpage 
\noindent\dy[完备事件组]{WBSHZ}
\par 完备事件组设$A_1,A_2,\cdots,A_n,\cdots$是有限或可数个事件,如果其满足
\par \quad (1)  $A_iA_j=\varnothing,i\ne j,\quad i,j=1,2,\cdots $
\par \quad (2)  $\bigcup\limits_iA_i=\Omega$
\par 则称$A_1,A_2,\cdots,A_n,\cdots$是一个完备事件组.\jg\\
\dya[事件的关系与运算的文氏图]\jg

\subsection{随机事件的运算律}
\dya[求和运算]\jg
\par \quad 交换律
\begin{equation}
A \cup B =B \cup A
\end{equation}
\par \quad 结合律
\begin{equation}
(A \cup B)\cup C =A\cup (B\cup C)=A\cup B\cup C
\end{equation}
\dya[求交运算]\jg
\par \quad 交换律
\begin{equation}
A\cap B=B\cap A
\end{equation}
\par \quad 结合律
\begin{equation}
(A \cap B)\cap C=A\cap (B\cap C)=A \cap B \cap C
\end{equation}
\dya[混合运算]\jg
\par \quad 第一分配律
\begin{equation}
A \cap (B \cup C)=(A\cap B)\cup (A \cap C)
\end{equation}
\par \quad 第二分配律
\begin{equation}
A \cup (B \cap C)=(A\cup B)\cap (A \cup C)
\end{equation}
\dya[求对立事件的运算]\jg
\par \quad 自反律
\begin{equation}
\overline{(\overline{A})}=A
\end{equation}
\dya[求和及交事件的对立事件]\jg
\par \quad 第一对偶律
\begin{equation}
\overline{A \cup B}=\overline{A} \cap \overline{B} 
\end{equation}
\par \quad 第二对偶律
\begin{equation}
\overline{A \cap B}=\overline{A} \cup \overline{B} 
\end{equation}

\section{随机事件的概率}
\subsection{概率及其频率解释}
参见$\rm{P}_9$
\subsection{从频率的性质看概率的性质}
参见$\rm{P}_{10}$
\subsection{概率的公理化定义}
\sj
\defination[概率公理化]
设$\Omega $是一个样本空间,定义在$\Omega $的事件域$F$上的一个实值函数$P(\cdot)$如果它满足下列三条公理:
\begin{enumerate}[1.]
	\setlength{\itemindent}{4em}
	\setlength{\topsep}{0.01em}
	\setlength{\itemsep}{0.01em}
	\item $P(\Omega )=1$
	\item 对任意事件$A$,有$P(A) \le 0$
	\item 对任意可数的两两不相容的事件$A_1,A_2,\cdots,A_n,\cdots$,有$\displaystyle P\left( \bigcup_{i=1}^{\infty } A_i\right)=\sum_{i=1}^{\infty }P(A_i) $
\end{enumerate}
则称实值函数$P(\cdot)$为$\Omega $上的一个概率测度.\jg

\subsection{概率测度的性质}
\begin{enumerate}[1.]
	\setlength{\itemindent}{4em}
	\setlength{\topsep}{0.01em}
	\setlength{\itemsep}{0.01em}
	\item $P(\varnothing)=0$
	\item 有限可加性:$\displaystyle P\left( \bigcup_{i=1}^{\infty } A_i\right)=\sum_{i=1}^{\infty }P(A_i) $
	\item $P(\overline{A})=1-P(A)$
	\item $P(A-B)=P(A)-P(AB)=P(B)-P(AB)$
	\item $0\le P(A) \le 1$
	\item $P(A\cup B)=P(A)+P(B)-P(AB)$
\end{enumerate}

\section{古典概型与集合概型}
\subsection{古典概型}
\tdefination[古典概型]
古典概型是满足下面两个假设条件的概率模型:
\begin{enumerate}[1.]
	\setlength{\itemindent}{4em}
	\setlength{\topsep}{0.01em}
	\setlength{\itemsep}{0.01em}
	\item 随机试验只有有限个结果
	\item 每一个可能记过发生的概率相同
\end{enumerate}
所以,古典概型的概率测度可表述为:
\begin{equation}
P(A)=\frac{A\mbox{中的元素个数}}{\Omega \mbox{中的元素个数}}=\frac{\mbox{使}A\mbox{发生的基本事件数}}{\mbox{基本事件总数}}
\end{equation}

\subsection{几何概型}
\tdefination[几何概型]
几何概型的概率测度可表述为
\begin{equation}
P(A)=\frac{S(A)}{S(\Omega)}
\end{equation}

\section{条件概率}
\subsection{条件概率的定义}
\tdefination[条件概率]
给定概率空间$\Omega,P$,$A,B$是其上的两个事件,且$P(A)>0$,则称$\displaystyle P(B|A)=\frac{P(AB)}{P(A)}$为已知事件$A$发生的条件下,事件$B$发生的条件概率.

\subsection{乘法公式}
\ttheorem[乘法公式]
乘法公式的两个形式:
\begin{equation}
P(AB)=P(A)\cdot P(B|A),\,P(A)>0
\end{equation}
\begin{equation}
P(AB)=P(B)\cdot P(A|B),\,P(B)>0
\end{equation}

\subsection{全概率公式}
\ttheorem[全概率公式]
设$\lbrace A_i \rbrace$是一列有限或可数无穷个两两不相容的非零概率事件,且$\bigcup\limits_{i}A_i=\Omega $,则对任意事件$B,P(B)>0$,有
\begin{equation}
P(B)=\sum\limits_{i}P(A_i)\cdot P(B|A_i)
\end{equation}

\subsection{贝叶斯公式}
\ttheorem[贝叶斯公式]
设$\lbrace A_i \rbrace$是一列有限或可数无穷个两两不相容的非零概率事件,且$\bigcup_{i=1}^{\infty}A_i=\Omega $,则对任意事件$B,P(B)>0$,有
\begin{equation}
P(A_i|B)=\frac{P(A_iB)}{P(B)}=\frac{P(A_i)\cdot P(B|A_j)}{\sum\limits_{j}P(A_j)\cdot P(B|A_j)}
\end{equation}

\section{事件的独立性}
\subsection{两个事件的独立性}
\dy[两个事件的独立性]{LGSJDDLX}
如果$P(AB)=P(A)P(B)$,则称$A$与$B$相互独立,简称$A$与$B$独立.\jg\\
\dy[有限个事件的独立性]{YXGSJDDLX}
(1)  如果有$n(n\le 2)$个事件:$A_i,A_2,\cdots,A_n$中任意两个使劲按均相互独立,即对任意$1\le i\le j \le n$,均有$P(A_iA_j)=P(A_i)P(A_j)$,则称$n$个事件$A_i,A_2,\cdots,A_n$两两独立.
\par (2)  设$A_i,A_2,\cdots,A_n$为$n(n\le 2)$个事件,如果对其中任何$k(2\le k\le n)$个事件$A_{i_1},A_{i_2},\cdots,A_{i_k}\,(1 \le i_1<i_2<\cdots<i_k\le n)$,均有$P(A_{i_1}A_{i_2}\cdots A_{i_k})=P(A_{i_1})P(A_{i_2})\cdots P(A_{i_k})$,则称
事件$A_i,A_2,\cdots,A_n$为$n(n\le 2)$相互独立.

\subsection{相互独立性的性质}
\ttheorem[相互独立性的性质]
1.  如果$n$个事件$A_1,A_2,⋯,A_n$相互独立,则将其中任何$m(1\leq m \leq n)$个事件改为相应的对立事件,形成的新的$n$个事件仍然相互独立.
\par 2.  如果$n$个事件$A_1,A_2,⋯,A_n$相互独立,则有
\begin{equation}
	P\left( \bigcup_{i=1}^{n} A_i\right) =1-\prod_{i=1}^{n}P\left(\overline{A_i} \right) =1-\prod_{i=1}^{n}\left[1- P\left(A_i \right)\right]
\end{equation}

\subsection{伯努利概型}
\tdefination[伯努利概型]
只有两个可能的结果的试验称为伯努利试验,一个伯努利试验独立重复$n$次形成的试验序列称为$n$重伯努利试验.
\jg

\theorem[伯努利定理]
在一次试验中,事件$A$发生的概率为$p(0<p<1)$,则在$n$重伯努利试验中,事件$A$恰好发生$k$次的概率$b(k;n,p)$为
\begin{equation}
b(k;n,p)=C_n^k\,p^k\,q^{n-k}
\end{equation}
其中,$q=1-p.$
\par 在伯努利试验序列中,设每次试验中事件$A$发生的概率为$p$,“事件$A$在第$k$次试验中才首次发生”$(k≥1)$这一事件的概率为
\begin{equation}
g(k,p)=p\,q^{k-1}
\end{equation}





%第七章-重积分
\chapter{重积分}
\section{二重积分}
\subsection{二重积分的定义}
\thispagestyle{empty}

\vspace*{-1em}

\defination[二重积分的定义]
设$z=f(x,y)$是定义在平面上的有界闭区域$D$上的函数,若对$D$的任意分割$\{D_1,D_2,\cdots,D_n\}$及任意选择的$(x_i,y_i) \in D_i (i=1,2,\cdots,n)$,当$\lambda \rightarrow 0$时,极限
\begin{equation}
	\lim_{\lambda \rightarrow 0} \sum^{n}_{i=1} f(x_i,y_i)\,\Delta \sigma_i
	\footnote{$\lambda$表示$n$个区域$D_i$其中的最大直径,$\Delta \sigma_i$表示$D_i$的最大面积.}
\end{equation}

总存在,则这个极限称为$f(x,y)$在$D$上的二重积分,记做
\begin{equation}
	\iint\limits_{D}f(x,y) \, \, \d \sigma \quad \mbox{或} \quad  \iint\limits_{D}f(x,y) \, \, \d x \d y
\end{equation}

\par 其中,$D$称作积分区域,而$f(x,y)$称作被积函数,$\d\sigma$称为面积元素.

\subsection{二重积分的性质}\label{二重积分的性质}

\vspace*{-1em}

\theorem[二重积分的三个基本性质]
\vspace*{-1.5em}
\begin{enumerate}
	\setlength{\itemindent}{1em}
	\setlength{\topsep}{0.01em}
	\setlength{\itemsep}{0.01em}
	
	\item 常数因子可以提取:($k$为常数)
	\begin{equation}
		\iint\limits_{D}kf(x,y) \, \, \d \sigma =k\iint\limits_{D}f(x,y) \, \, \d \sigma 
	\end{equation}
	\vspace*{-2.5em}
	\item 被积函数的可拆可合性:
	\begin{equation}
		\iint\limits_{D}\left[ f(x,y)\pm g(x,y)\right]  \, \, \d \sigma =\iint\limits_{D}f(x,y) \, \, \d \sigma \pm \iint\limits_{D}g(x,y) \, \, \d \sigma
	\end{equation}
	\vspace*{-2.5em}
	\item 积分区域的可拆可合性:(设$D \rightarrow D_1+D_2$)
	\begin{equation}
		\iint\limits_{D}f(x,y) \, \, \d \sigma =\iint\limits_{D_1}f(x,y) \, \, \d \sigma + \iint\limits_{D_2}f(x,y) \, \, \d \sigma 
	\end{equation}
\end{enumerate}

\theorem[积分的保号性]
若函数$f$及$g$在$D$上满足不等式
\[
f(x,y) \le g(x,y), \quad \forall (x,y) \in D 
\]
则
\begin{equation}
	\iint\limits_{D}f(x,y) \, \, \d \sigma \le \iint\limits_{D}g(x,y) \, \, \d \sigma
	\label{积分的保号性}
\end{equation}
\par 特别地,由于$-|f(x,y)| \le f(x,y) \le |f(x,y)|$,带入式\eqref{积分的保号性},得到
\begin{equation}
	\left| \iint\limits_{D}f(x,y) \, \, \d \sigma \right|  \le \iint\limits_{D}|f(x,y)| \, \, \d \sigma
\end{equation}

\theorem[积分中值定理]
若函数$f(x,y)$在有界闭区域$D$上连续,则在$D$上至少存在一点$(x_0,y_0)$,使
\begin{equation}
	\iint\limits_{D}f(x,y) \, \, \d \sigma=f(x_0,y_0) \cdot S
\end{equation}
\par 其中$S$为区域$D$的面积.

\subsection{二重积分的计算}

\vspace*{-1em}

\theorem[$X$型积分与$Y$型积分]
对于不同的积分区域主要可以划分为两种:$X$型积分与$Y$型积分
\begin{equation}
	\begin{split}
		\iint\limits_{D}f(x,y)\,\, \d x \d y &= \int_{a}^{b}\left[ \int_{\varphi_1(x)}^{\varphi_2(x)}f(x,y)\,\, \d x\right] \d y \\
		&= \int_{a}^{b}\left[ \int_{\varphi_1(y)}^{\varphi_2(y)}f(x,y)\,\, \d y\right] \d x 
	\end{split}
\end{equation}

\vspace*{-1em}

\theorem[极坐标变换]
设$x,y$的极坐标方程为
$
\begin{cases}
	x = r \cos \theta,\\
	y = r\sin \theta . \\
\end{cases}
$
则
\begin{equation}
	\iint\limits_{D}f(x,y)\,\, \d x \d y= \int_{\alpha}^{\beta }\d \theta \int_{r_1(\theta)}^{r_2(\theta)}f(r \cos \theta , r \sin \theta )r\,\,\d r
\end{equation}

\vspace*{-1em}

\theorem[广义极坐标变换]
设$x,y$的极坐标方程为
$
\begin{cases}
	x = ar \cos \theta,\\
	y = br\sin \theta . \\
\end{cases}
$
则
\begin{equation}
	\iint\limits_{D}f(x,y)\,\, \d x \d y= \int_{\alpha}^{\beta }\d \theta \int_{r_1(\theta)}^{r_2(\theta)}f(ar \cos \theta , br \sin \theta )abr\,\,\d r
\end{equation}

\vspace*{-1em}

\theorem[一般变换]
设$x,y$满足
$
\begin{cases}
	x = x(\xi,\eta),\\
	y = y(\xi,\eta). \\
\end{cases}
$
则
\begin{equation}
	\iint\limits_{D}f(x,y)\,\, \d x \d y= \iint\limits_{D‘}f[x(\xi,\eta),y(\xi,\eta)]\,|J|\, \d \xi \d \eta
\end{equation}
其中$J$是变换的雅克比行列式,即
\renewcommand{\arraystretch}{1.5}
\begin{equation*}
	|J|=\frac{D(x,y)}{D(\xi,\eta)}=
	\left| 
	\begin{array}{cc}
		\displaystyle \frac{\partial x}{\partial \xi} & \displaystyle \frac{\partial y}{\partial \xi} \\
		\displaystyle \frac{\partial x}{\partial \eta} & \displaystyle \frac{\partial y}{\partial \eta} 
	\end{array}
	\right| 
\end{equation*}
\renewcommand{\arraystretch}{1}

\subsection{二重积分的几何应用}
\vspace*{-1em}
\example[求隐函数的平面面积]
由二重积分的定义,记隐函数所围成的封闭曲面的面积为$S_D$,那么可以得到
\begin{equation}
	\iint\limits_{D} \, \, \d \sigma=\iint\limits_{D}\, \, \d x \d y=S_D
\end{equation}

\example[求空间曲面的面积]
若$S$由参数方程
$
\begin{cases}
	x = x(u,v),\\
	y = y(u,v),\\
	z= z(u,v).
\end{cases}
$
确定,记
$
\begin{cases}
	E = x_u^2 + y_u^2 +z_u^2,\\
	F = x_ux_v + y_uy_v + z_uz_v,\\
	G = x_v^2 + y_v^2 +z_v^2.
\end{cases}
$
则
\begin{equation}
	S=\iint\limits_{D'} \sqrt{EG-F^2}\, \, \d \sigma
\end{equation}

\section{三重积分}
\subsection{三重积分的定义}

\vspace*{-1em}

\defination[三重积分的定义]
设三元函数$f(x,y,z)$是定义在光滑曲面所围成的空间区域$\Omega$上,若对$\Omega$的任意分割$\{\Omega_1,\Omega_2,\cdots,\Omega_n\}$及任意选择的$(x_i,y_i,z_i) \in \Omega_i (i=1,2,\cdots,n)$,当$\lambda \rightarrow 0$时,极限
\begin{equation}
	\lim_{\lambda \rightarrow 0} \sum^{n}_{i=1} f(x_i,y_i,z_i)\,\Delta V_i
	\footnote[1]{$\lambda$表示$n$个区域$\Omega_i$其中的最大直径,$\Delta V_i$表示$\Omega_i$的最大体积.}
\end{equation}

总存在,则这个极限称为$f(x,y)$在$D$上的三重积分,记做
\begin{equation}
	\iiint\limits_{\Omega}f(x,y,z) \, \, \d V \quad \mbox{或} \quad  \iiint\limits_{\Omega}f(x,y,z) \, \, \d x \d y \d z
\end{equation}

\par 其中,$\Omega$称作积分区域,而$f(x,y,z)$称作被积函数,$\d V$称为体积元素.

\subsection{三重积分的性质}
三重积分的基本性质和二重积分完全类似。具体请参见\ref{二重积分的性质}.


\subsection{三重积分的计算}

\vspace*{-1em}

\theorem[投影法]
投影法可以认为是平行于$z$轴的线在投影区域内运动,连续地切割立体得到得到一条条立体内的线段$z_1(x,y) \rightarrow z_2(x,y)$,然后再把所有在投影区域内的所有线段进行积分,即
\begin{equation}
	\iiint\limits_{\Omega} \,\d x \d y  \d z = \iint\limits_{D_{xOy}}\,\d x \d y \int_{z_1(x,y)}^{z_2(x,y)}f(x,y,z) \,\d z
\end{equation}

\theorem[切片法]
切片法可以认为是用平行于$xOy$的平面$z=z_0\in [a,b]$去截立体得到的截面$D_{z_0}$,求出$D_{z_0}$后再把一片片截面积分拼成一个立体,即
\begin{equation}
	\iiint\limits_{\Omega} \,\d x \d y  \d z = \int_{a}^{b} \, \d z  \iint\limits_{D_{z}} f(x,y,z) \,\d x \d y
\end{equation}


\theorem[柱坐标变换]
柱坐标变换
$
\begin{cases}
	x =r \cos \theta,\\
	y = r \sin \theta ,\\
	z = z.
\end{cases}
$
下的三重积分计算公式为
\begin{equation}
	\iiint\limits_{\Omega} \,\d x \d y  \d z = \iiint\limits_{\Omega'} f(r\cos\theta,r\sin\theta,z)\,r \,\,\d r \d \theta  \d z
\end{equation}

\vspace*{-1em}

\theorem[球坐标变换]
球坐标变换
$
\begin{cases}
	x =\rho \,\sin \varphi \cos \theta,\\
	y = \rho \,\sin \varphi \sin \theta ,\\
	z = \rho \,\cos \varphi .
\end{cases}
$
下的三重积分计算公式为
\begin{equation}
	\iiint\limits_{\Omega} \,\d x \d y  \d z = \iiint\limits_{\Omega'} f(\rho \,\sin \varphi \cos \theta,\rho \,\sin \varphi \sin \theta , \rho \,\cos \varphi)\, \rho^2 \sin \varphi \,\,\d \rho \d \varphi  \d \theta 
\end{equation}

\vspace*{-1em}

\theorem[一般变换]
设$x,y,z$满足
$
\begin{cases}
	x = x(u,v,w),\\
	y = y(u,v,w), \\
	z = z(u,v,w).
\end{cases}
$
则
\begin{equation}
	\iiint\limits_{\Omega }f(x,y,z)\,\, \d x \d y= \iiint\limits_{\Omega‘}f[x(u,v,w),y(u,v,w),z(u,v,w)]\,|J|\, \,\d u \d v \d w
\end{equation}
其中$J$是变换的雅克比行列式,即
\renewcommand{\arraystretch}{1.5}
\begin{equation*}
	|J|=\frac{D(x,y,z)}{D(u,v,w)}=
	\left| 
	\begin{array}{ccc}
		\displaystyle \frac{\partial x}{\partial u} & \displaystyle \frac{\partial y}{\partial u} & \displaystyle \frac{\partial z}{\partial u} \\
		\displaystyle \frac{\partial x}{\partial v} & \displaystyle \frac{\partial y}{\partial v} & \displaystyle \frac{\partial z}{\partial v} \\
		\displaystyle \frac{\partial x}{\partial w} & \displaystyle \frac{\partial y}{\partial w} & \displaystyle \frac{\partial z}{\partial w} 
	\end{array}
	\right| 
\end{equation*}
\renewcommand{\arraystretch}{1}


\subsection{三重积分的几何应用}
\example[求立体的体积]
由三重积分的定义,记隐函数围成的封闭立体的体积为$V$,那么可以得到
\begin{equation}
	\iiint\limits_{\Omega} \, \, \d V=\iiint\limits_{\Omega}\, \, \d x \d y \d z=V
\end{equation}

%第八章-曲线积分和曲面积分
\thispagestyle{empty}
\chapter{曲线积分和曲面积分}
\section{第一型曲线积分}
\subsection{第一型曲线积分的基本概念}
\tdefination[第一型曲线积分的定义]
设$f(x,y,z)$在分段光滑的曲线$L$上有定义,对$L$任意分割成$n$段,第$i$段的弧长为$\Delta s_i$及在第$i$段任意选择的$(\xi_i,\eta_i,\zeta_i)$,当$\lambda = \max\limits_{1 \le i \le n} {\Delta s_i}\rightarrow 0$时,极限
\begin{equation}
\lim_{\lambda \rightarrow 0} \sum^{n}_{i=1} f(\xi_i,\eta_i,\zeta_i)\,\Delta s_i
\end{equation}
总存在,则这个极限称为函数$f(x,y,z)$沿曲线$L$的第一型曲线积分或弧长的曲线积分,记做
\begin{equation}
	\int_{L}f(x,y,z)\,\,\d s
\end{equation}
\par 其中,$L$称作积分曲线,而$f(x,y,z)$称作被积函数,$\d s$称为弧积分.

\subsection{第一型曲线积分的基本性质}
\ttheorem[第一型曲线积分的三个基本性质]
1.可拆可和性
\begin{equation}
\int_{L}[\,C_1f(x,y,z)+C_2g(x,y,z)\,]\,\,\d s =\int_{L}C_1f(x,y,z)\,\,\d s + \int_{L}C_2g(x,y,z)\,\,\d s
\end{equation}

\par 2.分段累加性$(L\rightarrow L_1,L_2,\cdots,L_m)$
\begin{equation}
\int_{L}f(x,y,z)\,\,\d s = \int_{L_1}f(x,y,z)\,\,\d s +\int_{L_2}f(x,y,z)\,\,\d s + \cdots +\int_{L_m}f(x,y,z)\,\,\d s
\end{equation}

\par 3.恒正性(无向性)
\begin{equation}
\int_{\widehat{AB}}f(x,y,z)\,\,\d s = \int_{\widehat{BA}}f(x,y,z)\,\,\d s 
\end{equation}

\subsection{第一型曲线积分的计算}
\ttheorem[直角坐标下平面曲线的积分]
若曲线$L$由$y=y(x)$确定,且$y=y(x)$在$[a,b]$上有连续导数,$f(x,y)$在$L$上连续,则
\begin{equation}
\int_{L}f(x,y)\,\,\d s =\int_{a}^{b}f[x,y(x)]\,\sqrt{1+[\,y'(x)\,]^2}\,\,\d x
\end{equation}

\theorem[参数方程下平面曲线的积分]
若曲线$L$由
$
\begin{cases}
x=x(t),\\
y=y(t)
\end{cases}
$
确定,且$x(t),y(t)$在$t \in [a,b]$上有连续导数,$f(x,y)$在$L$上连续,则
\begin{equation}
\int_{L}f(x,y)\,\,\d s =\int_{a}^{b}f[x(t),y(t)]\,\sqrt{[\,x'(t)\,]^2+[\,y'(t)\,]^2}\,\,\d t
\end{equation}

\theorem[参数方程下空间曲线的积分]
若曲线$L$由
$
\begin{cases}
x=x(t),\\
y=y(t),\\
z=z(t).
\end{cases}
$
确定,且$x(t),y(t),z(t)$在$t \in [a,b]$上有连续导数,$f(x,y,z)$在$L$上连续,则
\begin{equation}
\int_{L}f(x,y,z)\,\,\d s =\int_{a}^{b}f[x(t),y(t),z(t)]\,\sqrt{[\,x'(t)\,]^2+[\,y'(t)\,]^2+[\,z'(t)\,]^2}\,\,\d t
\end{equation}


\section{第二型曲线积分}
\subsection{第二型曲线积分的基本概念}
\tdefination[第一型曲线积分的定义]
$L$是从点$A$到点$B$的分段光滑有向曲线,向量函数$\bm{F}(x,y)=P(x,y)\bm{i}+Q(x,y)\bm{j}$在$L$上有定义,按照$L$的方向,对$L$的任意分割成$n$个有向的小线段$\overrightarrow{A_{i-1}A_i}$,记$\widehat{A_{i-1}A_i}$的弧长为$\Delta s_i$及在第$i$段任意选择的$(\xi_i,\eta_i,\zeta_i)$,当$\lambda = \max\limits_{1 \le i \le n} {\Delta s_i}\rightarrow 0$时,极限
\begin{equation}
\lim_{\lambda \rightarrow 0} \sum^{n}_{i=1} \bm{F}(\xi_i,\eta_i,\zeta_i) \cdot \overrightarrow{A_{i-1}A_i} \,\Delta s_i = \lim_{\lambda \rightarrow 0} \sum^{n}_{i=1} [P(\xi_i,\eta_i)\Delta x_i+Q(\xi_i,\eta_i)\Delta y_i] 
\end{equation}
总存在,则这个极限称为向量函数$\bm{F}(x,y)$沿曲线$L$从点$A$到点$B$的第二型曲线积分或对坐标的曲线积分,记做
\begin{equation}
\int_{\widehat{AB}}P\,\d x+Q\,\d y \huo \int_{\widehat{AB}}\bm{F}(x,y) \,\, \d \bm{r}
\end{equation}
\par 其中,$\d \bm{r}=(\d x,\d y)$,有向曲线$\widehat{AB}$称为积分路径.
\par 类似地,对于空间向量函数$\bm{F}(x,y,z)=P(x,y,z)\bm{i}+Q(x,y,z)\bm{j}+R(x,y,z)\bm{k}$,沿空间有向曲线$L$的第二型曲线积分为
\begin{equation}
\int_{L}P\,\d x+Q\,\d y+R\,\d z \huo \int_{L}\bm{F}(x,y,z) \,\, \d \bm{r}
\end{equation}

\subsection{第二型曲线积分的基本性质}
\ttheorem[第二型曲线积分的三个基本性质]
1.可拆可和性
\begin{equation}
\int_{\widehat{AB}}[k_1\bm{F}(M)+k_2\bm{G}(M)] \cdot \d \bm{r} = k_1\int_{\widehat{AB}}\bm{F}(M) \cdot \d \bm{r} +k_2 \int_{\widehat{AB}}\bm{G}(M) \cdot \d \bm{r} 
\end{equation}

2.分段累加性$\left( \widehat{AB} \rightarrow \widehat{AC} + \widehat{CB}\right) $
\begin{equation}
\int_{\widehat{AB}}\bm{F}(M) \cdot \d \bm{r} = \int_{\widehat{AC}}\bm{F}(M) \cdot \d \bm{r} +\int_{\widehat{CB}}\bm{F}(M) \cdot \d \bm{r}
\end{equation}

3.有向性
\begin{equation}
\int_{\widehat{AB}}\bm{F}(M) \cdot \d \bm{r} = -\int_{\widehat{BA}}\bm{F}(M) \cdot \d \bm{r}
\end{equation}

\subsection{第二型曲线积分的计算}
\ttheorem[平面曲线下第二型曲线积分的计算]
设曲线$L$的参数方程为
$
\begin{cases}
x = x(t),\\
y = y(t).
\end{cases}
$
其中$x(t),y(t)$有连续的一阶导数.当$t$单调地从$a$变化到$b$时,且$P(x,y),Q(x,y)$在$L$上连续,则
\begin{equation}
\int_{\widehat{AB}}P(x,y)\,\d x+Q(x,y)\,\d y = \int_{a}^{b}[\,P(x(t),y(t))\,x'(t) + Q(x(t),y(t))\,y'(t) \,]\,\d t
\end{equation}
特别地,当$y=g(x)$时,可以变为
\begin{equation}
\int_{\widehat{AB}}P(x,y)\,\d x+Q(x,y)\,\d y = \int_{a}^{b}[\,P(x,g(x)) + Q(x,g(x))\,g'(x) \,]\,\d x
\end{equation}

\ttheorem[空间曲线下第二型曲线积分的计算]
设曲线$L$的参数方程为
$
\begin{cases}
x = x(t),\\
y = y(t),\\
z =z(t).
\end{cases}
$
其中$x(t),y(t),z(t)$有连续的一阶导数,且$P(x,y,z),Q(x,y,z),R(x,y,z)$在$L$上连续,则
\begin{equation}
\begin{split}
&\quad \,\int_{\widehat{AB}}P(x,y,z)\,\d x+Q(x,y,z)\,\d y +R(x,y,z)\, \d z\\
&= \int_{a}^{b}[\,P(x(t),y(t),z(t))\,x'(t) + Q(x(t),y(t),z(t))\,y'(t) + R(x(t),y(t),z(t))\,z'(t) \,]\,\d t
\end{split}
\end{equation}

\theorem[格林公式]
对于闭区域的边界$L$规定其正方向$L^+$为使得沿这个方向前进时区域总在左侧,那么有
\begin{equation}
\oint_{L^+}P\,\d x+Q\,\d y = \iint\limits_{D}\left( \frac{\partial Q}{\partial x} -\frac{\partial P}{\partial y}\right) \,\, \d x \d y
\end{equation}
\par 特别地,当$\displaystyle\frac{\partial Q}{\partial x} =\frac{\partial P}{\partial y}$或$\displaystyle \oint_{L^+}P\,\d x+Q\,\d y =0$时,第二型曲面积分与积分路径无关。\\[0.5em]
此时可以找到一个函数$u(x,y)=P(x,y) \, \d x+Q(x,y)\,\d y$,即
\begin{equation}
\int_{\widehat{AB}}P\,\d x+Q\,\d y  =\int_{a}^{b} \d u = u(B)-u(A)
\end{equation}
\newpage

\subsection{第二型曲面积分与路径无关的判定}
\tinference[第二型曲面积分与路径无关的判定]
1. 用于判定路径有关的方法(也适用于判断$P\,\d x+Q\,\d y$在$D$上不存在原函数)
\par \quad \quad (1)\quad 存在一条分段光滑曲线$C\subset D,\displaystyle \oint_{C}P\,\d x+Q\,\d y\ne 0$.
\jg
\par \quad \quad (2)\quad 存在$(x,y)\in D,\displaystyle \frac{\partial Q(x,y)}{\partial x}\ne \frac{\partial P(x,y)}{\partial y}$.
\jg
\par 2. 用于判定路径无关的方法(方法(2),(3)可以用于判断$P\,\d x+Q\,\d y$在$D$上存在原函数)
\par \quad \quad (1)\quad 求得$u(x,y)$使得$\d u=P(x,y)\,\d x+Q(x,y)\,\d y$任意$(x,y)\in D$.
\jg
\par \quad \quad (2)\quad 若$D$是单连通的,又对于任意的$(x,y)\in D$都有$\displaystyle\frac{\partial Q}{\partial x} =\frac{\partial P}{\partial y}$.
\jg
\par \quad \quad (3)\quad 若$D=D_0\setminus\left\lbrace M_0\right\rbrace,\,D_0 $是单连通的,$M_0\in D_0$.若对于任意的$(x,y)\in D$都有$\displaystyle\frac{\partial Q}{\partial x} =\frac{\partial P}{\partial y}$,且存在一条包围点$M_0$的分段光滑闭曲线$C_0$,使得$\displaystyle \oint_{C_0}P\,\d x+Q\,\d y= 0$.


\section{第一型曲面积分}
\subsection{第一型曲面的基本概念}
\tdefination[第一型曲面积分的定义]
设$f(x,y,z)$在分片光滑的曲面$S$上有定义,对$S$任意分割成互补重叠的$n$片,第$i$段的面积为$\Delta S_i$及在第$i$段任意选择的$(\xi_i,\eta_i,\zeta_i)$,当$\lambda = \max\limits_{1 \le i \le n} \left\lbrace \Delta S_i\mbox{的直径}\right\rbrace \rightarrow 0$时,极限
\begin{equation}
\lim_{\lambda \rightarrow 0} \sum^{n}_{i=1} f(\xi_i,\eta_i,\zeta_i)\,\Delta S_i
\end{equation}
总存在,则这个极限称为函数$f(x,y,z)$在曲面$S$上的第一型曲面积分,记做
\begin{equation}
\iint_{S}f(x,y,z)\,\,\d S
\end{equation}
\par 其中,$S$称作积分曲面,而$f(x,y,z)$称作被积函数.特别地,如果积分曲面封闭,则记做
\begin{equation}
\oint_{S}f(x,y,z)\,\,\d S
\end{equation}

\subsection{第一型曲面积分的基本性质}
\ttheorem[第一型曲面积分的三个基本性质]
1.可拆可和性
\begin{equation}
\iint\limits_{S}[\,C_1f(x,y,z)+C_2g(x,y,z)\,]\,\,\d S =\iint\limits_{S}C_1f(x,y,z)\,\,\d S + \iint\limits_{S}C_2g(x,y,z)\,\,\d S
\end{equation}

\par 2.分片累加性$(S\rightarrow S_1,S_2,\cdots,S_i)$
\begin{equation}
\iint\limits_{S}f(x,y,z)\,\,\d S = \sum_{i=1}^{m}\iint\limits_{S_i}f(x,y,z)\,\,\d S
\end{equation}

\par 3.恒正性(无向性)
\begin{equation}
\iint\limits_{S}f(x,y,z)\,\,\d S \geq 0
\end{equation}

\subsection{第一型曲面积分的计算}
\ttheorem[二元函数下第一型曲面积分的计算]
对于二元函数$z=z(x,y),y=y(x,z),x=x(y,z)$,
\renewcommand\arraystretch{1.5}
\begin{equation}
\iint\limits_{\Sigma}f(x,y,z)\,\,\d S 
=
\left\lbrace 
\begin{array}{c}
\displaystyle \iint\limits_{D_{xy}}f(x,y,z(x,y))\,\sqrt{1+z_x^2+z_y^2}\,\,\d x\d y\quad \Sigma :z=z(x,y) \\
\displaystyle  \iint\limits_{D_{xz}}f(x,y(x,z),z)\,\sqrt{1+y_x^2+y_z^2}\,\,\d x\d z\quad \Sigma :y=y(x,z) \\
\displaystyle  \iint\limits_{D_{yz}}f(x(y,z),y,z)\,\sqrt{1+x_y^2+x_z^2}\,\,\d y\d z\quad \Sigma :x=x(y,z)
\end{array}
\right.
\end{equation}
\renewcommand\arraystretch{1}
提示:根据曲面方程的特点选择恰当的积分形式,$D$代表投影到某个坐标平面的平面区域.

\theorem[参数方程下第一型曲面积分的计算]
若$S$由参数方程
$
\begin{cases}
x = x(u,v),\\
y = y(u,v),\\
z= z(u,v).
\end{cases}
$
确定,记
$
\begin{cases}
E = x_u^2 + y_u^2 +z_u^2,\\
F = x_ux_v + y_uy_v + z_uz_v,\\
G = x_v^2 + y_v^2 +z_v^2.
\end{cases}
$
则
\begin{equation}
\iint\limits_{\Sigma}f(x,y,z)\,\,\d S =\iint\limits_{\Sigma}f(x(u,v),y(u,v),z(u,v))\,\sqrt{EG-F^2}\,\,\d u \d v
\end{equation}
\par 特别地,当参数方程是柱坐标变换方程时,
\begin{equation}
\d S = R \,\d \theta \d z
\end{equation}
当参数方程是球坐标变换方程时,
\begin{equation}
\d S = R^2\,|\sin \varphi| \,\,\d \theta \d \varphi
\end{equation}

\section{第二型曲面积分}
\subsection{第二型曲面积分的基本概念}
\defination[第二型曲面积分]
设$S$是一个分片光滑的双侧曲面,在曲面$S$上选定了一侧,记选定一侧的单位法向量为$\bm{n}(P)$.假设在$S$上给定了一个向量函数$\bm{F}(x,y,z)$.我们将$S$分割成$n$个不相重叠的小曲面片$\Delta S_i(i=1,2,\cdots,n)$,其面积也用$\Delta S_i$表示.在$\Delta S_i$上任意取一点$M_i(\xi_i,\eta_i,\zeta_i)$,如果$\lambda = \max\limits_{1 \le i \le n} \left\lbrace \Delta S_i\mbox{的直径}\right\rbrace \rightarrow 0$时,极限
\begin{equation}
\lim_{\lambda \rightarrow 0} \sum^{n}_{i=1} \bm{F}(\xi_i,\eta_i,\zeta_i)\cdot \bm{n}(\xi_i,\eta_i,\zeta_i)\,\,\Delta S_i
\end{equation}
总存在,则这个极限称为向量函数$\bm{F}(x,y,z)$在曲面$S$上的第二型曲面积分,记做
\begin{equation}
\iint\limits_{S} \bm{F}(x,y,z)\cdot \bm{n}(x,y,z)\,\,\Delta S
\end{equation}

\tdefination[曲面的方向]
对于不同曲面的方程形式,曲面方向判定见下表\ref{曲面方向的判定}.
\begin{table}[!htb]
	\centering
	\setlength{\tabcolsep}{10mm}{
		\begin{tabular}{ccc}
			\toprule[2pt] 
			\rowcolor[gray]{0.9}   曲面方程形式 & 法向量  & 方向的规定\\  
			\midrule[1.2pt]
			\multirow{2}{*}{$z=z(x,y)$} & $\bm{n}_1 = (-z_x,-z_y,1) $ &上侧\\
			\cline{2-3}
			& $\bm{n}_2 = (z_x,z_y,-1) \hspace{0.8em}$ & 下侧\\
			\midrule[1.2pt]
			\multirow{2}{*}{$y=y(x,z)$} & $\bm{n}_1 = (-y_x,1,-y_z)$ &右侧\\
			\cline{2-3}
			& $\bm{n}_2= (y_x,-1,y_z)\hspace{0.8em} $ & 左侧\\
			\midrule[1.2pt]
			\multirow{2}{*}{$x=x(y,z)$} & $\bm{n}_1 = (1,-x_y,-x_z) $ &前侧\\
			\cline{2-3}
			& $\bm{n}_2= (-1,x_y,x_z)\hspace{0.8em} $ & 后侧\\
			\bottomrule[2pt]
		\end{tabular}  
	}
	\caption{曲面方向的判定}
	\label{曲面方向的判定}
\end{table} 
\par 对于曲面而言,法向量指向曲面的内部为曲面内侧;法向量指向曲面的外部为曲面外侧.



\subsection{第二型曲面积分的基本性质}
\ttheorem[第二型曲面积分的三个基本性质]
1.可拆可和性
\begin{equation}
\iint\limits_{S}[\,C_1\bm{F}_1+C_2\bm{F}_2\,]\,\,\d \bm{S} =\iint\limits_{S}C_1\bm{F}_1\,\,\d \bm{S}  + \iint\limits_{S}C_2\bm{F}_2\,\,\d \bm{S} 
\end{equation}

\par 2.分片累加性$(S\rightarrow S_1+S_2)$
\begin{equation}
\iint\limits_{S}\bm{F}\,\,\d \bm{S}  = \iint\limits_{S_1}\bm{F}\,\,\d \bm{S}  +\iint\limits_{S_2}\bm{F}\,\,\d \bm{S} 
\end{equation}

\par 3.有向性
\begin{equation}
\iint\limits_{S^+}\bm{F}\,\,\d \bm{S} = -\iint\limits_{S^-}\bm{F}\,\,\d \bm{S} 
\end{equation}

\subsection{第二型曲面积分与第一型曲面积分的关系}
\ttheorem[第二型曲面积分与第一型曲面积分的关系]
设向量函数$\bm{F}(P(x,y,z),Q(x,y,z),R(x,y,z))$的单位法向量为$\bm{n}=(x,y,z)$,其方向余弦为
\[
\cos \alpha (x,y,z),\cos \beta (x,y,z),\cos \gamma(x,y,z)
\]
则二重积分可写成
\begin{equation}
\begin{split}
\iint\limits_{S}\bm{F}\cdot \bm{n}\,\,\d S&=\iint\limits_{S}(P\cos \alpha +Q\cos \beta +R\cos \gamma )\,\,\d S\\
&=\iint\limits_{S}P \,\d y\d z+Q\,\d z\d x+R\,\d x\d y
\end{split}
\end{equation}

\subsection{第二型曲面积分的计算}
\begin{table}[h]
	\centering
	\renewcommand{\arraystretch}{1.6}
	\setlength{\tabcolsep}{20mm}{
		\begin{tabular}{cc}
			\toprule[1.5pt] 
			\rowcolor[gray]{0.9}   曲面方程形式 &  方向余弦 \\  
			\midrule
			$z=z(x,y)$& $\displaystyle \frac{\pm 1}{\sqrt{1+z_x^2+z_y^2}}\, (-z_x,-z_y,1)$\\
			\hline
			$y=y(x,z)$ & $\displaystyle \frac{\pm 1}{\sqrt{1+y_x^2+y_z^2}} \,(-y_x,1,-y_z)$ \\
			\hline
			$x=x(y,z)$& $\displaystyle \frac{\pm 1}{\sqrt{1+x_y^2+x_z^2}} \,(1,-x_y,-x_z)$ \\
			\bottomrule[1.5pt]
		\end{tabular}  
	}
	\caption{不同的曲面方程的方向余弦}
	\renewcommand{\arraystretch}{1}
	\label{方向余弦}
\end{table} 

\ttheorem[直接转换为二重积分计算]
由上表\ref{方向余弦}并利用公式$\displaystyle \iint\limits_{S}\bm{F}\cdot \bm{n}\,\,\d S=\iint\limits_{S}(P\cos \alpha +Q\cos \beta +R\cos \gamma )\,\,\d S$可以得到转换公式如下表.
\begin{table}[h]
	\centering
	\renewcommand{\arraystretch}{1.6}
	\setlength{\tabcolsep}{3mm}{
		\begin{tabular}{cc}
			\toprule[1.5pt] 
			\rowcolor[gray]{0.9}   曲面方程形式 & 结果\\  
			\midrule
			$z=z(x,y)$ &$\displaystyle \pm\iint\limits_{D_{xy}}[\,P(x,y,z(x,y))(-z_x) +Q(x,y,z(x,y))(-z_y) +R(x,y,z(x,y))\,]\,\,\d \sigma $\\
			\hline
			$y=y(x,z)$ &$\displaystyle \pm\iint\limits_{D_{xz}}[\,P(x,y,z(x,y))(-y_x) +Q(x,y,z(x,y)) +R(x,y,z(x,y))(-y_z)\,]\,\,\d \sigma $\\
			\hline
			$x=x(y,z)$ &$\displaystyle \pm\iint\limits_{D_{xy}}[\,P(x,y,z(x,y)) +Q(x,y,z(x,y))(-x_y) +R(x,y,z(x,y))(-x_z)\,]\,\,\d \sigma $\\
			\bottomrule[1.5pt]
		\end{tabular}  
	}
	\caption{转换为二重积分计算的计算公式}
	\renewcommand{\arraystretch}{1}
	\label{第二型曲面积分的直接计算}
\end{table} 
\par 注:上表\ref{第二型曲面积分的直接计算}中正负号的选取与方向余弦的$``1"$的符号相同.$D$表示投影到相应坐标平面的平面区域.

\inference[转换为二重积分计算第二型曲面积分]
\noindent \quad 总结\quad ``一投、二代、三定号"
\par \quad 1. 将曲面$\Sigma$的方程写成上述三种形式的其中一种.
\par \quad 2. 将曲面$\Sigma$投影到相应的坐标平面,得到投影区域$D$.
\par \quad 3. 将$x=x(y,z)$或$y=y(x,z)$或$z=z(x,y)$代入被积函数,将$\Sigma $换成$D$.
\par \quad 4. 根据方向余弦确定侧向进而确定二重积分的符号.

\quad 提示:若投影区域面积为0,则相应的二重积分为0.
\newpage
\theorem[高斯公式]
表达了空间闭区域上的三重积分与其边界曲面上的曲面积分之间的关系, 这个关系可陈述如下:
\par 设空间闭区域$\Omega $是由分片光滑的闭曲面$\Sigma $所围成,若函数$P(x, y, z),Q(x, y, z),R(x, y, z)$在$\Omega $上具有一阶连续偏导数.则有
\begin{equation}
\oiint\limits_{S^+}P \,\d y\d z+Q\,\d z\d x+R\,\d x\d y=\iiint\limits_{\Omega}\left( \frac{\partial P}{\partial x}+\frac{\partial Q}{\partial y}+\frac{\partial R}{\partial z}\right)\,\d V 
\end{equation}
\par 其中$S^+$是曲面$S$的外侧.\\
注:若$S$不是封闭曲面,可利用补片法,常用平行于坐标面的平面来补片.
\par 若$S$不是外侧,则在积分前面加负号,所求的结果和外侧的结果互为相反数.

\example[曲面积分求体积]
若$\Sigma$封闭且方向取外侧,则
\begin{equation}
\begin{split}
\oiint\limits_{\Sigma^+}x \,\d y\d z+y\,\d z\d x+z\,\d x\d y=\iiint\limits_{\Omega}\left(1+1+1\right)\,\d V =3\iiint\limits_{\Omega} \,\d V =3V
\end{split}
\end{equation}

\section{斯托克斯公式}
\ttheorem[斯托克斯公式]
设$\Gamma$为分段光滑的空间有向闭曲线,$\Sigma$是以$\Gamma$为边界的分片光滑的有向曲面,$\Gamma$的正向与$\Sigma$的侧符合右手法则, 若函数$P(x, y, z),Q(x, y, z),R(x, y, z)$在曲面$\Sigma$(连同边界$\Gamma$)上具有一阶连续偏导数,则有
\begin{equation}
\oint_{L^+}P\,\d x+Q\,\d y+R\,\d z=\iint\limits_{S^+}\left( \frac{\partial R}{\partial y}-\frac{\partial Q}{\partial z}\right)\,\d y\d z+\left( \frac{\partial P}{\partial z}-\frac{\partial R}{\partial x} \right) \,\d z \d x +\left( \frac{\partial Q}{\partial x}-\frac{\partial P}{\partial y}\right) \, \d x\d y. 
\end{equation}
\par 斯托克斯公式是格林公式的推广.格林公式表达了平面闭区域上的二重积分与其边界曲线上的曲线积分间的关系;而斯托克斯公式则把曲面$\Sigma$上的曲面积分与沿着$\Sigma$的边界曲线$\Gamma$的曲线积分联系起来.
\par 为了便于记忆,也可写成行列式的形式
\begin{equation}
\renewcommand{\arraystretch}{1.5}
\iint\limits_{S^+}
\left| 
\begin{array}{ccc}
\d y \d z &\d z \d x &\d x \d y \\
\displaystyle \frac{\partial }{\partial x} &\displaystyle \frac{\partial }{\partial y} & \displaystyle \frac{\partial }{\partial z}\\
P & Q & R
\end{array}
\right| 
\huo
\iint\limits_{S^+}
\left| 
\begin{array}{ccc}
\cos \alpha & \cos \beta &\cos \gamma\\
\displaystyle \frac{\partial }{\partial x} &\displaystyle \frac{\partial }{\partial y} & \displaystyle \frac{\partial }{\partial z}\\
P & Q & R
\end{array}
\right| 
\d S
	\renewcommand{\arraystretch}{1}
\end{equation}

\section{积分的特点}
\subsection{积分区域的可代入性}
当\ds[积分区域是确定方程(等式)],而不是非确定方程(含有不等号)的时候可以将积分区域的函数代入被积函数,或者被积函数构造成积分区域的方程.通常\ds[曲线积分和曲面积分都可以直接代入积分区域],因为这些积分的积分区域通常都是由确定的方程来决定的。

\subsection{多元函数的奇偶性}
设三元函数$f(x,y,z)$,则定义函数的奇偶性如下表\ref{多元函数的奇偶性}.
\begin{table}[h]
	\centering
	\renewcommand{\arraystretch}{1}
	\setlength{\tabcolsep}{6mm}{
		\begin{tabular}{ccc}
			\toprule[1.5pt] 
			\rowcolor[gray]{0.9}   满足等式  & 函数奇偶性 & 图像特点 \\  
			\midrule
			$f(x,y,z)=-f(-x,y,z)$& $f(x,y,z)$是关于$x$的奇函数&无\\
			\hline
			$f(x,y,z)=f(-x,y,z)$ & $f(x,y,z)$是关于$x$的偶函数&$f(x,y,z)$的图形关于$Ozy$平面对称\\
			\hline
			$f(x,y,z)=-f(x,-y,z)$& $f(x,y,z)$是关于$y$的奇函数&无\\
			\hline
			$f(x,y,z)=f(x,-y,z)$ & $f(x,y,z)$是关于$y$的偶函数&$f(x,y,z)$的图形关于$Ozx$平面对称\\
			\hline
			$f(x,y,z)=-f(x,y,-z)$& $f(x,y,z)$是关于$z$的奇函数&无\\
			\hline
			$f(x,y,z)=f(x,y,-z)$ & $f(x,y,z)$是关于$z$的偶函数&$f(x,y,z)$的图形关于$Oxy$平面对称\\
			\bottomrule[1.5pt]
		\end{tabular}  
	}
	\caption{多元函数的奇偶性}
	\renewcommand{\arraystretch}{1}
	\label{多元函数的奇偶性}
\end{table} 

\section{积分的轮换对称性}
二元函数的轮换对称性
\begin{equation}
f(x,y)=f(y,x)
\end{equation}
其几何意义是$f(x,y)$的图形关于$y=x$对称.
\par 三元函数的轮换对称性
\begin{equation}
f(x,y,z)=f(y,x,z)=f(x,z,y)
\end{equation}
\begin{table}[!htb]
	\centering
	\renewcommand{\arraystretch}{1.8}
	\setlength{\tabcolsep}{9mm}{
		\begin{tabular}{cc}
			\toprule[2pt] 
			\rowcolor[gray]{0.9}   积分类型  & 轮换表达式 \\  
			\midrule[1.3pt]
			二重积分 &  $\displaystyle \iint\limits_{D}f(x,y)\,\,\d \sigma =\iint\limits_{D}f(y,x)\,\,\d \sigma = \frac{1}{2}\iint\limits_{D}\left[\, f(y,x)+f(x,y) \, \right]\,\,\d \sigma$\\
			\hline
			三重积分 &$\displaystyle \iiint\limits_{S}f(x,y,z)\,\,\d V =\iiint\limits_{S}f(y,x,z)\,\,\d V = \iiint\limits_{S}f(z,x,y)\,\,\d V $ \\
			\hline
			第一型曲线积分 & $\displaystyle \int_{L}f(x,y)\,\,\d \sigma =\int_{L}f(y,x)\,\,\d \sigma = \frac{1}{2}\int_{L}\left[\, f(y,x)+f(x,y) \, \right]\,\,\d \sigma$ \\
			\hline
			第二型曲线积分 &  $\displaystyle \int_{L}f(x,y)\,\,\d x+\int_{L}f(y,x)\,\,\d y= 0$ \\
			\hline
			第一型曲面积分 &  $\displaystyle \iint\limits_{S}f(x,y,z)\,\,\d S =\iint\limits_{S}f(y,x,z)\,\,\d S = \iint\limits_{S}f(z,x,y)\,\,\d S $ \\
			\hline
			第二型曲面积分 &  $\displaystyle \iint\limits_{S}f(x,y,z)\,\,\d y \d z =\iint\limits_{S}f(y,x,z)\,\,\d x\d z = \iint\limits_{S}f(z,x,y)\,\,\d x\d y $ \\
			\bottomrule[2pt]
		\end{tabular}  
	}
	\caption{积分的轮换对称性}
	\renewcommand{\arraystretch}{1}
	\label{积分的轮换对称性}
\end{table} 

\section{积分的奇偶对称性}
设被积函数为$f(x,y)$或$f(x,y,z)$,根据函数的奇偶性,可以得到积分的对称性
\begin{table}[!htb]
	\centering
	\renewcommand{\arraystretch}{1.6}
	\setlength{\tabcolsep}{9mm}{
		\begin{tabular}{cccc}
			\toprule[2pt] 
			\rowcolor[gray]{0.9}   积分类型  & 积分区域  &  被积函数 & 化简结果 \\  
			\midrule[1.3pt]
			\multirow{4}{*}{二重积分} & \multirow{2}{*}{关于$y$轴对称}  & 关于$x$的偶函数 & $\displaystyle I=2\iint\limits_{D_1} f(x,y)\,\, \d \sigma $\\
			\cline{3-4}
			&  & 关于$x$的奇函数 & $\displaystyle I=0$\\
			\cline{2-4}
			& \multirow{2}{*}{关于$x$轴对称}  & 关于$y$的偶函数 & $\displaystyle I=2\iint\limits_{D_1} f(x,y) \,\,\d \sigma $\\
			\cline{3-4}
			&  & 关于$y$的奇函数 & $\displaystyle I=0$\\
			\midrule[1.3pt]
			\multirow{6}{*}{三重积分} & \multirow{2}{*}{关于$Oxy$对称}  & 关于$z$的偶函数 & $\displaystyle I=2\iiint\limits_{\Omega _1} f(x,y) \,\,\d V$\\
			\cline{3-4}
			&  & 关于$z$的奇函数 & $\displaystyle I=0$\\
			\cline{2-4}
			& \multirow{2}{*}{关于$Oxz$对称}  & 关于$y$的偶函数 & $\displaystyle I=2\iiint\limits_{\Omega _1} f(x,y) \,\,\d V$\\
			\cline{3-4}
			&  & 关于$y$的奇函数 & $\displaystyle I=0$\\
			\cline{2-4}
			& \multirow{2}{*}{关于$Oyz$对称}  & 关于$x$的偶函数 & $\displaystyle I=2\iiint\limits_{\Omega _1} f(x,y) \,\,\d V$\\
			\cline{3-4}
			&  & 关于$x$的奇函数 & $\displaystyle I=0$\\
			\midrule[1.3pt]
			\multirow{4}{*}{\makecell[c]{第一型\\曲线积分\\(平面)}} & \multirow{2}{*}{关于$y$轴对称}  & 关于$x$的偶函数 & $\displaystyle I=2\int_{L_1} f(x,y) \,\,\d s $\\
			\cline{3-4}
			&  & 关于$x$的奇函数 & $\displaystyle I=0$\\
			\cline{2-4}
			& \multirow{2}{*}{关于$x$轴对称}  & 关于$y$的偶函数 & $\displaystyle I=2\int_{L_1} f(x,y)\,\,\d s  $\\
			\cline{3-4}
			&  & 关于$y$的奇函数 & $\displaystyle I=0$\\
			\midrule[1.3pt]
			\multirow{6}{*}{\makecell[c]{第一型\\曲线积分\\(空间)}} & \multirow{2}{*}{关于$Oxy$对称}  & 关于$z$的偶函数 & $\displaystyle I=2\int_{L_1} f(x,y,z)\,\,\d s  $\\
			\cline{3-4}
			&  & 关于$z$的奇函数 & $\displaystyle I=0$\\
			\cline{2-4}
			& \multirow{2}{*}{关于$Oxz$对称}  & 关于$y$的偶函数 & $\displaystyle I=2\int_{L_1} f(x,y,z)\,\,\d s  $\\
			\cline{3-4}
			&  & 关于$y$的奇函数 & $\displaystyle I=0$\\
			\cline{2-4}
			& \multirow{2}{*}{关于$Oyz$对称}  & 关于$x$的偶函数 & $\displaystyle I=2\int_{L_1} f(x,y,z)\,\,\d s  $\\
			\cline{3-4}
			&  & 关于$x$的奇函数 & $\displaystyle I=0$\\
			\bottomrule[2pt]
		\end{tabular}  
	}
	\caption{积分的奇偶对称性\uppercase\expandafter{\romannumeral1}}
	\renewcommand{\arraystretch}{1}
	\label{积分的奇偶对称性1}
\end{table} 
\newpage 
续表
\begin{table}[!htb]
	\centering
	\renewcommand{\arraystretch}{1.6}
	\setlength{\tabcolsep}{9mm}{
		\begin{tabular}{cccc}
			\toprule[2pt] 
			\rowcolor[gray]{0.9}   积分类型  & 积分区域  &  被积函数 & 化简结果 \\  
			\midrule[1.3pt]
			\multirow{6}{*}{\makecell[c]{第一型\\曲面积分}} & \multirow{2}{*}{关于$Oxy$对称}  & 关于$z$的偶函数 & $\displaystyle I=2\iint\limits_{S_1} f(x,y,z) \,\,\d S$\\
			\cline{3-4}
			&  & 关于$z$的奇函数 & $\displaystyle I=0$\\
			\cline{2-4}
			& \multirow{2}{*}{关于$Oxz$对称}  & 关于$y$的偶函数 & $\displaystyle I=2\iint\limits_{S_1} f(x,y,z) \,\,\d S$\\
			\cline{3-4}
			&  & 关于$y$的奇函数 & $\displaystyle I=0$\\
			\cline{2-4}
			& \multirow{2}{*}{关于$Oyz$对称}  & 关于$x$的偶函数 & $\displaystyle I=2\iint\limits_{S_1} f(x,y,z) \,\,\d S$\\
			\cline{3-4}
			&  & 关于$x$的奇函数 & $\displaystyle I=0$\\
			\midrule[1.3pt]
			\multirow{6}{*}{\makecell[c]{第二型\\曲线积分}} & \multirow{2}{*}{关于$Oxy$对称}  & 关于$z$的偶函数 & $\displaystyle I=0$\\
			\cline{3-4}
			&  & 关于$z$的奇函数 & $\displaystyle I=2\int_{L_1} f(x,y,z)\,\,\d z  $\\
			\cline{2-4}
			& \multirow{2}{*}{关于$Oxz$对称}  & 关于$y$的偶函数 & $\displaystyle I=0$\\
			\cline{3-4}
			&  & 关于$y$的奇函数 & $\displaystyle I=2\int_{L_1} f(x,y,z)\,\,\d y  $\\
			\cline{2-4}
			& \multirow{2}{*}{关于$Oyz$对称}  & 关于$x$的偶函数 & $\displaystyle I=0$\\
			\cline{3-4}
			&  & 关于$x$的奇函数 & $\displaystyle I=2\int_{L_1} f(x,y,z)\,\,\d y  $\\
			\midrule[1.3pt]
			\multirow{6}{*}{\makecell[c]{第二型\\曲面积分}} & \multirow{2}{*}{关于$Oxy$对称}  & 关于$z$的偶函数 &  $\displaystyle I=0$\\
			\cline{3-4}
			&  & 关于$z$的奇函数 &$\displaystyle I=2\iint\limits_{S_1} f(x,y,z)\,\,\d x\d y  $\\
			\cline{2-4}
			& \multirow{2}{*}{关于$Oxz$对称}  & 关于$y$的偶函数 &  $\displaystyle I=0$\\
			\cline{3-4}
			&  & 关于$y$的奇函数 & $\displaystyle I=2\iint\limits_{S_1} f(x,y,z)\,\,\d z\d x  $\\
			\cline{2-4}
			& \multirow{2}{*}{关于$Oyz$对称}  & 关于$x$的偶函数 &  $\displaystyle I=0$\\
			\cline{3-4}
			&  & 关于$x$的奇函数 &  $\displaystyle I=2\iint\limits_{S_1} f(x,y,z)\,\,\d y\d z  $\\
			\bottomrule[2pt]
		\end{tabular}  
	}
	\caption{积分的奇偶对称性\uppercase\expandafter{\romannumeral2}}
	\renewcommand{\arraystretch}{1}
	\label{积分的奇偶对称性2}
\end{table} 
\newpage






%第九章-常微分方程
\chapter{热力学第一定律} 
\thispagestyle{empty}
\section{功 \quad 热量 \quad 热力学第一定律}
\thispagestyle{empty}
从微观上看(在力学上就是把系统当分子组成的质点系处理),系统和外界交换能量的过程有两种情况:
\par \dy [功]{G} 
系统和外界的边界发生宏观位移,这种情况下外界对系统做宏观功,简称功。它实质上是系统和外界交换的分子有规則运动的能量。
\par \dy [热量]{RL} 
系统和外界的分子通过碰撞对系统做微观功而交换无规则运动的能量。这种交换只有在系统和外界分子的无规则运动平均动能不同,即在系统和外界的温度不同时才能发生。这种交换方式叫热传递,所传递的无规则透动能量的多少叫热量。
\par \dy[内能]{NN}
系统中所有分子的无规则运动能是的总和。\jg
\par \dy[热力学第一定律]{RLXDYDL}
\margin{\\ \\ \kg 热力学第一定律是普遍的能量守恒定律的“初级形式“。它适用于系统的任意过程。}
从微观上应用对质心系的机械能守恒定律,以$A$表示外界对系统做的宏观功,以$Q$表示外界对系统做的徼观功,即输入系统的热量。以$E$表示系统的内能,则有
$$A'+Q=\Delta E$$
常用$A$表示系统对外界做的功。由于$A'=-A$,即
\begin{equation}
\eq[Q=\Delta E+A]
\end{equation}

\section{准静态过程}
\dy[准静态过程]{ZJTGC} 过程进行中的每一时刻,系统的状态都无限接近与平衡态。``无限缓慢"的过程就是准静态过程。
\margin{准静态过程可以用状态图上的曲线表示}
\par 在无摩擦的准静态过程中系统对外做的``体积功"为
\margin{\\ \kg 注:功是``过程量"}
\begin{equation}
\eq[A=\int_{V_1}^{V_2}p\,\d V]
\end{equation}

\section{热容[量]}
热量也是``过程量",和温度变化有关的热量可用热容量计算。
\par 对于固体或液体,如果吸热仅引起温度的升高,则
\begin{equation}
\eq[Q=cm\Delta T]
\end{equation}
\par \dy[潜热]{QR} 物体在相变时所吸收或放出的热量。\jg
\par \dy[融化热]{RHR} 固体融化时吸收的热量。\jg
\par \dy[汽化热]{QHR} 液体在沸点汽化时吸收的热量。\jg
\par \dy[摩尔定压热容]{MEDYRR}
\begin{equation}
\eq[C_{p,m}=\frac{1}{\nu}\left(\frac{\d Q}{\d T} \right)_p ]
\end{equation}

\par \dy[摩尔定体热容]{MEDTRR}
\begin{equation}
\eq[C_{V,m}=\frac{1}{\nu}\left(\frac{\d Q}{\d T} \right)_V ]
\end{equation}
对于理想气体,
\margin{$i$是气体分子的自由度,可以参看表\ref{气体分子的自由度}.理想气体的内能改变可以直接由定体热容求出:$\Delta E =E_2-E_1 = \nu \, C_{V,m} \Delta T$}
\begin{equation}
\eq[C_{p,m}=\frac{i+2}{2}R],\,\,\eq[C_{V,m}=\frac{i}{2}R]
\end{equation}

\dy[麦耶公式]{MYGS}
\begin{equation}
\eq[C_{p,m}-C_{V,m}=R]
\end{equation}

\dy[比热比]{BRB}
\begin{equation}
	\eq[\gamma = \frac{C_{p,m}}{C_{V,m}}=\frac{i+2}{i}]
\end{equation}


\section{绝热过程}\label{绝热过程}
特点:$Q=0$,热力学第一定律给出$A=\Delta E$.\jg
\par \dy[理想气体的准静态绝热过程]{LXQTDZJTJRGC}\jg
\margin{泊松公式的变形方程有\scriptsize{$$TV^{\gamma -1} =C_2$$$$p^{\gamma - 1}T^{-\gamma } = C_3$$}}
\par \quad \quad \dy[泊松公式]{PSGS}
\begin{equation}
\eq[pV^{\gamma} =C_1]
\end{equation}
\par \quad \quad 对外做的功
\begin{equation}
\eq[A=\int_{V_1}^{V_2}p\,\d V=\frac{1}{\gamma -1}(p_2V_2-p_1V_1)]
\end{equation}
\par 而由泊松公式
\begin{equation*}
	p_1V_1^{\gamma} =p_2V_2^{\gamma} 
\end{equation*}
\par 那么上式可写为
\begin{equation*}
\begin{split}
A&=\frac{1}{\gamma -1}(p_2V_2-p_1V_1)\\
&=\frac{p_1V_1}{\gamma -1}\left( \frac{p_2V_2}{p_1V_1} -1\right) \\
&=\frac{p_1V_1}{\gamma -1}\left[ \left( \frac{V_2}{V_1}\right)^{\gamma -1}-1 \right] 
\end{split}
\end{equation*}
\par \quad \quad 绝热线比等温线陡,即在两条曲线中前者的斜率较大。\vspace*{2em}

\par \dy[绝热自由碰膨胀过程]{JRZYPZGC}\jg
\par \quad \quad 气体想真空的膨胀,是一种非准静态的过程。理想气体经绝热自由膨胀后内能不变,即$E=0$,说明$\Delta T= 0$。而绝热过程有$Q=0$,则$A=0$。

\section{几个热力学过程分析}
\margin{对于等温过程,由理想气体状态方程可得:
	\scriptsize{
		\begin{equation*}
		p=\frac{\nu \,RT}{V}
		\end{equation*}
	}
	那么,
	\scriptsize{
		\begin{equation*}
		\begin{split}
		A&=\int_{V_1}^{V_2}p\,\d V\\
		&=\int_{V_1}^{V_2}\frac{\nu \,RT}{V} \d V\\
		&=\nu RT\frac{\ln V_2}{\ln V_1}\\
		&=\nu RT\frac{\ln p_2}{\ln p_1}
		\end{split}
		\end{equation*}
	}
}
\margin{\\[3em] \kg 绝热过程和自由膨胀过程的具体推导参见\ref{绝热过程}节}
\margin{\\[5em] \kg 对于一般过程,系统对外界做功$A$可以用$p-V$图的面积计算,其内能变化为
	\scriptsize{
		\begin{equation*}
		\begin{split}
		\Delta E&=\nu \, C_{V,m} \Delta T\\
		&=\frac{i}{2}\,(p_2V_2-p_1V_1)
		\end{split}
		\end{equation*}
	}
}
\begin{table}[h]
	\centering
	\caption{热力学过程能量分析}
	\renewcommand\arraystretch{2}
	\setlength{\tabcolsep}{2.5mm}{
		\begin{tabular}{cccc}
			\toprule[2pt] 
			& & & \vspace*{-4.5em}\\
			   热力学过程 &对外界做功$A$ & 从外界吸收的能量$Q$   &系统内能变化$\Delta E$\\  
			\midrule[1.5pt]
			等体过程 & 0 & $\nu \,C_{V,m} \Delta T$  & $Q$ \\
			等压过程 & $ p \Delta V$ & $\nu \,C_{p,m} \Delta T$  & $Q-A$\\
			等温过程 & $\displaystyle \nu RT\,\frac{\ln p_2}{\ln p_1}$ & $-A$ & 0\\
			自由膨胀 & 0 & 0 & 0\\
			绝热过程 &$\displaystyle \frac{p_1V_1}{\gamma -1}\left[ \left( \frac{V_2}{V_1} \right)^{\gamma -1}-1 \right] $  & 0  &$ -A$\\
			循环过程 & $A$ &$-A$ &0\\
			一般过程 & $\displaystyle \int_{V_1}^{V_2}p\,\d V$ & $\Delta E -A$ & $\displaystyle \frac{i}{2}\,(p_2V_2-p_1V_1)$\\
			\bottomrule[2pt]
		\end{tabular}  
	}
	\label{热力学过程分析}
	\renewcommand\arraystretch{1}
\end{table} 



\section{循环过程}

\dy[工质]{GZ} 在热机中被利用来吸收热量并对外做功的物质.\jg

\par \dy[循环]{XH} 一个系统经历一系列变化后又回到初始状态的整个过程.\jg

\par 循环过程的特点:由于系统状态复原,所以$\Delta E = 0$,由热力学第一定律可知$Q=A$,即系统从外界吸收的净热量等于系统对外做的净功。\jg

\par \dy[做功循环]{ZGXH} 系统从高温热库吸热$Q_1$,对外做净功$A$,向低温热库放热$Q_2=Q_1-A.$循环的效率为
\begin{equation}
\eq[\eta = \frac{A}{Q_1}=1-\frac{Q_2}{Q_1}]
\end{equation}

\par \dy[制冷循环]{ZLXH} 系统从低温热库吸热$Q_2$,接受外界对它做的功$A$,向高温热库放热$Q_1=A+Q_2.$制冷系数
\begin{equation}
\eq[\omega = \frac{Q_2}{A}=\frac{Q_2}{Q_1-Q_2}]
\end{equation}

\section{卡诺循环}
\dy[卡诺循环]{KNXH} 系统只在两个恒温热库$(T_2>T_1)$进行热交换的准静态循环过程(无摩擦),循环效率和制冷系数分别为
\begin{equation}
\eq[\eta_C = 1 -\frac{T_2}{T_1}],\quad \eq[\omega_C=\frac{T_2}{T_1-T_2}]
\end{equation}

\section{热力学温标}
\dy[热力学温标]{RLXWB} 利用卡诺循环定义的温标:
\begin{equation}
\frac{T_1}{T_2}=\frac{Q_1}{Q_2}
\end{equation}
定点取水的三相点温度为$T_3=273.16\,$K.

\section{解题方法}
1. 认系统\quad 过程要确定题目中要作为分析对象的系统。这同时也就确定了外界。\jg
\par 2. 辨状态\quad 即要辨别清楚所选定的系统的初状态和末状态以及相应的状态参量,并对同一状态参量$p,V,T$等加注同一数字下标,如$p_1,V_1,T_1$等。对所关注的状态要弄清楚是否为平衡态。对理想气体的平衔状态的各状态参量才能应用理想气体状态方程。内能表示式也只能用于平衡态。\jg
\par 3. 明过程\quad 即要明确所选定的系统经历的是什么过程。首先要分清是否是准静态过程。有很多公式,如求体积功的积分公式和绝热过程的过程方程,都只是准静态过程才适用的方程。其次要明确是怎样的具体过程,如等温、等压、等体、绝热等。\jg
\par 4. 列方程\quad 即根据以上分析列相应的方程求解。功是过程量,可以利用求体积功的积分公式直接计算功的大小。热量也是过程量,可以直接利用定压或定容热容量计算热量的多少,也可以利用热力学第一定律公式由已知热量求功或已知功求热量。要注意$A,Q,\Delta E$各量的正负的物理意义。\jg
\par 5. 画图线\quad 在解题过程中,最好能画出过程图线。这样对理解题目和分析求解都有帮助。





%打印索引—————————————
\newpage
\addcontentsline{toc}{chapter}{附录}
\addcontentsline{toc}{section}{索引}
\color{titlepurplec}
\appendix
\CJKfamily{kai}
\printindex
%———————————————

\end{document}