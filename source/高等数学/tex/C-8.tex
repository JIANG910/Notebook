\thispagestyle{empty}
\chapter{曲线积分和曲面积分}
\section{第一型曲线积分}
\subsection{第一型曲线积分的基本概念}
\tdefination[第一型曲线积分的定义]
设$f(x,y,z)$在分段光滑的曲线$L$上有定义,对$L$任意分割成$n$段,第$i$段的弧长为$\Delta s_i$及在第$i$段任意选择的$(\xi_i,\eta_i,\zeta_i)$,当$\lambda = \max\limits_{1 \le i \le n} {\Delta s_i}\rightarrow 0$时,极限
\begin{equation}
	\lim_{\lambda \rightarrow 0} \sum^{n}_{i=1} f(\xi_i,\eta_i,\zeta_i)\,\Delta s_i
\end{equation}
总存在,则这个极限称为函数$f(x,y,z)$沿曲线$L$的第一型曲线积分或弧长的曲线积分,记做
\begin{equation}
	\int_{L}f(x,y,z)\,\,\d s
\end{equation}
\par 其中,$L$称作积分曲线,而$f(x,y,z)$称作被积函数,$\d s$称为弧积分.

\subsection{第一型曲线积分的基本性质}
\ttheorem[第一型曲线积分的三个基本性质]
1.可拆可和性
\begin{equation}
	\int_{L}[\,C_1f(x,y,z)+C_2g(x,y,z)\,]\,\,\d s =\int_{L}C_1f(x,y,z)\,\,\d s + \int_{L}C_2g(x,y,z)\,\,\d s
\end{equation}

\par 2.分段累加性$(L\rightarrow L_1,L_2,\cdots,L_m)$
\begin{equation}
	\int_{L}f(x,y,z)\,\,\d s = \int_{L_1}f(x,y,z)\,\,\d s +\int_{L_2}f(x,y,z)\,\,\d s + \cdots +\int_{L_m}f(x,y,z)\,\,\d s
\end{equation}

\par 3.恒正性(无向性)
\begin{equation}
	\int_{\wideparen{AB}}f(x,y,z)\,\,\d s = \int_{\wideparen{BA}}f(x,y,z)\,\,\d s 
\end{equation}

\subsection{第一型曲线积分的计算}
\ttheorem[直角坐标下平面曲线的积分]
若曲线$L$由$y=y(x)$确定,且$y=y(x)$在$[a,b]$上有连续导数,$f(x,y)$在$L$上连续,则
\begin{equation}
	\int_{L}f(x,y)\,\,\d s =\int_{a}^{b}f[x,y(x)]\,\sqrt{1+[\,y'(x)\,]^2}\,\,\d x
\end{equation}

\theorem[参数方程下平面曲线的积分]
若曲线$L$由
$
\begin{cases}
	x=x(t),\\
	y=y(t)
\end{cases}
$
确定,且$x(t),y(t)$在$t \in [a,b]$上有连续导数,$f(x,y)$在$L$上连续,则
\begin{equation}
	\int_{L}f(x,y)\,\,\d s =\int_{a}^{b}f[x(t),y(t)]\,\sqrt{[\,x'(t)\,]^2+[\,y'(t)\,]^2}\,\,\d t
\end{equation}

\theorem[参数方程下空间曲线的积分]
若曲线$L$由
$
\begin{cases}
	x=x(t),\\
	y=y(t),\\
	z=z(t).
\end{cases}
$
确定,且$x(t),y(t),z(t)$在$t \in [a,b]$上有连续导数,$f(x,y,z)$在$L$上连续,则
\begin{equation}
	\int_{L}f(x,y,z)\,\,\d s =\int_{a}^{b}f[x(t),y(t),z(t)]\,\sqrt{[\,x'(t)\,]^2+[\,y'(t)\,]^2+[\,z'(t)\,]^2}\,\,\d t
\end{equation}


\section{第二型曲线积分}
\subsection{第二型曲线积分的基本概念}
\tdefination[第一型曲线积分的定义]
$L$是从点$A$到点$B$的分段光滑有向曲线,向量函数$\bm{F}(x,y)=P(x,y)\bm{i}+Q(x,y)\bm{j}$在$L$上有定义,按照$L$的方向,对$L$的任意分割成$n$个有向的小线段$\overrightarrow{A_{i-1}A_i}$,记$\wideparen{A_{i-1}A_i}$的弧长为$\Delta s_i$及在第$i$段任意选择的$(\xi_i,\eta_i,\zeta_i)$,当$\lambda = \max\limits_{1 \le i \le n} {\Delta s_i}\rightarrow 0$时,极限
\begin{equation}
	\lim_{\lambda \rightarrow 0} \sum^{n}_{i=1} \bm{F}(\xi_i,\eta_i,\zeta_i) \cdot \overrightarrow{A_{i-1}A_i} \,\Delta s_i = \lim_{\lambda \rightarrow 0} \sum^{n}_{i=1} [P(\xi_i,\eta_i)\Delta x_i+Q(\xi_i,\eta_i)\Delta y_i] 
\end{equation}
总存在,则这个极限称为向量函数$\bm{F}(x,y)$沿曲线$L$从点$A$到点$B$的第二型曲线积分或对坐标的曲线积分,记做
\begin{equation}
	\int_{\wideparen{AB}}P\,\d x+Q\,\d y \huo \int_{\wideparen{AB}}\bm{F}(x,y) \,\, \d \bm{r}
\end{equation}
\par 其中,$\d \bm{r}=(\d x,\d y)$,有向曲线$\wideparen{AB}$称为积分路径.
\par 类似地,对于空间向量函数$\bm{F}(x,y,z)=P(x,y,z)\bm{i}+Q(x,y,z)\bm{j}+R(x,y,z)\bm{k}$,沿空间有向曲线$L$的第二型曲线积分为
\begin{equation}
	\int_{L}P\,\d x+Q\,\d y+R\,\d z \huo \int_{L}\bm{F}(x,y,z) \,\, \d \bm{r}
\end{equation}

\subsection{第二型曲线积分的基本性质}
\ttheorem[第二型曲线积分的三个基本性质]
1.可拆可和性
\begin{equation}
	\int_{\wideparen{AB}}[k_1\bm{F}(M)+k_2\bm{G}(M)] \cdot \d \bm{r} = k_1\int_{\wideparen{AB}}\bm{F}(M) \cdot \d \bm{r} +k_2 \int_{\wideparen{AB}}\bm{G}(M) \cdot \d \bm{r} 
\end{equation}

2.分段累加性$\left( \wideparen{AB} \rightarrow \wideparen{AC} + \wideparen{CB}\right) $
\begin{equation}
	\int_{\wideparen{AB}}\bm{F}(M) \cdot \d \bm{r} = \int_{\wideparen{AC}}\bm{F}(M) \cdot \d \bm{r} +\int_{\wideparen{CB}}\bm{F}(M) \cdot \d \bm{r}
\end{equation}

3.有向性
\begin{equation}
	\int_{\wideparen{AB}}\bm{F}(M) \cdot \d \bm{r} = -\int_{\wideparen{BA}}\bm{F}(M) \cdot \d \bm{r}
\end{equation}

\subsection{第二型曲线积分的计算}
\ttheorem[平面曲线下第二型曲线积分的计算]
设曲线$L$的参数方程为
$
\begin{cases}
	x = x(t),\\
	y = y(t).
\end{cases}
$
其中$x(t),y(t)$有连续的一阶导数.当$t$单调地从$a$变化到$b$时,且$P(x,y),Q(x,y)$在$L$上连续,则
\begin{equation}
	\int_{\wideparen{AB}}P(x,y)\,\d x+Q(x,y)\,\d y = \int_{a}^{b}[\,P(x(t),y(t))\,x'(t) + Q(x(t),y(t))\,y'(t) \,]\,\d t
\end{equation}
特别地,当$y=g(x)$时,可以变为
\begin{equation}
	\int_{\wideparen{AB}}P(x,y)\,\d x+Q(x,y)\,\d y = \int_{a}^{b}[\,P(x,g(x)) + Q(x,g(x))\,g'(x) \,]\,\d x
\end{equation}

\ttheorem[空间曲线下第二型曲线积分的计算]
设曲线$L$的参数方程为
$
\begin{cases}
	x = x(t),\\
	y = y(t),\\
	z =z(t).
\end{cases}
$
其中$x(t),y(t),z(t)$有连续的一阶导数,且$P(x,y,z),Q(x,y,z),R(x,y,z)$在$L$上连续,则
\begin{equation}
	\begin{split}
		&\quad \,\int_{\wideparen{AB}}P(x,y,z)\,\d x+Q(x,y,z)\,\d y +R(x,y,z)\, \d z\\
		&= \int_{a}^{b}[\,P(x(t),y(t),z(t))\,x'(t) + Q(x(t),y(t),z(t))\,y'(t) + R(x(t),y(t),z(t))\,z'(t) \,]\,\d t
	\end{split}
\end{equation}

\theorem[格林公式]
对于闭区域的边界$L$规定其正方向$L^+$为使得沿这个方向前进时区域总在左侧,那么有
\begin{equation}
	\oint_{L^+}P\,\d x+Q\,\d y = \iint\limits_{D}\left( \frac{\partial Q}{\partial x} -\frac{\partial P}{\partial y}\right) \,\, \d x \d y
\end{equation}
\par 特别地,当$\displaystyle\frac{\partial Q}{\partial x} =\frac{\partial P}{\partial y}$或$\displaystyle \oint_{L^+}P\,\d x+Q\,\d y =0$时,第二型曲面积分与积分路径无关。\\[0.5em]
此时可以找到一个函数$u(x,y)=P(x,y) \, \d x+Q(x,y)\,\d y$,即
\begin{equation}
	\int_{\wideparen{AB}}P\,\d x+Q\,\d y  =\int_{a}^{b} \d u = u(B)-u(A)
\end{equation}
\newpage

\subsection{第二型曲面积分与路径无关的判定}
\tinference[第二型曲面积分与路径无关的判定]
1. 用于判定路径有关的方法(也适用于判断$P\,\d x+Q\,\d y$在$D$上不存在原函数)
\par \quad \quad (1)\quad 存在一条分段光滑曲线$C\subset D,\displaystyle \oint_{C}P\,\d x+Q\,\d y\ne 0$.
\jg
\par \quad \quad (2)\quad 存在$(x,y)\in D,\displaystyle \frac{\partial Q(x,y)}{\partial x}\ne \frac{\partial P(x,y)}{\partial y}$.
\jg
\par 2. 用于判定路径无关的方法(方法(2),(3)可以用于判断$P\,\d x+Q\,\d y$在$D$上存在原函数)
\par \quad \quad (1)\quad 求得$u(x,y)$使得$\d u=P(x,y)\,\d x+Q(x,y)\,\d y$任意$(x,y)\in D$.
\jg
\par \quad \quad (2)\quad 若$D$是单连通的,又对于任意的$(x,y)\in D$都有$\displaystyle\frac{\partial Q}{\partial x} =\frac{\partial P}{\partial y}$.
\jg
\par \quad \quad (3)\quad 若$D=D_0\setminus\left\lbrace M_0\right\rbrace,\,D_0 $是单连通的,$M_0\in D_0$.若对于任意的$(x,y)\in D$都有$\displaystyle\frac{\partial Q}{\partial x} =\frac{\partial P}{\partial y}$,且存在一条包围点$M_0$的分段光滑闭曲线$C_0$,使得$\displaystyle \oint_{C_0}P\,\d x+Q\,\d y= 0$.


\section{第一型曲面积分}
\subsection{第一型曲面的基本概念}
\tdefination[第一型曲面积分的定义]
设$f(x,y,z)$在分片光滑的曲面$S$上有定义,对$S$任意分割成互补重叠的$n$片,第$i$段的面积为$\Delta S_i$及在第$i$段任意选择的$(\xi_i,\eta_i,\zeta_i)$,当$\lambda = \max\limits_{1 \le i \le n} \left\lbrace \Delta S_i\mbox{的直径}\right\rbrace \rightarrow 0$时,极限
\begin{equation}
	\lim_{\lambda \rightarrow 0} \sum^{n}_{i=1} f(\xi_i,\eta_i,\zeta_i)\,\Delta S_i
\end{equation}
总存在,则这个极限称为函数$f(x,y,z)$在曲面$S$上的第一型曲面积分,记做
\begin{equation}
	\iint_{S}f(x,y,z)\,\,\d S
\end{equation}
\par 其中,$S$称作积分曲面,而$f(x,y,z)$称作被积函数.特别地,如果积分曲面封闭,则记做
\begin{equation}
	\oint_{S}f(x,y,z)\,\,\d S
\end{equation}

\subsection{第一型曲面积分的基本性质}
\ttheorem[第一型曲面积分的三个基本性质]
1.可拆可和性
\begin{equation}
	\iint\limits_{S}[\,C_1f(x,y,z)+C_2g(x,y,z)\,]\,\,\d S =\iint\limits_{S}C_1f(x,y,z)\,\,\d S + \iint\limits_{S}C_2g(x,y,z)\,\,\d S
\end{equation}

\par 2.分片累加性$(S\rightarrow S_1,S_2,\cdots,S_i)$
\begin{equation}
	\iint\limits_{S}f(x,y,z)\,\,\d S = \sum_{i=1}^{m}\iint\limits_{S_i}f(x,y,z)\,\,\d S
\end{equation}

\par 3.恒正性(无向性)
\begin{equation}
	\iint\limits_{S}f(x,y,z)\,\,\d S \geq 0
\end{equation}

\subsection{第一型曲面积分的计算}
\ttheorem[二元函数下第一型曲面积分的计算]
对于二元函数$z=z(x,y),y=y(x,z),x=x(y,z)$,
\renewcommand\arraystretch{1.5}
\begin{equation}
	\iint\limits_{\Sigma}f(x,y,z)\,\,\d S 
	=
	\left\lbrace 
	\begin{array}{c}
		\displaystyle \iint\limits_{D_{xy}}f(x,y,z(x,y))\,\sqrt{1+z_x^2+z_y^2}\,\,\d x\d y\quad \Sigma :z=z(x,y) \\
		\displaystyle  \iint\limits_{D_{xz}}f(x,y(x,z),z)\,\sqrt{1+y_x^2+y_z^2}\,\,\d x\d z\quad \Sigma :y=y(x,z) \\
		\displaystyle  \iint\limits_{D_{yz}}f(x(y,z),y,z)\,\sqrt{1+x_y^2+x_z^2}\,\,\d y\d z\quad \Sigma :x=x(y,z)
	\end{array}
	\right.
\end{equation}
\renewcommand\arraystretch{1}
提示:根据曲面方程的特点选择恰当的积分形式,$D$代表投影到某个坐标平面的平面区域.

\theorem[参数方程下第一型曲面积分的计算]
若$S$由参数方程
$
\begin{cases}
	x = x(u,v),\\
	y = y(u,v),\\
	z= z(u,v).
\end{cases}
$
确定,记
$
\begin{cases}
	E = x_u^2 + y_u^2 +z_u^2,\\
	F = x_ux_v + y_uy_v + z_uz_v,\\
	G = x_v^2 + y_v^2 +z_v^2.
\end{cases}
$
则
\begin{equation}
	\iint\limits_{\Sigma}f(x,y,z)\,\,\d S =\iint\limits_{\Sigma}f(x(u,v),y(u,v),z(u,v))\,\sqrt{EG-F^2}\,\,\d u \d v
\end{equation}
\par 特别地,当参数方程是柱坐标变换方程时,
\begin{equation}
	\d S = R \,\d \theta \d z
\end{equation}
当参数方程是球坐标变换方程时,
\begin{equation}
	\d S = R^2\,|\sin \varphi| \,\,\d \theta \d \varphi
\end{equation}

\section{第二型曲面积分}
\subsection{第二型曲面积分的基本概念}
\defination[第二型曲面积分]
设$S$是一个分片光滑的双侧曲面,在曲面$S$上选定了一侧,记选定一侧的单位法向量为$\bm{n}(P)$.假设在$S$上给定了一个向量函数$\bm{F}(x,y,z)$.我们将$S$分割成$n$个不相重叠的小曲面片$\Delta S_i(i=1,2,\cdots,n)$,其面积也用$\Delta S_i$表示.在$\Delta S_i$上任意取一点$M_i(\xi_i,\eta_i,\zeta_i)$,如果$\lambda = \max\limits_{1 \le i \le n} \left\lbrace \Delta S_i\mbox{的直径}\right\rbrace \rightarrow 0$时,极限
\begin{equation}
	\lim_{\lambda \rightarrow 0} \sum^{n}_{i=1} \bm{F}(\xi_i,\eta_i,\zeta_i)\cdot \bm{n}(\xi_i,\eta_i,\zeta_i)\,\,\Delta S_i
\end{equation}
总存在,则这个极限称为向量函数$\bm{F}(x,y,z)$在曲面$S$上的第二型曲面积分,记做
\begin{equation}
	\iint\limits_{S} \bm{F}(x,y,z)\cdot \bm{n}(x,y,z)\,\,\Delta S
\end{equation}

\tdefination[曲面的方向]
对于不同曲面的方程形式,曲面方向判定见下表\ref{曲面方向的判定}.
\begin{table}[!htb]
	\centering
	\setlength{\tabcolsep}{10mm}{
		\begin{tabular}{ccc}
			\toprule[2pt] 
			曲面方程形式 & 法向量  & 方向的规定\\  
			\midrule[1.2pt]
			\multirow{2}{*}{$z=z(x,y)$} & $\bm{n}_1 = (-z_x,-z_y,1) $ &上侧\\
			\cline{2-3}
			& $\bm{n}_2 = (z_x,z_y,-1) \hspace{0.8em}$ & 下侧\\
			\hline
			\multirow{2}{*}{$y=y(x,z)$} & $\bm{n}_1 = (-y_x,1,-y_z)$ &右侧\\
			\cline{2-3}
			& $\bm{n}_2= (y_x,-1,y_z)\hspace{0.8em} $ & 左侧\\
			\hline
			\multirow{2}{*}{$x=x(y,z)$} & $\bm{n}_1 = (1,-x_y,-x_z) $ &前侧\\
			\hline
			& $\bm{n}_2= (-1,x_y,x_z)\hspace{0.8em} $ & 后侧\\
			\bottomrule[2pt]
		\end{tabular}  
	}
	\caption{曲面方向的判定}
	\label{曲面方向的判定}
\end{table} 
\par 对于曲面而言,法向量指向曲面的内部为曲面内侧;法向量指向曲面的外部为曲面外侧.



\subsection{第二型曲面积分的基本性质}
\ttheorem[第二型曲面积分的三个基本性质]
1.可拆可和性
\begin{equation}
	\iint\limits_{S}[\,C_1\bm{F}_1+C_2\bm{F}_2\,]\,\,\d \bm{S} =\iint\limits_{S}C_1\bm{F}_1\,\,\d \bm{S}  + \iint\limits_{S}C_2\bm{F}_2\,\,\d \bm{S} 
\end{equation}

\par 2.分片累加性$(S\rightarrow S_1+S_2)$
\begin{equation}
	\iint\limits_{S}\bm{F}\,\,\d \bm{S}  = \iint\limits_{S_1}\bm{F}\,\,\d \bm{S}  +\iint\limits_{S_2}\bm{F}\,\,\d \bm{S} 
\end{equation}

\par 3.有向性
\begin{equation}
	\iint\limits_{S^+}\bm{F}\,\,\d \bm{S} = -\iint\limits_{S^-}\bm{F}\,\,\d \bm{S} 
\end{equation}

\subsection{第二型曲面积分与第一型曲面积分的关系}
\ttheorem[第二型曲面积分与第一型曲面积分的关系]
设向量函数$\bm{F}(P(x,y,z),Q(x,y,z),R(x,y,z))$的单位法向量为$\bm{n}=(x,y,z)$,其方向余弦为
\[
\cos \alpha (x,y,z),\cos \beta (x,y,z),\cos \gamma(x,y,z)
\]
则二重积分可写成
\begin{equation}
	\begin{split}
		\iint\limits_{S}\bm{F}\cdot \bm{n}\,\,\d S&=\iint\limits_{S}\big(P\cos \alpha +Q\cos \beta +R\cos \gamma \big)\,\,\d S\\
		&=\iint\limits_{S}P \,\d y\d z+Q\,\d z\d x+R\,\d x\d y
	\end{split}
\end{equation}

\subsection{第二型曲面积分的计算}
\begin{table}[h]
	\centering
	\setlength{\tabcolsep}{20mm}{
		\begin{tabular}{cc}
			\toprule[1.5pt] 
			曲面方程形式 &  方向余弦 \\  
			\midrule
			& \vspace*{-1.7em} \\
			$z=z(x,y)$& $\displaystyle \frac{\pm 1}{\sqrt{1+z_x^2+z_y^2}}\, (-z_x,-z_y,1)$\\
			& \vspace*{-1.5em} \\
			\hline
			& \vspace*{-1.5em} \\
			$y=y(x,z)$ & $\displaystyle \frac{\pm 1}{\sqrt{1+y_x^2+y_z^2}} \,(-y_x,1,-y_z)$ \\
			& \vspace*{-1.5em} \\
			\hline
			& \vspace*{-1.5em} \\
			$x=x(y,z)$& $\displaystyle \frac{\pm 1}{\sqrt{1+x_y^2+x_z^2}} \,(1,-x_y,-x_z)$ \\
			& \vspace*{-1.7em} \\
			\bottomrule[1.5pt]
		\end{tabular}  
	}
	\caption{不同的曲面方程的方向余弦}
	\renewcommand{\arraystretch}{1}
	\label{方向余弦}
\end{table} 

\ttheorem[直接转换为二重积分计算]
由上表\ref{方向余弦}并利用公式$\displaystyle \iint\limits_{S}\bm{F}\cdot \bm{n}\,\,\d S=\iint\limits_{S}(P\cos \alpha +Q\cos \beta +R\cos \gamma )\,\,\d S$可以得到转换公式如下表.
\begin{table}[h]
	\centering
	\setlength{\tabcolsep}{3.2mm}{
		\begin{tabular}{cc}
			\toprule[1.5pt] 
			曲面方程形式 & 结果\\  
			\midrule
			& \vspace*{-1.7em} \\
			$z=z(x,y)$ &$\displaystyle \pm\iint\limits_{D_{xy}}[\,P(x,y,z(x,y))(-z_x) +Q(x,y,z(x,y))(-z_y) +R(x,y,z(x,y))\,]\,\,\d \sigma $\\
			& \vspace*{-1.5em} \\
			\hline
			& \vspace*{-1.5em} \\
			$y=y(x,z)$ &$\displaystyle \pm\iint\limits_{D_{xz}}[\,P(x,y,z(x,y))(-y_x) +Q(x,y,z(x,y)) +R(x,y,z(x,y))(-y_z)\,]\,\,\d \sigma $\\
			& \vspace*{-1.5em} \\
			\hline
			& \vspace*{-1.5em} \\
			$x=x(y,z)$ &$\displaystyle \pm\iint\limits_{D_{xy}}[\,P(x,y,z(x,y)) +Q(x,y,z(x,y))(-x_y) +R(x,y,z(x,y))(-x_z)\,]\,\,\d \sigma $\\
			& \vspace*{-1.7em} \\
			\bottomrule[1.5pt]
		\end{tabular}  
	}
	\caption{转换为二重积分计算的计算公式}
	\renewcommand{\arraystretch}{1}
	\label{第二型曲面积分的直接计算}
\end{table} 
\par 注:上表\ref{第二型曲面积分的直接计算}中正负号的选取与方向余弦的$``1"$的符号相同.$D$表示投影到相应坐标平面的平面区域.

\inference[转换为二重积分计算第二型曲面积分]
\noindent \quad 总结\quad ``一投、二代、三定号"
\par \quad 1. 将曲面$\Sigma$的方程写成上述三种形式的其中一种.
\par \quad 2. 将曲面$\Sigma$投影到相应的坐标平面,得到投影区域$D$.
\par \quad 3. 将$x=x(y,z)$或$y=y(x,z)$或$z=z(x,y)$代入被积函数,将$\Sigma $换成$D$.
\par \quad 4. 根据方向余弦确定侧向进而确定二重积分的符号.

\quad 提示:若投影区域面积为0,则相应的二重积分为0.

\theorem[高斯公式]
表达了空间闭区域上的三重积分与其边界曲面上的曲面积分之间的关系, 这个关系可陈述如下:
\par 设空间闭区域$\Omega $是由分片光滑的闭曲面$\Sigma $所围成,若函数$P(x, y, z),Q(x, y, z),R(x, y, z)$在$\Omega $上具有一阶连续偏导数.则有
\begin{equation}
	\oiint\limits_{S^+}P \,\d y\d z+Q\,\d z\d x+R\,\d x\d y=\iiint\limits_{\Omega}\left( \frac{\partial P}{\partial x}+\frac{\partial Q}{\partial y}+\frac{\partial R}{\partial z}\right)\,\d V 
\end{equation}
\par 其中$S^+$是曲面$S$的外侧.\\
注:若$S$不是封闭曲面,可利用补片法,常用平行于坐标面的平面来补片.
\par 若$S$不是外侧,则在积分前面加负号,所求的结果和外侧的结果互为相反数.

\example[曲面积分求体积]
若$\Sigma$封闭且方向取外侧,则
\begin{equation}
	\begin{split}
		\oiint\limits_{\Sigma^+}x \,\d y\d z+y\,\d z\d x+z\,\d x\d y=\iiint\limits_{\Omega}\left(1+1+1\right)\,\d V =3\iiint\limits_{\Omega} \,\d V =3V
	\end{split}
\end{equation}

\section{斯托克斯公式}
\ttheorem[斯托克斯公式]
设$\Gamma$为分段光滑的空间有向闭曲线,$\Sigma$是以$\Gamma$为边界的分片光滑的有向曲面,$\Gamma$的正向与$\Sigma$的侧符合右手法则, 若函数$P(x, y, z),Q(x, y, z),R(x, y, z)$在曲面$\Sigma$(连同边界$\Gamma$)上具有一阶连续偏导数,则有
\begin{equation}
	\oint_{L^+}P\,\d x+Q\,\d y+R\,\d z=\iint\limits_{S^+}\left( \frac{\partial R}{\partial y}-\frac{\partial Q}{\partial z}\right)\,\d y\d z+\left( \frac{\partial P}{\partial z}-\frac{\partial R}{\partial x} \right) \,\d z \d x +\left( \frac{\partial Q}{\partial x}-\frac{\partial P}{\partial y}\right) \, \d x\d y. 
\end{equation}
\par 斯托克斯公式是格林公式的推广.格林公式表达了平面闭区域上的二重积分与其边界曲线上的曲线积分间的关系;而斯托克斯公式则把曲面$\Sigma$上的曲面积分与沿着$\Sigma$的边界曲线$\Gamma$的曲线积分联系起来.
\par 为了便于记忆,也可写成行列式的形式
\begin{equation}
	\renewcommand{\arraystretch}{1.5}
	\iint\limits_{S^+}
	\left| 
	\begin{array}{ccc}
		\d y \d z &\d z \d x &\d x \d y \\
		\displaystyle \frac{\partial }{\partial x} &\displaystyle \frac{\partial }{\partial y} & \displaystyle \frac{\partial }{\partial z}\\
		P & Q & R
	\end{array}
	\right| 
	\huo
	\iint\limits_{S^+}
	\left| 
	\begin{array}{ccc}
		\cos \alpha & \cos \beta &\cos \gamma\\
		\displaystyle \frac{\partial }{\partial x} &\displaystyle \frac{\partial }{\partial y} & \displaystyle \frac{\partial }{\partial z}\\
		P & Q & R
	\end{array}
	\right| 
	\d S
	\renewcommand{\arraystretch}{1}
\end{equation}

\section{积分的特点}
\subsection{积分区域的可代入性}
当\ds[积分区域是确定方程(等式)],而不是非确定方程(含有不等号)的时候可以将积分区域的函数代入被积函数,或者被积函数构造成积分区域的方程.通常\ds[曲线积分和曲面积分都可以直接代入积分区域],因为这些积分的积分区域通常都是由确定的方程来决定的。

\subsection{多元函数的奇偶性}
设三元函数$f(x,y,z)$,则定义函数的奇偶性如下表\ref{多元函数的奇偶性}.
\begin{table}[h]
	\centering
	\renewcommand{\arraystretch}{1}
	\setlength{\tabcolsep}{6mm}{
		\begin{tabular}{ccc}
			\toprule[1.5pt] 
			满足等式  & 函数奇偶性 & 图像特点 \\  
			\midrule
			$f(x,y,z)=-f(-x,y,z)$& $f(x,y,z)$是关于$x$的奇函数&无\\
			\hline
			$f(x,y,z)=f(-x,y,z)$ & $f(x,y,z)$是关于$x$的偶函数&$f(x,y,z)$的图形关于$Ozy$平面对称\\
			\hline
			$f(x,y,z)=-f(x,-y,z)$& $f(x,y,z)$是关于$y$的奇函数&无\\
			\hline
			$f(x,y,z)=f(x,-y,z)$ & $f(x,y,z)$是关于$y$的偶函数&$f(x,y,z)$的图形关于$Ozx$平面对称\\
			\hline
			$f(x,y,z)=-f(x,y,-z)$& $f(x,y,z)$是关于$z$的奇函数&无\\
			\hline
			$f(x,y,z)=f(x,y,-z)$ & $f(x,y,z)$是关于$z$的偶函数&$f(x,y,z)$的图形关于$Oxy$平面对称\\
			\bottomrule[1.5pt]
		\end{tabular}  
	}
	\caption{多元函数的奇偶性}
	\renewcommand{\arraystretch}{1}
	\label{多元函数的奇偶性}
\end{table} 

\section{积分的轮换对称性}
二元函数的轮换对称性
\begin{equation}
	f(x,y)=f(y,x)
\end{equation}
其几何意义是$f(x,y)$的图形关于$y=x$对称.
\par 三元函数的轮换对称性
\begin{equation}
	f(x,y,z)=f(y,x,z)=f(x,z,y)
\end{equation}
\begin{table}[!htb]
	\centering
	\setlength{\tabcolsep}{9mm}{
		\begin{tabular}{cc}
			\toprule[2pt] 
			积分类型  & 轮换表达式 \\  
			\midrule[1.3pt]
			& \vspace*{-1.3em} \\
			二重积分 &  $\displaystyle \iint\limits_{D}f(x,y)\,\,\d \sigma =\iint\limits_{D}f(y,x)\,\,\d \sigma = \frac{1}{2}\iint\limits_{D}\left[\, f(y,x)+f(x,y) \, \right]\,\,\d \sigma$\\
			& \vspace*{-1.5em} \\
			\hline
			& \vspace*{-1em} \\
			三重积分 &$\displaystyle \iiint\limits_{S}f(x,y,z)\,\,\d V =\iiint\limits_{S}f(y,x,z)\,\,\d V = \iiint\limits_{S}f(z,x,y)\,\,\d V $ \\
			& \vspace*{-1.5em} \\
			\hline
			& \vspace*{-1.3em} \\
			第一型曲线积分 & $\displaystyle \int_{L}f(x,y)\,\,\d \sigma =\int_{L}f(y,x)\,\,\d \sigma = \frac{1}{2}\int_{L}\left[\, f(y,x)+f(x,y) \, \right]\,\,\d \sigma$ \\
			& \vspace*{-1.3em} \\
			\hline
			& \vspace*{-1.3em} \\
			第二型曲线积分 &  $\displaystyle \int_{L}f(x,y)\,\,\d x+\int_{L}f(y,x)\,\,\d y= 0$ \\
			& \vspace*{-1.3em} \\
			\hline
			& \vspace*{-1em} \\
			第一型曲面积分 &  $\displaystyle \iint\limits_{S}f(x,y,z)\,\,\d S =\iint\limits_{S}f(y,x,z)\,\,\d S = \iint\limits_{S}f(z,x,y)\,\,\d S $ \\
			& \vspace*{-1.5em} \\
			\hline
			& \vspace*{-1.1em} \\
			第二型曲面积分 &  $\displaystyle \iint\limits_{S}f(x,y,z)\,\,\d y \d z =\iint\limits_{S}f(y,x,z)\,\,\d x\d z = \iint\limits_{S}f(z,x,y)\,\,\d x\d y $ \\
			& \vspace*{-1.7em} \\
			\bottomrule[2pt]
		\end{tabular}  
	}
	\caption{积分的轮换对称性}
	\renewcommand{\arraystretch}{1}
	\label{积分的轮换对称性}
\end{table} 

\section{积分的奇偶对称性}
设被积函数为$f(x,y)$或$f(x,y,z)$,根据函数的奇偶性,可以得到积分的对称性
\begin{table}[!htb]
	\centering
	\setlength{\tabcolsep}{9mm}{
		\begin{tabular}{cccc}
			\toprule[2pt] 
			积分类型  & 积分区域  &  被积函数 & 化简结果 \\  
			\midrule[1.3pt]
			\multirow{4}{*}{\makecell[c]{\\[1.2em]二重积分}} & \multirow{2}{*}{关于$y$轴对称}  & 关于$x$的偶函数 & \makecell[c]{\\[-1.5em]$\displaystyle I=2\iint\limits_{D_1} f(x,y)\,\, \d \sigma $\\[1.2em]}\\
			\cline{3-4}
			&  & 关于$x$的奇函数 & \makecell[c]{\\[-1.5em]$\displaystyle I=0$\\[0.3em]}\\
			\cline{2-4}
			& \multirow{2}{*}{关于$x$轴对称}  & 关于$y$的偶函数 & \makecell[c]{\\[-1.5em]$\displaystyle I=2\iint\limits_{D_1} f(x,y) \,\,\d \sigma $\\[1em]}\\
			\cline{3-4}
			&  & 关于$y$的奇函数 & \makecell[c]{\\[-1.5em]$\displaystyle I=0$\\[0.3em]}\\
			\midrule[1.3pt]
			\multirow{6}{*}{\makecell[c]{\\[2em]三重积分}} & \multirow{2}{*}{关于$Oxy$对称}  & 关于$z$的偶函数 & \makecell[c]{\\[-1.5em]$\displaystyle I=2\iiint\limits_{\Omega _1} f(x,y) \,\,\d V$\\[1em]}\\
			\cline{3-4}
			&  & 关于$z$的奇函数 & \makecell[c]{\\[-1.5em]$\displaystyle I=0$\\[0.3em]}\\
			\cline{2-4}
			& \multirow{2}{*}{关于$Oxz$对称}  & 关于$y$的偶函数 & \makecell[c]{\\[-1.5em]$\displaystyle I=2\iiint\limits_{\Omega _1} f(x,y) \,\,\d V$\\[1em]}\\
			\cline{3-4}
			&  & 关于$y$的奇函数 & \makecell[c]{\\[-1.5em]$\displaystyle I=0$\\[0.3em]}\\
			\cline{2-4}
			& \multirow{2}{*}{关于$Oyz$对称}  & 关于$x$的偶函数 & \makecell[c]{\\[-1.5em]$\displaystyle I=2\iiint\limits_{\Omega _1} f(x,y) \,\,\d V$\\[0.3em]}\\
			\cline{3-4}
			&  & 关于$x$的奇函数 & \makecell[c]{\\[-1.5em]$\displaystyle I=0$\\[0.3em]}\\
			\midrule[1.3pt]
			\multirow{4}{*}{\makecell[c]{\\[0.7em]第一型\\曲线积分\\(平面)}} & \multirow{2}{*}{关于$y$轴对称}  & 关于$x$的偶函数 & \makecell[c]{\\[-1.5em]$\displaystyle I=2\int_{L_1} f(x,y) \,\,\d s $\\[1em]}\\
			\cline{3-4}
			&  & 关于$x$的奇函数 & \makecell[c]{\\[-1.5em]$\displaystyle I=0$\\[0.3em]}\\
			\cline{2-4}
			& \multirow{2}{*}{关于$x$轴对称}  & 关于$y$的偶函数 & \makecell[c]{\\[-1.5em]$\displaystyle I=2\int_{L_1} f(x,y)\,\,\d s  $\\[1em]}\\
			\cline{3-4}
			&  & 关于$y$的奇函数 & \makecell[c]{\\[-1.5em]$\displaystyle I=0$\\[0.3em]}\\
			\midrule[1.3pt]
			\multirow{6}{*}{\makecell[c]{\\[1.5em]第一型\\曲线积分\\(空间)}} & \multirow{2}{*}{关于$Oxy$对称}  & 关于$z$的偶函数 & \makecell[c]{\\[-1.5em]$\displaystyle I=2\int_{L_1} f(x,y,z)\,\,\d s  $\\[1em]}\\
			\cline{3-4}
			&  & 关于$z$的奇函数 & \makecell[c]{\\[-1.5em]$\displaystyle I=0$\\[0.3em]}\\
			\cline{2-4}
			& \multirow{2}{*}{关于$Oxz$对称}  & 关于$y$的偶函数 & \makecell[c]{\\[-1.5em]$\displaystyle I=2\int_{L_1} f(x,y,z)\,\,\d s  $\\[1em]}\\
			\cline{3-4}
			&  & 关于$y$的奇函数 & \makecell[c]{\\[-1.5em]$\displaystyle I=0$\\[0.3em]}\\
			\cline{2-4}
			& \multirow{2}{*}{关于$Oyz$对称}  & 关于$x$的偶函数 & \makecell[c]{\\[-1.5em]$\displaystyle I=2\int_{L_1} f(x,y,z)\,\,\d s  $\\[1em]}\\
			\cline{3-4}
			&  & 关于$x$的奇函数 & \makecell[c]{\\[-1.5em]$\displaystyle I=0$\\[0.3em]}\\
			\bottomrule[2pt]
		\end{tabular}  
	}
	\caption{积分的奇偶对称性\uppercase\expandafter{\romannumeral1}}
	\renewcommand{\arraystretch}{1}
	\label{积分的奇偶对称性1}
\end{table} 
\clearpage 
续表
\begin{table}[!htb]
	\centering
	\setlength{\tabcolsep}{9mm}{
		\begin{tabular}{cccc}
			\toprule[2pt] 
			 积分类型  & 积分区域  &  被积函数 & 化简结果 \\  
			\midrule[1.3pt]
			\multirow{6}{*}{\makecell[c]{\\[0.5em]第一型\\曲面积分}} & \multirow{2}{*}{关于$Oxy$对称}  & 关于$z$的偶函数 & \makecell[c]{\\[-1.5em]$\displaystyle I=2\iint\limits_{S_1} f(x,y,z) \,\,\d S$\\[1em]}\\
			\cline{3-4}
			&  & 关于$z$的奇函数 & \makecell[c]{\\[-1.5em]$\displaystyle I=0$\\[0.3em]}\\
			\cline{2-4}
			& \multirow{2}{*}{关于$Oxz$对称}  & 关于$y$的偶函数 & \makecell[c]{\\[-1.5em]$\displaystyle I=2\iint\limits_{S_1} f(x,y,z) \,\,\d S$\\[1em]}\\
			\cline{3-4}
			&  & 关于$y$的奇函数 & \makecell[c]{\\[-1.5em]$\displaystyle I=0$\\[0.3em]}\\
			\cline{2-4}
			& \multirow{2}{*}{关于$Oyz$对称}  & 关于$x$的偶函数 & \makecell[c]{\\[-1.5em]$\displaystyle I=2\iint\limits_{S_1} f(x,y,z) \,\,\d S$\\[1em]}\\
			\cline{3-4}
			&  & 关于$x$的奇函数 & \makecell[c]{\\[-1.5em]$\displaystyle I=0$\\[0.3em]}\\
			\midrule[1.3pt]
			\multirow{6}{*}{\makecell[c]{\\[1.5em]第二型\\曲线积分}} & \multirow{2}{*}{关于$Oxy$对称}  & 关于$z$的偶函数 & \makecell[c]{\\[-1.5em]$\displaystyle I=0$\\[0.3em]}\\
			\cline{3-4}
			&  & 关于$z$的奇函数 & \makecell[c]{\\[-1.5em]$\displaystyle I=2\int_{L_1} f(x,y,z)\,\,\d z  $\\[1em]}\\
			\cline{2-4}
			& \multirow{2}{*}{关于$Oxz$对称}  & 关于$y$的偶函数 & \makecell[c]{\\[-1.5em]$\displaystyle I=0$\\[0.3em]}\\
			\cline{3-4}
			&  & 关于$y$的奇函数 & \makecell[c]{\\[-1.5em]$\displaystyle I=2\int_{L_1} f(x,y,z)\,\,\d y  $\\[1em]}\\
			\cline{2-4}
			& \multirow{2}{*}{关于$Oyz$对称}  & 关于$x$的偶函数 & \makecell[c]{\\[-1.5em]$\displaystyle I=0$\\[0.3em]}\\
			\cline{3-4}
			&  & 关于$x$的奇函数 & \makecell[c]{\\[-1.5em]$\displaystyle I=2\int_{L_1} f(x,y,z)\,\,\d y  $\\[1em]}\\
			\midrule[1.3pt]
			\multirow{6}{*}{\makecell[c]{\\[2em]第二型\\曲面积分}} & \multirow{2}{*}{关于$Oxy$对称}  & 关于$z$的偶函数 &  \makecell[c]{\\[-1.5em]$\displaystyle I=0$\\[0.3em]}\\
			\cline{3-4}
			&  & 关于$z$的奇函数 &\makecell[c]{\\[-1.5em]$\displaystyle I=2\iint\limits_{S_1} f(x,y,z)\,\,\d x\d y  $\\[1em]}\\
			\cline{2-4}
			& \multirow{2}{*}{关于$Oxz$对称}  & 关于$y$的偶函数 &  \makecell[c]{\\[-1.5em]$\displaystyle I=0$\\[0.3em]}\\
			\cline{3-4}
			&  & 关于$y$的奇函数 & \makecell[c]{\\[-1.5em]$\displaystyle I=2\iint\limits_{S_1} f(x,y,z)\,\,\d z\d x  $\\[1em]}\\
			\cline{2-4}
			& \multirow{2}{*}{关于$Oyz$对称}  & 关于$x$的偶函数 &  \makecell[c]{\\[-1.5em]$\displaystyle I=0$\\[0.3em]}\\
			\cline{3-4}
			&  & 关于$x$的奇函数 &  \makecell[c]{\\[-1.5em]$\displaystyle I=2\iint\limits_{S_1} f(x,y,z)\,\,\d y\d z  $\\[1em]}\\
			\bottomrule[2pt]
		\end{tabular}  
	}
	\caption{积分的奇偶对称性\uppercase\expandafter{\romannumeral2}}
	\renewcommand{\arraystretch}{1}
	\label{积分的奇偶对称性2}
\end{table} 
\newpage




