\chapter{向量与空间解析几何}
\section{向量及其基本运算}
\subsection{向量的概念}
\defination[向量的定义]
既有大小也有方向的量称为\highlight{red}{\index{XL@向量}向量}(或\highlight{red}{\index{SL@矢量}矢量})。通常我们用有向线段来表示向量。
\par 以$A$为\highlight{red}{\index{QSD@起始点}起始点}、$B$为\highlight{red}{\index{ZD@终点}终点}的有向线段所表示的向量记为$\boldsymbol{AB}$.有时也会用一个黑体字母来表示向量。例如$\boldsymbol{a}$,$\boldsymbol{b}$.向量$\boldsymbol{AB}$的长度表示\highlight{red}{\index{XLDDX@向量的大小}向量的大小},记作\highlight{red}{\index{XLDM@向量的模}向量的模}。
\par 向量的四要素:起始点、终点、方向、模。

\defination[自由向量的定义]
在实际研究中,我们只关心向量的方向和大小,称为\highlight{red}{\index{ZYXL@自由向量}自由向量}。\\

\sj
\defination[向量的夹角定义]
设有两个非零向量$\boldsymbol{a}$,$\boldsymbol{b}$,任取空间中一点$O$,作$\boldsymbol{OA}=\boldsymbol{a}$,$\boldsymbol{OB}=\boldsymbol{b}$,规定$0\leq \varphi=\angle AOB\leq \pi$,那么$\angle AOB$称为\highlight{red}{\index{XLDJJ@向量的夹角}向量$\boldsymbol{a}$,$\boldsymbol{b}$的夹角},记作$\langle\boldsymbol{a}$,$\boldsymbol{b}\rangle$或$\langle\boldsymbol{b}$,$\boldsymbol{a}\rangle$,即$\langle\boldsymbol{a}$,$\boldsymbol{b}\rangle=\varphi$.
\par 特别地,如果$\langle\boldsymbol{a}$,$\boldsymbol{b}\rangle=0$或$\pi$,就称向量$\langle\boldsymbol{a},\boldsymbol{b}
\rangle$平行记作$\boldsymbol{a}\parallel\boldsymbol{b}$,当$\boldsymbol{a}$,$\boldsymbol{b}$平移至起点相同的时候,那么$\langle\boldsymbol{a}$,$\boldsymbol{b}\rangle$一定共线。所以,两个平行的向量也称为\highlight{red}{\index{GXXL@共线向量}共线向量}。
\par 如果$\langle\boldsymbol{a}$,$\boldsymbol{b}\rangle$=$\di\frac{\pi}{2}$,就称向量$\boldsymbol{a}$,$\boldsymbol{b}$垂直,记作$\langle\boldsymbol{a}\perp\boldsymbol{b}\rangle$.
\warn[\par 零向量与另外的向量的夹角可以任意在{[0,$\pi$]}中取值。因此,可以认为零向量与任何向量都平行(垂直)]
\defination[向量相等的定义]
如果两个向量$\boldsymbol{a}$,$\boldsymbol{b}$大小和方向都相同,那么向量$\boldsymbol{a}$,$\boldsymbol{b}$是相等的,记作$\boldsymbol{a}=\boldsymbol{b}$.\\

\sj
\defination[零向量的定义]
零向量的模长为0,方向可以为任意方向。用$\boldsymbol{0}$或者$\vec{0}$表示。
\subsection{向量的线性运算}
向量的线性运算分为加法运算、减法运算以及数乘运算。
\\ 1.$\,$加法运算法则\\

\sj\sj
\defination[向量的加法]
平行移动使向量$\boldsymbol{a},\boldsymbol{b}$有共同起点,然后由\highlight{red}{\index{PXSBXFZ@平行四边形法则}平行四边形法则},以$\boldsymbol{a},\boldsymbol{b}$为相邻两边作平行四边形,将平行四边形的对角线所形成的向量定义为\highlight{red}{\index{HXL@和向量}和向量}。
\par 或者由\highlight{red}{\index{SJXFZ@三角形法则}三角形法则},将$\boldsymbol{b}$的起点移至$\boldsymbol{a}$的终点(即首尾相连),然后和向量为$\boldsymbol{a}$的起点指向$\boldsymbol{b}$的终点。
\\ \textbf{加法的运算规律:}\\
(1)$\,$ \highlight{red}{\index{JFJHL@加法交换律}加法交换律} \qquad $\boldsymbol{a}+\boldsymbol{b}=\boldsymbol{b}+\boldsymbol{a} $\\
(2)$\,$ \highlight{red}{\index{JFJHL@加法结合律}加法结合律} \qquad ($\boldsymbol{a}+\boldsymbol{b})+\boldsymbol{c}=\boldsymbol{a}+(\boldsymbol{b}+\boldsymbol{c}) $\\
2.$\,$减法运算法则\\

\sj\sj
\defination[向量的差定义]
与向量$\boldsymbol{a}$模长相同而方向相反的向量叫做$\boldsymbol{a}$的\highlight{red}{\index{FXL@反向量}反向量},记作$-\boldsymbol{a}$,那么\highlight{red}{\index{XLDC@向量的差}向量$\boldsymbol{b},\boldsymbol{a}$的差}为:
\begin{equation}
	\boldsymbol{b}-\boldsymbol{a}=\boldsymbol{b}+(-\boldsymbol{a})
\end{equation}
可以理解为把向量$-\boldsymbol{a}$加到向量$\boldsymbol{b}$上。(即理解为加法运算的逆运算,其运算满足加法运算的所有规律)
\\ 3.$\,$数乘运算法则\\

\sj \sj
\defination[向量的数乘定义]
向量$\boldsymbol{a}$与实数$\lambda$的乘积记作$\lambda\boldsymbol{a}$,规定$\lambda\boldsymbol{a}$也是一个向量,它的模
\begin{equation}
	|\lambda\boldsymbol{a}|=|\lambda||\boldsymbol{a}|
\end{equation}
它的方向有三种情况:\\
(1)$\,$ 当$\lambda>0$时,其方向与$\boldsymbol{a}$相同;\\
(2)$\,$ 当$\lambda<0$时,其方向与$\boldsymbol{a}$相反;\\
(3)$\,$ 当$\lambda=0$时,$\lambda\boldsymbol{a}$为零向量,这时其方向可以是任意的。\\
\textbf{数乘的运算规律:}\\
(1)$\,$ \highlight{red}{\index{SCJHL@数乘交换律}数乘结合律} \qquad
$\lambda(\mu\boldsymbol{a})=\mu(\lambda\boldsymbol{a})=(\lambda\mu)\boldsymbol{a}$\\
(2)$\,$ \highlight{red}{\index{SCFPL@数乘分配律}数乘分配律} \qquad
$(\lambda+\mu)\boldsymbol{a}=\lambda\boldsymbol{a}+\mu \boldsymbol{a}$\qquad $\lambda(\boldsymbol{a}+\boldsymbol{b})=\lambda\boldsymbol{a}+\lambda\boldsymbol{b}$
\subsection{向量的内积、外积及混合积}
\noindent1.$\,$向量的内积\\

\sj\sj
\defination[向量的内积定义]
向量$\boldsymbol{a},\boldsymbol{b}$的内积为
\begin{equation}
	\boldsymbol{a}\cdot\boldsymbol{b}=|\boldsymbol{a}||\boldsymbol{b}|\cos\langle\boldsymbol{a},\boldsymbol{b}\rangle
\end{equation}
特别地,当向量$\boldsymbol{a},\boldsymbol{b}$中有一个为零向量,其内积为0.向量的\highlight{red}{\index{NJ@内积}内积}也称为\highlight{red}{\index{SLJ@数量积}数量积}或\highlight{red}{\index{DC@点乘}点乘}。\\
\textbf{点乘的运算规律:}\\
(1)$\,$\highlight{red}{\index{JHL@交换律}交换律}\qquad $\boldsymbol{a}\cdot\boldsymbol{b}=\boldsymbol{b}\cdot\boldsymbol{a}$\\
(2)$\,$\highlight{red}{\index{YSCDJHL@与数乘的结合律}与数乘的结合律}\qquad
$(\lambda\boldsymbol{a})\cdot\boldsymbol{b}=\lambda(\boldsymbol{a}\cdot\boldsymbol{b})$\\
(3)$\,$\highlight{red}{\index{FPL@分配律}分配律}\qquad
$(\boldsymbol{a}+\boldsymbol{b})\cdot\boldsymbol{c}=\boldsymbol{a}\cdot\boldsymbol{c}+\boldsymbol{b}\cdot\boldsymbol{c}$\\
\textbf{内积的重要应用:}
\begin{equation}
	\boldsymbol{a}\perp\boldsymbol{b}\Leftrightarrow \boldsymbol{a}\cdot\boldsymbol{b}=0 
\end{equation}
2.$\,$向量的外积\\

\sj\sj
\defination[向量的外积定义]
向量$\boldsymbol{a}$,$\boldsymbol{b}$的外积$\boldsymbol{c}$为
\begin{equation}
	|\boldsymbol{c}|=|\boldsymbol{a}\times\boldsymbol{b}|=|\boldsymbol{a}||\boldsymbol{b}|\sin\langle\boldsymbol{a},\boldsymbol{b}\rangle
\end{equation}
其中外积$\boldsymbol{c}$的方向根据右手法则确定,就是手掌立在$\boldsymbol{a}$,$\boldsymbol{b}$所在平面的向量$\boldsymbol{a}$上,掌心向$\boldsymbol{b}$,那么大拇指方向就是垂直于该平面的方向,即为外积的方向。向量的\highlight{red}{\index{WJ@外积}外积}也称为\highlight{red}{\index{XLJ@向量积}向量积}或\highlight{red}{\index{XLDCC@向量的叉乘}向量的叉乘}。\\
\textbf{外积的几何意义:}
\begin{equation}
	|\boldsymbol{a}\times\boldsymbol{b}|=|\boldsymbol{a}||\boldsymbol{b}|\sin \theta
\end{equation}
其中,$|\boldsymbol{a}\times\boldsymbol{b}|$是以$|\boldsymbol{a}|,|\boldsymbol{b}|$为邻边的平行四边形面积。\\
\textbf{外积的运算规律:}\\
(1)$\,$\highlight{red}{\index{FJHL@交换律}反交换律}\qquad $\boldsymbol{a}\times\boldsymbol{b}=\boldsymbol{b}\times\boldsymbol{a}$\\
(2)$\,$\highlight{red}{\index{FPL@分配律}分配律}\qquad
$(\boldsymbol{a}+\boldsymbol{b})\times\boldsymbol{c}=\boldsymbol{a}\times\boldsymbol{c}+\boldsymbol{b}\times\boldsymbol{c}$\\
(3)$\,$\highlight{red}{\index{YSCDJHL@与数乘的结合律}与数乘的结合律}\qquad
$(\lambda\boldsymbol{a})\times\boldsymbol{b}=\boldsymbol{a}\times(\lambda\boldsymbol{b})=\lambda(\boldsymbol{a}\times\boldsymbol{b})$\\
\textbf{外积的重要应用:}
\begin{equation}
	\boldsymbol{a}\parallel\boldsymbol{b}\Leftrightarrow \boldsymbol{a}\times\boldsymbol{b}=0
\end{equation}
3.$\,$向量的混合积\\

\sj\sj
\defination[向量混合积的定义]
设已知三个向量$\boldsymbol{a}$,$\boldsymbol{b}$,$\boldsymbol{c}$,数量$(\boldsymbol{a}\times\boldsymbol{b})\cdot \boldsymbol{c}$称为这三个向量的\highlight{red}{\index{HHJ@混合积}混合积},记为$[\boldsymbol{a}\,\boldsymbol{b}\,\boldsymbol{c}]$.
\\ \textbf{混合积的几何意义:}
\par 如右图,设非零向量$\boldsymbol{a},\boldsymbol{b},\boldsymbol{c}$,那么记$\boldsymbol{d}=\boldsymbol{a}\times\boldsymbol{b}$,$\boldsymbol{a},\boldsymbol{b},\boldsymbol{c}$构成一个平行六面体,那么将$\boldsymbol{c}$分成与$\boldsymbol{d}$正交与平行的两个分向量$\boldsymbol{c_1},\boldsymbol{c_2}$.\vspace{-0.5em}
\begin{equation}
	|[\boldsymbol{a}\,\boldsymbol{b}\,\boldsymbol{c}]|=(\boldsymbol{a}\times\boldsymbol{b})\cdot\boldsymbol{c}=\boldsymbol{d}\cdot\boldsymbol{c}=\boldsymbol{d}\cdot(\boldsymbol{c_1}+\boldsymbol{c_2})=\boldsymbol{d}\cdot\boldsymbol{c_2}=|\boldsymbol{d}|\cdot|\boldsymbol{c_2}|=|\boldsymbol{a}\times\boldsymbol{b}|\cdot\boldsymbol{c_2}=S_{\text{底}}\times h=V\vspace{-0.5em}
\end{equation}
也就是说混合积$[\boldsymbol{a},\boldsymbol{b},\boldsymbol{c}]$的绝对值就是这三个向量$\boldsymbol{a},\boldsymbol{b},\boldsymbol{c}$所构成的平行六面体的体积。
\textbf{混合积的重要应用:}
\\ (1)$\,$求三个向量$\boldsymbol{a},\boldsymbol{b},\boldsymbol{c}$所构成的平行六面体的体积。
\\ (2)$\,$三个向量$\boldsymbol{a},\boldsymbol{b},\boldsymbol{c}$共面$\Leftrightarrow$ $[\boldsymbol{a},\boldsymbol{b},\boldsymbol{c}]=0$.
\section{向量的空间坐标}
\subsection{空间直角坐标系}
\defination[空间直角坐标系的定义]
空间任意选定一点$O$,过点$O$作三条互相垂直的数轴$Ox,Oy,Oz$,它们都以$O$为原点且具有相同的长度单位。这三条轴分别称作$x$轴(横轴),$y$(纵轴),$z$(竖轴),统称为坐标轴。它们的正方向符合右手规则,即以右手握住$z$轴,当右手的四个手指$x$轴的正向以$90\circ$转向$y$轴正向时,大拇指的指向就是$z$轴的正向。
\par 简单地讲,就是\textbf{一个原点,三个坐标轴,三个坐标面,八个卦限}。而数组$(x,y,z)$对应空间上的一点坐标。
\subsection{向量的空间坐标的表示与计算}
\defination[向量的空间坐标定义]
设向量$\boldsymbol{i},\boldsymbol{j},\boldsymbol{k}$分别为$x$轴,$y$轴,$z$轴的单位向量,那么对于空间点$A(x,y,z)$,$\boldsymbol{OA}$可以表示为
\begin{equation}
	\boldsymbol{OA}=x\boldsymbol{i}+y\boldsymbol{j}+z\boldsymbol{k}
\end{equation}
设$\boldsymbol{r}=\boldsymbol{OA}$,那么我们规定向量$\boldsymbol{r}$的坐标为$(x,y,z)$,记作$\boldsymbol{r}=(x,y,z)$.\\
1.$\,$向量的线性运算的坐标表示
\par 记向量$\boldsymbol{a}=(x_1,y_1,z_1)$,$\boldsymbol{b}=(x_2,y_2,z_2)$,那么\\
(1)$\quad \boldsymbol{a}+\boldsymbol{b}=(x_1+x_2,y_1+y_2,z_1+z_2)$\\
(2)$\quad \boldsymbol{a}-\boldsymbol{b}=(x_1-x_2,y_1-y_2,z_1-z_2)$\\
(3)$\quad \lambda \boldsymbol{a}=(\lambda x_1,\lambda y_1,\lambda z_1)$\\
2.$\,$向量的内积、外积的坐标表示\\
记向量$\boldsymbol{a}=(x_1,y_1,z_1).\boldsymbol{b}=(x_2,y_2,z_2)$,那么\\
\textbf{内积:}$\boldsymbol{a}\cdot\boldsymbol{b}=x_1x_2+y_1y_2+z_1z_2$\\
特别地,$\boldsymbol{a}^2=\boldsymbol{a}\cdot\boldsymbol{a}=x_1^2+y_1^2+z_1^2$,那么向量$\boldsymbol{a}$的模长为$|\boldsymbol{a}|=\sqrt{\boldsymbol{a}^2}=\sqrt{x_1^2+y_1^2+z_1^2}$\\

\noindent 那么由$\boldsymbol{a}\cdot\boldsymbol{b}=|\boldsymbol{a}||\boldsymbol{b}|\cos\langle \boldsymbol{a},\boldsymbol{b}\rangle$,得$\di\cos\langle \boldsymbol{a},\boldsymbol{b}\rangle=\frac{\boldsymbol{a}\cdot\boldsymbol{b}}{|\boldsymbol{a}||\boldsymbol{b}|}=\frac{x_1x_2+y_1y_2+z_1z_2}{\sqrt{x_1^2+y_1^2+z_1^2}\sqrt{x_2^2+y_2^2+z_2^2}}$\\

\noindent \textbf{外积:}$\boldsymbol{a}\times\boldsymbol{b}=(y_1z_2-z_1y_2)\boldsymbol{i}+(z_1x_2-x_1z_2)\boldsymbol{j}+(x_1y_2-y_1x-2)\boldsymbol{k}=
	\begin{bmatrix}
		\boldsymbol{i} & \boldsymbol{j} & \boldsymbol{k}\\
		x_1 & y_1 & z_1\\
		x_2 & y_2 & z_2\\
	\end{bmatrix}$\\
特别地,
\begin{equation}
	\boldsymbol{a}\parallel\boldsymbol{b}\Leftrightarrow \frac{x_1}{y_1}=\frac{y_1}{y-2}=\frac{z_1}{z_2}=
\end{equation}
\textbf{混合积:}
\begin{equation}
	\nonumber
	\boldsymbol{c}\cdot(\boldsymbol{a}\times\boldsymbol{b})=(x_0\boldsymbol{i}+y_0\boldsymbol{j}+z_0\boldsymbol{k})	
	\begin{bmatrix}
		\boldsymbol{i} & \boldsymbol{j} & \boldsymbol{k}\\
		x_1 & y_1 & z_1\\
		x_2 & y_2 & z_2\\
	\end{bmatrix}=x_0\boldsymbol{i}
	\begin{bmatrix}
	\boldsymbol{i} & \boldsymbol{j} & \boldsymbol{k}\\
	x_1 & y_1 & z_1\\
	x_2 & y_2 & z_2\\
\end{bmatrix}+y_0\boldsymbol{j}	
\begin{bmatrix}
\boldsymbol{i} & \boldsymbol{j} & \boldsymbol{k}\\
x_1 & y_1 & z_1\\
x_2 & y_2 & z_2\\
\end{bmatrix}+z_0\boldsymbol{k}	
\begin{bmatrix}
\boldsymbol{i} & \boldsymbol{j} & \boldsymbol{k}\\
x_1 & y_1 & z_1\\
x_2 & y_2 & z_2\\
\end{bmatrix}
\end{equation}
\begin{equation}
	\nonumber
	=\begin{bmatrix}
		x_0\boldsymbol{i}\cdot\boldsymbol{i} & x_0\boldsymbol{i}\cdot\boldsymbol{j} & x_0\boldsymbol{i}\cdot\boldsymbol{k}\\
	x_1 & y_1 & z_1\\
	x_2 & y_2 & z_2\\
	\end{bmatrix}+
\begin{bmatrix}
	y_0\boldsymbol{j}\cdot\boldsymbol{i} & y_0\boldsymbol{j}\cdot\boldsymbol{j} & y_0\boldsymbol{j}\cdot\boldsymbol{k}\\
	x_1 & y_1 & z_1\\
	x_2 & y_2 & z_2\\
\end{bmatrix}+
\begin{bmatrix}
	z_0\boldsymbol{k}\cdot\boldsymbol{i} & z_0\boldsymbol{k}\cdot\boldsymbol{j} & z_0\boldsymbol{k}\cdot\boldsymbol{k}\\
	x_1 & y_1 & z_1\\
	x_2 & y_2 & z_2\\
\end{bmatrix}=
\begin{bmatrix}
	x_0 & y_0 & z_0\\
	x_1 & y_1 & z_1\\
	x_2 & y_2 & z_2\\
\end{bmatrix}
\end{equation}
(注:$\boldsymbol{i}\cdot\boldsymbol{i}=1,\boldsymbol{i}\cdot\boldsymbol{j}=0,\boldsymbol{i}\cdot\boldsymbol{k}=0$其余类似可得,本质上就是利用三个坐标轴上的单位向量相乘)
