\chapter{向量与空间解析几何}
\section{向量及其基本运算}
\subsection{向量的概念}
\defination[向量的定义]
既有大小也有方向的量称为\highlight{red}{\index{XL@向量}向量}(或\highlight{red}{\index{SL@矢量}矢量})。通常我们用有向线段来表示向量。
\par 以$A$为\highlight{red}{\index{QSD@起始点}起始点}、$B$为\highlight{red}{\index{ZD@终点}终点}的有向线段所表示的向量记为$\boldsymbol{AB}$.有时也会用一个黑体字母来表示向量。例如$\boldsymbol{a}$,$\boldsymbol{b}$.向量$\boldsymbol{AB}$的长度表示\highlight{red}{\index{XLDDX@向量的大小}向量的大小},记作\highlight{red}{\index{XLDM@向量的模}向量的模}。
\par 向量的四要素:起始点、终点、方向、模。

\defination[自由向量的定义]
在实际研究中,我们只关心向量的方向和大小,称为\highlight{red}{\index{ZYXL@自由向量}自由向量}。\\

\sj
\defination[向量的夹角定义]
设有两个非零向量$\boldsymbol{a}$