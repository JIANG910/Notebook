\chapter{平面不可压缩势流理论}
\thispagestyle{empty}
\section{基本方程}
\subsection{引言}
理想流动的连续(质量)方程和欧拉(动量)方程:
\begin{equation*}
	\dfrac{\D \rho}{\D t} + \rho (\nabla \bm{V}) = 0 \qquad \bm{f} - \dfrac{1}{\rho} \nabla p = \dfrac{\D \bm{V}}{\D t}
\end{equation*}
在不可压缩的条件下,可以简化为
\begin{equation}
	\begin{cases}
		\, \nabla \bm{V} = 0\\[0.5em]
		\, \dfrac{\D \bm{V}}{\D t} = \bm{f} - \dfrac{1}{\rho} \nabla p
	\end{cases}
\end{equation}
\vspace*{0.5em}

\noindent
\begin{minipage}{0.4\linewidth}
	给定初始条件
	\begin{equation*}
		\begin{cases}
			\, \bm{V}(x,y,z,t_0)\\
			\, p(x,y,z,t_0)
		\end{cases}
	\end{equation*}
\end{minipage}
\begin{minipage}{0.6\linewidth}
	边界条件\vspace*{-0.5em}
	\begin{itemize}
		\item 物体表面:$\bm{V}_n = 0$\vspace*{-0.5em}
		\item 无穷远处:$\bm{V} = \bm{V}, p = p_\infty$
	\end{itemize}
\end{minipage}
\vspace*{1em}

\noindent
\begin{minipage}{0.4\linewidth}
	\noindent 求解困难:\vspace*{-0.5em}
	\begin{itemize}
		\item 存在非线性项\vspace*{-0.5em}
		\item 速度和压强耦合\vspace*{-0.5em}
		\item 物体(如飞行器)边界复杂
	\end{itemize}
\end{minipage}
\begin{minipage}{0.6\linewidth}
	\noindent 求解思路:\vspace*{-0.5em}
	\begin{itemize}
		\item 基本流动的速度位或流函数\vspace*{-0.5em}
		\item 对于简单边界条件,基本流动叠加即可。\vspace*{-0.5em}
		\item 对于复杂边界条件,基本流动叠加并利用数值方法求解。
	\end{itemize}
\end{minipage}
\vspace*{1em}

\noindent 求解途径:\vspace*{-0.5em}
\begin{itemize}
	\item 利用无旋条件简化方程(线性化)\vspace*{-0.5em}
	\item 解耦速度和压强(即分别求解速度和压强)\vspace*{-0.5em}
\end{itemize}
\vspace*{1em}


\subsection{位函数和流函数、叠加原理和边界条件}

\sssection[位函数]

位函数存在的条件是流场无旋:
\begin{equation*}
	\text{rot} \bm{V} = 0 \qquad \bm{\omega}_z = 0 \qquad \dfrac{\partial u}{\partial y} = \dfrac{\partial v}{\partial x}
\end{equation*}
即存在\dy[位函数]{WHS}$\d \phi = u \d x + v \d y$,其中
\begin{equation}
	\begin{cases}
		\, u = \dfrac{\partial \phi}{\partial x} \\[0.5em]
		\, v = \dfrac{\partial \phi}{\partial y}
	\end{cases}
\end{equation}
又由连续方程
\begin{equation}
	\dfrac{\partial u}{\partial x} + \dfrac{\partial v}{\partial y} = 0 \quad \Rightarrow \quad \dfrac{\partial^2 \phi}{\partial x^2} + \dfrac{\partial^2 \phi}{\partial y^2} = 0 \quad \Rightarrow \quad \nabla^2 \phi = 0
\end{equation}
得到速度位,即可求解速度分量;一系列的速度位等值线称为\dy[等位线]{DWX}。
\vspace*{1em}

\sssection[流函数]

由连续方程
\begin{equation*}
	\dfrac{\partial u}{\partial x} + \dfrac{\partial v}{\partial y} = 0 \quad \Rightarrow \quad \dfrac{\partial u}{\partial x} = -\dfrac{\partial v}{\partial y} = \dfrac{\partial (-v)}{\partial x}
\end{equation*}
即存在\dy[流函数]{LHS}$\d \psi = -v \d x + u \d y$,其中
\begin{equation}
	\begin{cases}
		\, v = -\dfrac{\partial \psi}{\partial x} \\[0.5em]
		\, u = \dfrac{\partial \psi}{\partial y}
	\end{cases}
\end{equation}
若流场无旋,则
\begin{equation}
	\dfrac{\partial u}{\partial y} - \dfrac{\partial v}{\partial x} = 0 \quad \Rightarrow \quad  \dfrac{\partial^2 \psi}{\partial y^2} + \dfrac{\partial^2 \psi}{\partial^2 x^2} = 0 \quad \Rightarrow \quad \nabla^2 \psi = 0
\end{equation}
得到流函数,即可求解速度分量;一系列的流函数等值线称为\dy[流线]{LX}。
\vspace*{1em}

\sssection[叠加原理]

拉普拉斯方程是线性方程,可以用\dy[叠加原理]{DJYL}求复合解。

如果多个位函数分别满足拉普拉斯方程,即
\begin{equation}
	\nabla^2 \phi_i = 0
\end{equation}
则这些位函数的线性组合也必定满足别满足拉普拉斯方程,即
\begin{equation}
	\phi = \sum_{i=1}^{n} a_i \phi_i \quad \Rightarrow \quad \nabla^2 \phi = \sum_{i=1}^{n} a_i \nabla^2 \phi_i
\end{equation}
由于速度分量和位函数也是线性关系,则速度分量也满足叠加原理,即
\begin{equation}
	\begin{cases}
		\, \displaystyle u = \dfrac{\partial \phi}{\partial x} = \sum_{i = 1}^{n}a_i \dfrac{\partial \phi_i}{\partial x} = \sum_{i=1}^n a_iu_i\\[1em]
		\, \displaystyle v = \dfrac{\partial \phi}{\partial y} = \sum_{i = 1}^{n}a_i \dfrac{\partial \phi_i}{\partial y} = \sum_{i=1}^n a_iv_i
	\end{cases}
\end{equation}

\sssection[边界条件]
\begin{equation*}
	\mbox{\dy[边界条件]{BJTJ}}
	\, 
	\begin{cases}
		\, \mbox{\blue[外边界条件](足够远处)}\\
		\, \mbox{\blue[内边界条件](物体表面)}
	\end{cases}
\end{equation*}

\noindent 对于空气动力学问题,已知速度位$\phi$,则
\begin{itemize}
	\item 外边界条件(无穷远处):
	\begin{equation}
		\dfrac{\partial \phi}{\partial x} = V_\infty \quad \dfrac{\partial \phi}{\partial y} = \dfrac{\partial \phi}{\partial z} = 0
	\end{equation}
	\item 内边界条件:(物体表面法向速度分量为零)
	\begin{equation}
		\dfrac{\partial \phi}{\partial \bm{n}} = 0
	\end{equation}
\end{itemize}
在边界条件下,拉普拉斯方程的解唯一。

\subsection{位函数和流函数的性质及其相互联系}

\sssection[位函数的性质]
\vspace*{-1em}
\begin{enumerate}[\hspace*{1.5em} (1) ]
	\item 有无旋流动定义得到;位函数值可以相差任意常数;\vspace*{-0.5em}
	
	\item 对于理想不可压无旋流动,位函数满足拉普拉斯方程和叠加原理;\vspace*{-0.5em}
	
	\item 位函数沿某一方向的偏导数等于该方向的速度分量;($\bm{s}$为速度切向)
	\begin{equation}
		\begin{aligned}
			v_s &= u \cos(\bm{x},\bm{s}) + v \cos (\bm{y}, \bm{s}) \\
			& = \dfrac{\partial \phi}{\partial x} \cdot \dfrac{\d x}{\d s} + \dfrac{\partial \phi}{\partial y}\cdot \dfrac{\d y}{\d s} \\
			&= \dfrac{\partial \phi}{\partial \bm{s}}
		\end{aligned}
	\quad \Rightarrow \quad v_s = \dfrac{\partial \phi}{\partial \bm{s}}
	\end{equation}

	\item 速度位函数沿着流线方向增加
	\begin{equation}
			\begin{aligned}
			\d \phi &= \dfrac{\partial \phi}{\partial x}\, \d x + \dfrac{\partial \phi}{\partial y}\d y\\
			& = u\, \d x + v \, \d y\\
			& = \bm{V}\cdot \d \bm{s}
		\end{aligned} 
	\quad \Rightarrow \quad \d \phi = \bm{V}\cdot \bm{s}
	\end{equation}

	\item 速度位函数值相等的点连成的线称为\dy[等位(势)线]{DWX},与速度方向垂直;
	\begin{equation}
		\begin{cases}
			\, \d \phi = 0\\
			\, \d \phi = \bm{V} \cdot \bm{s}
		\end{cases}
		\quad \Rightarrow \quad 
		\bm{V}  \d \bm{s}
	\end{equation}
	
	\item 任意两点的速度线积分等于这两点的速度位函数之差;速度线积分与路径无关,仅决定于两点的位置;对封闭曲线,速度环量为零。
	\begin{align}
		\int_A^B \bm{V}\cdot \d \bm{s} & = \int_A^B (u\,d x + v\,d y + w \, \d z) \notag \\[0.5em]
		& = \int_A^B \, \d \phi = \phi_B - \phi_A
	\end{align}
\end{enumerate}

\sssection[流函数的性质]
\vspace*{-1em}
\begin{enumerate}[\hspace*{1.5em}(1) ]
	\item 
\end{enumerate}



















