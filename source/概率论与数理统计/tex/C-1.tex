\chapter{随机事件}
\thispagestyle{empty}
\section{随机事件}
\subsection{随机现象}
\dy[确定性现象]{QDXXX}
在一定条件下必然出现的结果\jg\\
\dy[随机现象]{SJXX}
事先无法准确与之其结果的现象\jg
\subsection{随机现象的统计性规律}
\dy[统计规律性]{TJGLX}
随机现象在大量重复出现时所表现出来的规律性.\jg\\
\dy[随机试验]{SJSY}
对随机现象的观察.\jg\\
\dya[随机试验的特点]
\begin{enumerate}[1.]
	\setlength{\itemindent}{3em}
	\setlength{\topsep}{0.01em}
	\setlength{\itemsep}{0.01em}
	\item 可重复性
	\item 可观察性
	\item 随机性
\end{enumerate}


\subsection{样本空间}
\dy[样本点]{YBD}
随机试验的每一个可能结果.\jg\\
\dy[样本空间]{YBKJ}
样本点的全体.\jg

\subsection{随机事件}
\dy[事件]{SJ}
实验结果具备的某一可观察的特征.\jg\\
\dy[随机事件]{SJSJ}
在随机试验中可能发生也可能不发生.\jg\\
\dy[必然事件]{BRSJ}
在试验中必然发生.\jg\\
\dy[不可能事件]{BKNSJ}
在试验中一定不发生.\jg\\
\dy[基本事件]{JBSJ}
对应一个唯一的可能结果,即样本点.\jg

\subsection{事件的集合表示}

\subsection{事件建的关系和运算}
\dy[事件的包含]{SJDBH}
$A$发生必然导致$B$发生,则称事件$B$包含事件$A$,记作$B\supset A$或$A\subset B$.\jg\\
\dy[事件的相等]{SJDXD}
事件$A$包含事件$B$,事件$B$也包含事件$A$,则称事件$A$与$B$相等,记作$A=B$.\jg\\
\dy[事件的并(或和)]{SJDB}
``事件$A$与$B$至少有一个发生"这一事件称为事件$A$和$B$的并(或和),记作$A\cup B$或$A+B$.\jg\\
\dy[事件的交(或积)]{SJDJ}
``事件$A$与$B$都发生"这一事件称为事件$A$与$B$的交(或积),记作$A\cap B$.\jg\\
\dy[事件的差]{SJDC}
``事件$A$发生而$B$不发生"这一事件称为事件$A$和$B$的差,记作$A-B$.\jg\\
\dy[互不相容事件]{HBXRSJ}
若事件$A$与$B$不能同时发生,也就是说$AB$时不可能事件,即$AB=\varnothing$,则称事件$A$与$B$是不可能事件.\jg\\
\dy[对立事件]{DLSJ}
``事件$A$不发生"这一事件称为事件$A$的对立事件,记作$\overline{A}$,易见,$\overline{A}=\Omega -A$,且$\overline{(\overline{A})}$.\jg\\
\dya[有限个事件的并与交]\jg
\newpage 
\noindent\dy[完备事件组]{WBSHZ}
\par 完备事件组设$A_1,A_2,\cdots,A_n,\cdots$是有限或可数个事件,如果其满足
\par \quad (1)  $A_iA_j=\varnothing,i\ne j,\quad i,j=1,2,\cdots $
\par \quad (2)  $\bigcup\limits_iA_i=\Omega$
\par 则称$A_1,A_2,\cdots,A_n,\cdots$是一个完备事件组.\jg\\
\dya[事件的关系与运算的文氏图]\jg

\subsection{随机事件的运算律}
\dya[求和运算]\jg
\par \quad 交换律
\begin{equation}
A \cup B =B \cup A
\end{equation}
\par \quad 结合律
\begin{equation}
(A \cup B)\cup C =A\cup (B\cup C)=A\cup B\cup C
\end{equation}
\dya[求交运算]\jg
\par \quad 交换律
\begin{equation}
A\cap B=B\cap A
\end{equation}
\par \quad 结合律
\begin{equation}
(A \cap B)\cap C=A\cap (B\cap C)=A \cap B \cap C
\end{equation}
\dya[混合运算]\jg
\par \quad 第一分配律
\begin{equation}
A \cap (B \cup C)=(A\cap B)\cup (A \cap C)
\end{equation}
\par \quad 第二分配律
\begin{equation}
A \cup (B \cap C)=(A\cup B)\cap (A \cup C)
\end{equation}
\dya[求对立事件的运算]\jg
\par \quad 自反律
\begin{equation}
\overline{(\overline{A})}=A
\end{equation}
\dya[求和及交事件的对立事件]\jg
\par \quad 第一对偶律
\begin{equation}
\overline{A \cup B}=\overline{A} \cap \overline{B} 
\end{equation}
\par \quad 第二对偶律
\begin{equation}
\overline{A \cap B}=\overline{A} \cup \overline{B} 
\end{equation}

\section{随机事件的概率}
\subsection{概率及其频率解释}
参见$\rm{P}_9$
\subsection{从频率的性质看概率的性质}
参见$\rm{P}_{10}$
\subsection{概率的公理化定义}
\sj
\defination[概率公理化]
设$\Omega $是一个样本空间,定义在$\Omega $的事件域$F$上的一个实值函数$P(\cdot)$如果它满足下列三条公理:
\begin{enumerate}[1.]
	\setlength{\itemindent}{4em}
	\setlength{\topsep}{0.01em}
	\setlength{\itemsep}{0.01em}
	\item $P(\Omega )=1$
	\item 对任意事件$A$,有$P(A) \le 0$
	\item 对任意可数的两两不相容的事件$A_1,A_2,\cdots,A_n,\cdots$,有$\displaystyle P\left( \bigcup_{i=1}^{\infty } A_i\right)=\sum_{i=1}^{\infty }P(A_i) $
\end{enumerate}
则称实值函数$P(\cdot)$为$\Omega $上的一个概率测度.\jg

\subsection{概率测度的性质}
\begin{enumerate}[1.]
	\setlength{\itemindent}{4em}
	\setlength{\topsep}{0.01em}
	\setlength{\itemsep}{0.01em}
	\item $P(\varnothing)=0$
	\item 有限可加性:$\displaystyle P\left( \bigcup_{i=1}^{\infty } A_i\right)=\sum_{i=1}^{\infty }P(A_i) $
	\item $P(\overline{A})=1-P(A)$
	\item $P(A-B)=P(A)-P(AB)=P(B)-P(AB)$
	\item $0\le P(A) \le 1$
	\item $P(A\cup B)=P(A)+P(B)-P(AB)$
\end{enumerate}

\section{古典概型与集合概型}
\subsection{古典概型}
\tdefination[古典概型]
古典概型是满足下面两个假设条件的概率模型:
\begin{enumerate}[1.]
	\setlength{\itemindent}{4em}
	\setlength{\topsep}{0.01em}
	\setlength{\itemsep}{0.01em}
	\item 随机试验只有有限个结果
	\item 每一个可能记过发生的概率相同
\end{enumerate}
所以,古典概型的概率测度可表述为:
\begin{equation}
P(A)=\frac{A\mbox{中的元素个数}}{\Omega \mbox{中的元素个数}}=\frac{\mbox{使}A\mbox{发生的基本事件数}}{\mbox{基本事件总数}}
\end{equation}

\subsection{几何概型}
\tdefination[几何概型]
几何概型的概率测度可表述为
\begin{equation}
P(A)=\frac{S(A)}{S(\Omega)}
\end{equation}

\section{条件概率}
\subsection{条件概率的定义}
\tdefination[条件概率]
给定概率空间$\Omega,P$,$A,B$是其上的两个事件,且$P(A)>0$,则称$\displaystyle P(B|A)=\frac{P(AB)}{P(A)}$为已知事件$A$发生的条件下,事件$B$发生的条件概率.

\subsection{乘法公式}
\ttheorem[乘法公式]
乘法公式的两个形式:
\begin{equation}
P(AB)=P(A)\cdot P(B|A),\,P(A)>0
\end{equation}
\begin{equation}
P(AB)=P(B)\cdot P(A|B),\,P(B)>0
\end{equation}

\subsection{全概率公式}
\ttheorem[全概率公式]
设$\lbrace A_i \rbrace$是一列有限或可数无穷个两两不相容的非零概率事件,且$\bigcup\limits_{i}A_i=\Omega $,则对任意事件$B,P(B)>0$,有
\begin{equation}
P(B)=\sum\limits_{i}P(A_i)\cdot P(B|A_i)
\end{equation}

\subsection{贝叶斯公式}
\ttheorem[贝叶斯公式]
设$\lbrace A_i \rbrace$是一列有限或可数无穷个两两不相容的非零概率事件,且$\bigcup_{i=1}^{\infty}A_i=\Omega $,则对任意事件$B,P(B)>0$,有
\begin{equation}
P(A_i|B)=\frac{P(A_iB)}{P(B)}=\frac{P(A_i)\cdot P(B|A_j)}{\sum\limits_{j}P(A_j)\cdot P(B|A_j)}
\end{equation}

\section{事件的独立性}
\subsection{两个事件的独立性}
\dy[两个事件的独立性]{LGSJDDLX}
如果$P(AB)=P(A)P(B)$,则称$A$与$B$相互独立,简称$A$与$B$独立.\jg\\
\dy[有限个事件的独立性]{YXGSJDDLX}
(1)  如果有$n(n\le 2)$个事件:$A_i,A_2,\cdots,A_n$中任意两个使劲按均相互独立,即对任意$1\le i\le j \le n$,均有$P(A_iA_j)=P(A_i)P(A_j)$,则称$n$个事件$A_i,A_2,\cdots,A_n$两两独立.
\par (2)  设$A_i,A_2,\cdots,A_n$为$n(n\le 2)$个事件,如果对其中任何$k(2\le k\le n)$个事件$A_{i_1},A_{i_2},\cdots,A_{i_k}\,(1 \le i_1<i_2<\cdots<i_k\le n)$,均有$P(A_{i_1}A_{i_2}\cdots A_{i_k})=P(A_{i_1})P(A_{i_2})\cdots P(A_{i_k})$,则称
事件$A_i,A_2,\cdots,A_n$为$n(n\le 2)$相互独立.

\subsection{相互独立性的性质}
\ttheorem[相互独立性的性质]
1.  如果$n$个事件$A_1,A_2,⋯,A_n$相互独立,则将其中任何$m(1\leq m \leq n)$个事件改为相应的对立事件,形成的新的$n$个事件仍然相互独立.
\par 2.  如果$n$个事件$A_1,A_2,⋯,A_n$相互独立,则有
\begin{equation}
	P\left( \bigcup_{i=1}^{n} A_i\right) =1-\prod_{i=1}^{n}P\left(\overline{A_i} \right) =1-\prod_{i=1}^{n}\left[1- P\left(A_i \right)\right]
\end{equation}

\subsection{伯努利概型}
\tdefination[伯努利概型]
只有两个可能的结果的试验称为伯努利试验,一个伯努利试验独立重复$n$次形成的试验序列称为$n$重伯努利试验.
\jg

\theorem[伯努利定理]
在一次试验中,事件$A$发生的概率为$p(0<p<1)$,则在$n$重伯努利试验中,事件$A$恰好发生$k$次的概率$b(k;n,p)$为
\begin{equation}
b(k;n,p)=C_n^k\,p^k\,q^{n-k}
\end{equation}
其中,$q=1-p.$
\par 在伯努利试验序列中,设每次试验中事件$A$发生的概率为$p$,“事件$A$在第$k$次试验中才首次发生”$(k≥1)$这一事件的概率为
\begin{equation}
g(k,p)=p\,q^{k-1}
\end{equation}



